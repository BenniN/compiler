\chapter{Usage}

\section{Invocation}
The inter module checker is invoked with
\begin{verbatim}
imc -b <target> -o <output> pearlModules...
\end{verbatim}

The option -b is  recommended. The given parameter defines the
platform definition XML file, which is expected ether in the current 
working directory (first)
or in installation directory of OpenPEARL.

The option -o defaults to system.cc in the current working directory.

The list of pearl modules is passed without extension. Pathnames are 
possible.


\section{Tests}
\subsection{Tests for System Part}
The following tests are performed for system part elements:
\begin{itemize}
\item all defined username need a predefined system name.
\item the number and type of actual parameters must fit to the 
      predefined system element
\item the value of the actual parameters are checked, if the predefined
   elements limits the parameter range
\item each used association must be supported by the specified provider
\item actual parameters of association providers must fit in type and value
   to the formal parameters, like username definitions
\item the number of association clients must agree with the supported
    number of clients
\item the number of instances of a system element does not excess the 
   allowed number of instances
\end{itemize}

\subsection{Tests for Problem Part}
The following tests are performed for problem part elements:
\begin{itemize}
\item each specified system dation, signal and interrupt must fit
  to the definition is the system part.
\end{itemize}

\subsection{Not implemented yet}
Not supported up to now:
\begin{itemize}
\item problem part global declarations and their specifications
\end{itemize}I

\section{Output}
The tool creates a C++ file for the system part if no errors are
detected. 
This file must be passed to the g++ compiler and linked with all 
other object modules.

