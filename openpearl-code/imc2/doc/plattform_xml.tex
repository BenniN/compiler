\chapter{Platform Specific Names}

\section{Introduction}
Each specific platform may support a different set of system names.
Thus each target platform is characterized by a separate xml-file.
These files are created during the build phase of the runtime system
for each target platform.

\section{Structure of the XML-File}
All possible system elements are located as unordered separate 
elements under the root tag \verb|platform|.
For the individual types of system elements, the following tags are
provided:
\begin{itemize}
\item \verb|<signal ...>| defines a system signal
\item \verb|<interrupt ...>| defines a system interupt
\item \verb|<dation ...>| defines a system dation
\item \verb|<configuration ...>| defines a configuration item
\item \verb|<connection ...>| defines a association provider, if 
   non of the above fit
\end{itemize}

These elements have attributes 
\begin{description}
\item [name] is the system name
\item [instances] the number of instances, if specified. If this attribute 
   is not specified, there is no instance limit.
\end{description}

\subsection{Signal-Tag}
The signal tag contains only the attribute \verb|name| for the
system signal name.

\subsection{Interrupt-Tag}
The interrupt tag contains the attribute \verb|name| for the
system interrupt name.
Possible parameters are listed as subtree \verb|<parameters>|.

\subsection{Dation-Tag}
The dation tag contains the attribute \verb|name| for the
system dation name.
Possible parameters are listed as subtree \verb|<parameters>|.
Possible attributes are listed as subtree \verb|<attributes>|.
Possible associations to this element are
 listed as subtree \verb|<associationProvider>|.


\subsection{Configuration-Tag}
The configuration element has the same structure as the dation tag
except the attribute \verb|name|, which has not to be present.

\subsection{Associations}
The connection between two system elements is treated by the following
elements:

\begin{itemize}
\item \verb|<needAssociation>| with the attribute 
 \verb|name| and the identifier of the expected interface. The 
  interface names are described in the runtime documentation. 
\item \verb|<associationProvider>|  with elements \verb|<associationType>|.
  Attributes for an associationType are:
  \begin{itemize}
   \item  \verb|<name>|,  which is an unique free selectable
     identifier for this kind of  association.
   \item \verb|<clients>|  if given it is the number of allowed
      clients for this connection provider. If not given, there is no
      limits of allowed clients.
   \end{itemize} 
\end{itemize}

\subsection{Parameters}
Only parameters of system elements of types bit, fixed and char are supported.
The actual value may be restricted to definite values and ranges.

Tags are defined for each of them:
\begin{itemize}
\item \verb|BIT| with the attribute \verb|length|, which specifies the 
   allowed length of an actual parameter. \verb|<BIT length="8">...</BIT>|
   specifies a BIT(8) parameter. Shorter actual parameters are extended --- 
   longer parameters induce an error message.
\item \verb|FIXED| with the attribute \verb|length|, which specifies the 
   allowed length of an actual parameter. \verb|<FIXED length="15">...</FIXED>|
   specifies a FIXED(15) parameter. Shorter actual parameters are extended --- 
   longer parameters induce an error message.
\item \verb|CHAR| with the attribute \verb|length|, which specifies the 
   allowed length of an actual parameter. \verb|<CHAR length="15">...</CHAR>|
   specifies a CHAR(15) parameter. Shorter actual parameters are allowed --- 
   longer parameters induce an error message.
\end{itemize}

Each parameter may have an attribute 'nick' which allows to access 
the concrete value of the parameter in expressions of subsequent parameters
or in the 'data'-attribute. The nicknames consist of letters ('A'-'Z', 
'a'-'z') only. To access a nickname, ar '\$' must be used as prefix.
Expressions are supported for integer results. The Java Script machine,
which is used to evaluate the expressions, operates on double values ---
thus the division must be used with care.
Expressions must be placed in square brackets.
e.g. \verb|anyText[$start+3]anyOtherText|.

Parameters may be checked against rules. The list of rules may be extended
in the class TargetPlattformXml, if other rules are required.

Supported rules:

\begin{tabular}{|l|c|p{8cm}|}
\hline
rule & type & description \\
\hline
VALUES & BIT,FIXED, CHAR &
   a comma separated list of allowed values
   is given. Other values produce an error message. 
No nicknames or expressions are allowed.\\
\hline
ConsistsOf & CHAR &
a comma separeted list of allowed values is given. 
The actual parameter must consist of only one or more of these
elements.
No nicknames or expressions are allowed.\\
\hline
NotEmpty & CHAR & 
   at least one character is required \\   
\hline
FIXEDRANGE & FIXED &
   two comma separated integers or expressions with nicknames
 limits the allowed range
   of actual values --- borders included \\ 
\hline
FIXEDGT & FIXED &
   one integer or expression with nicknames
 limits the allowed range at the low side
   of actual value --- border NOT included \\ 
\hline
\end{tabular}

\section{Sample Platform File}
\begin{XMLCode}
<?xml version="1.0" encoding="UTF-8"?>
<platform file="testPlatform.xml">
   <signal name="FixedRangeSignal"/>
   <signal name="FixedDivideByZeroSignal"/>
   <signal name="FloatIsNaNSignal"/>
   <interrupt name="UnixSignal">
      <parameters>
         <FIXED length="31">
            <VALUES>1,2,3,15,16,17</VALUES>
         </FIXED>
      </parameters>
   </interrupt>
   <dation name="Disc">
      <parameters>
         <CHAR length="32767"/>	<!-- beliebiger String -->
         <FIXED length="31">
            <FIXEDRANGE>1,9999</FIXEDRANGE>
         </FIXED>
      </parameters>
      <attributes>FORWARD, DIRECT, IN, OUT, INOUT, SYSTEM</attributes>
      <data>ALL</data>
      <associationProvider>
         <associationType name="NamedAlphicOutProvider" />
      </associationProvider>
   </dation>
   <dation name="StdIn" instances="1">
      <parameters>
      </parameters>
      <attributes>
         FORWARD, IN, SYSTEM
      </attributes>
      <DATA>ALPHIC</DATA>
   </dation>
   <dation name="StdOut" instances="1">
      <parameters>
      </parameters>
      <attributes> FORWARD, OUT, SYSTEM</attributes>
      <data>ALPHIC</data>
      <associationProvider>
         <associationType name="AlphicOutProvider" />
      </associationProvider>
   </dation>
   <connection name="I2CBus">
      <parameters>
         <CHAR length="32767">
            <NotEmpty />
         </CHAR>
         <FIXED length="31">	<!-- max int32_t -->
            <FIXEDGT>0</FIXEDGT>
         </FIXED>
      </parameters>
      <associationProvider>
         <associationType name="I2CBusProvider" />
      </associationProvider>
   </connection>
   <dation name="LM75">
      <parameters>
         <BIT length="8">
            <VALUES>'48'B4, '49'B4, '4A'B4, '4B'B4,
                    '4C'B4, '4D'B4 , '4E'B4, '4F'B4</VALUES>
         </BIT>
      </parameters>
      <needAssociation provider="I2CBusProvider"/>
   </dation>
   <dation name="Pcf8574">
      <parameters>
         <BIT length="8">
            <VALUES>'20'B4, '21'B4, '22'B4, '23'B4,
                    '24'B4, '25'B4 , '26'B4, '27'B4,
                    '38'B4, '39'B4 , '3A'B4, '3B'B4,
                    '3C'B4, '3D'B4 , '3E'B4, '3F'B4
            </VALUES>
         </BIT>
         <FIXED length="7" nick="start">
            <FIXEDRANGE>0,7</FIXEDRANGE>
         </FIXED>
         <FIXED length="7" nick="width">
            <FIXEDRANGE>1,[$start+1]</FIXEDRANGE>
         </FIXED>
      </parameters>
      <attributes>
         BASIC, SYSTEM, IN, OUT
      </attributes>
      <data>BIT($width)</data>
      <needAssociation provider="I2CBusProvider"/>
   </dation>
  <configuration name="Log" instances="1">
    <parameters>
      <CHAR length="4">
         <NotEmpty/>
      </CHAR>
    </parameters>
    <needAssociation provider="AlphicOutProvider"/>
  </configuration>
</plattform>
\end{XMLCode}
