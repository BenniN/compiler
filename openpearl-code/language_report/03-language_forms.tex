\chapter{Rules for the Construction of PEARL Language Forms}   % 3

A PEARL program can be written free of format; particularly, it is not
necessary to pay attention to the fact that a statement starts in a
certain column.

All elements of a PEARL program are produced from characters of the
following character set. Certainly, character string constants and
comments may contain any character that is allowed by ISO 646 and
national variants of ISO 646.

\section{Character Set}   % 3.1

The character set of PEARL is a partial set of the ISO 7 bit character
set (ISO 646). It contains

\begin{itemize}
\item the upper-case letters A to Z
\item the lower-case letters a to z
\item the digits 0 to 9, and
\item the special characters
      \begin{tabbing}
      \x \=   \kill
      \_ \> underline\\
  \> space (for better clearness sometimes, underline \_ is used)\\
      !  \> exclamation mark (start of the line comment)\\
      '  \> apostrophe\\
      (  \> open round parenthesis\\
      )  \> close round parenthesis\\
      ,  \> comma\\
      .  \> full stop\\
      ;  \> semicolon\\
      :  \> colon\\
      +  \> plus sign (e.g. for addition, algebraic sign)\\
      -  \> minus sign (e.g. for subtraction, algebraic sign)\\
      $\ast$  \> asterisk (e.g. for multiplication)\\
      /  \> oblique stroke (e.g. for division)\\
      =  \> equals sign (e.g. for assignment)\\
      $<$  \> less sign\\
      $>$  \> greater sign\\
      $[$  \> square bracket\\
      $]$  \> square bracket\\
      $\backslash$ \> backslash (for control signs in character strings)
      \end{tabbing}
\end{itemize}

The following character combinations are interpreted as an entity (a
compound symbol):

\begin{tabbing}
== \= exponentiation symbol \kill

:=    \> assignment symbol\\ \relax
**    \> exponentiation symbol\\
/*    \> start comment\\ \relax
*/    \> end comment\\
//    \> symbol for integer division\\
==    \> equals symbol\\
/=    \> not equals symbol\\
$<=$  \> greater or equal symbol\\
$>=$  \> less or equal symbol\\
$<>$  \> cyclic-shift symbol\\
$><$  \> concatenation symbol\\
$'\backslash$ \> start of a control character sequence in character string constants\\
$\backslash'$ \> end of a control character sequence in character string constants
\end{tabbing}

If not all symbols required for program notation are available on a
device, the following character sequences can be used alternatively:

\begin{tabbing}
CSHIFT \= for $<>$ \kill

LT     \> for $<$\\
GT     \> for $>$\\
EQ     \> for ==\\
NE     \> for /=\\
LE     \> for $<=$\\
GE     \> for $>=$\\
CSHIFT \> for $<>$\\
CAT    \> for $><$\\
(/     \> for $[$\\
/)     \> for $]$\\
\end{tabbing}
This feature is kept for backward compatibility.
%%%% added
The equality (==) and not equality (/=) operators are avoided
in this document due to possible misinterpretation 
with the divide operator in C and C++. The substitutes EQ and NE are 
used.

\section{Basic Elements}    % 3.2

A PEARL program is built up from the following basic elements:

\begin{itemize}
\item identifiers
\item constants
\item delimiters (meaning special characters and compound symbols), and
\item comments
\end{itemize}

Character sequences for identifiers and constants must be followed by
delimiters or comments.

\subsection{Identifiers}   % 3.2.1

Identifiers are used for constructing names of objects (e.g. numerical
variables, procedures). They consist of a sequence of letters
(upper-case or lower-case), the underline and/or numerals; this sequence
has to start with a letter. For identifiers, PEARL distinguishes between
upper-case and lower-case letters, i.e., valve, VALVE and Valve denote
different objects.

Examples: counter\_1, DISPO, wait

Some words have a specific meaning at prescribed positions in the PEARL
program; these words are called keywords. E.g., the words BIT or GOTO
are such keywords. The appendix contains a list of all keywords. They
must not be used as identifiers and have always to be written with
upper-case letters.

\subsection{Constants}   % 3.2.2

Constants are integer numbers, floating point numbers, bit strings,
character strings, times and durations. They are described in Chapter 5
together with the corresponding variables.

\subsection{Comments}    % 3.2.3

Comments are used to explain the program and are of no importance for
running of the program. There are two kinds of comments: One kind may
cover several lines and is put in brackets by the compound symbols ``/*''
and ``*/''. Within these brackets, any characters may occur except the
compound symbol ``*/'' for the end of the comment.

The other kind, the line of comment, starts with an exclamation
mark ``!'' and is terminated by the end of the line.

Comments may be inserted wherever spaces are allowed.

Examples:

/* This comment is not ended\\
\x by the end of the line */\\
! This comment is limited to 1 line

\section{Construction of Language Forms}  % 3.3

In the following chapters, the language forms allowed in PEARL are
described.  In order to make these descriptions exact and as compact as
possible, some formal possibilities are needed apart from the verbal
formulation:

Each language form has a name, by which it is defined with the help of
the (meta) symbol ``::='' :

NameOfTheLanguageForm ::=\\
\x DefinitionOfTheLanguageForm

Example:

UpperCaseLetter ::=\\
\x A or B or ... or Z

Digit ::=\\
\x 0 or 1 or ... or 9

As this example shows, the definition of a language form may contain
elements that are given alternatively when constructing the language
form. To shorten that, in the following the alternatively possible
elements are divided by the symbol ``$\mid$'':

Example:

Letter ::=\\
\x A $\mid$ B $\mid$ ... $\mid$ Z $\mid$ a $\mid$ b $\mid$ ... $\mid$ z

Digit ::=\\
\x 0 $\mid$ 1 $\mid$ 2 $\mid$ ... $\mid$ 9

If one element shall occur on different occasions, but at least once, it
is to be provided with three superscribed dots.

Example:

SimpleInteger ::=\\
\x Digit $^{...}$

To express that an element may be missing while constructing the
language form, it is given in square brackets ``['' ... ``]'' (in case
of doubt, the square brackets are printed boldly).

Example:

Identifier ::=\\
\x Letter [ \{ Letter $\mid$ Digit $\mid$ \_ \} $^{...}$ ]

Here, already two further rules were used: The definition of a language
form may again contain names of language forms; furthermore the braces
and square brackets are also used to put together elements to new
elements. Thus, the last example is equivalent to the following:

Example:

Identifier ::=\\
\x Letter [ Letter $\mid$ Digit $\mid$ \_ ] $^{...}$

For the description of lists whose elements are separated by a certain
symbol, the list element and the delimiter are given in the form

ListElement [ Delimiter ListElement ] $^{...}$

to define the corresponding language form.

Example:

IdentifierList ::=\\
\x Identifier [ , Identifier ] $^{...}$

For a better understanding or a more exact description of the definition
of a language form, elements are often provided with an explanatory or
restricting comment that is separated from the element by the symbol
\S .

Example:

DeviceList ::=\\
\x Identifier\S  Device [ , Identifier\S Device ] $^{...}$

