\chapter{Program Structure}   % 4

\section{Modules}  % 4.1
\label{sec_modules}

A PEARL program is constructed of one or several parts, so-called
modules, which are translated independently. Each module consists of a
system part and/or a problem part.

The general form of a PEARL program reads as follows:

%%%\begin{front}
%%%PEARL program ::=\\
%%%\x Module $^{...}$
%%%\end{front}

\input{Program.bnf}

%%%\begin{front}
%%%Module ::=\\
%%%\x {\bf MODULE} [ (Identifier\S  OfTheModule) ];\\
%%%\x \x \{ SystemPart [ ProblemPart ] \} $\mid$ ProblemPart\\
%%%\x {\bf MODEND};

\input{Module.bnf}

%%%SystemPart ::=\\
%%%\x {\bf SYSTEM}; [ UserNameDeclaration $^{...}$ ]
%%%\end{front}

\input{SystemPart.bnf}

\input{ProblemPart.bnf}

In the system part, the connections of the projected computer are
described with the elements of the technical process (sensors,
actuators, etc.) and the standard peripherals (keyboard, monitor,
printer, disks, tapes, etc.). The programmer can assign freely
selectable names to the entries of the interrupt controller and the
peripherals addressed in the I/O statements (in the problem part) to
refer to these (computer independent) names in the problem part.

In the problem part, the algorithm for solving the given automation
problem is described. For this, the programmer declares the following
objects:

\begin{itemize}
\item variables and constants for integers, floating point numbers, bit
strings, character strings, durations, times, references
\item labels
\item procedures for frequently occurring partial evaluations
\item tasks for the temporarily parallel execution of tasks
\item blocks for structuring procedures and tasks
\item interrupts
\item signals
\item synchronisation variables (Sema and Bolt variables) as well as
\item data stations and formats for input/output
\end{itemize}

The required statements are given in the procedures and tasks,
together with other ``local'' declarations which are only needed there.
In general, objects may not be used (in statements) until they are
declared.

Example:

{\bf MODULE};

{\bf SYSTEM};\\
\x description of the connections and introduction of names for the
peripherals

{\bf PROBLEM};\\
\x declaration of constants and variables\\
\x declaration of interrupts\\
\x declaration of data stations

\x declaration of a task\\
\x \x declaration of local constants and variables\\
\x \x statements

\x declaration of a procedure\\
\x \x declaration of local constants and variables\\
\x \x declaration of local procedures\\
\x \x statements

...

{\bf MODEND};

Objects are declared at module level, i.e., outside procedures and
tasks, or in procedures and tasks. Objects declared at module level are
known in the entire module and can be used or, if needed, changed by
each task and procedure of the module, when mentioning the identifier.
A declared object in the task or procedure is only known in the
respective task or procedure and can only be used or, if needed,
changed there.

\section{Declarations and Specifications} %  4.2
\label{sec_dcl_and_spc}

Objects are introduced by {\it declaration} or {\it specification}.

\subsection{Declaration (DCL)}  % 4.2.1

The declaration serves to introduce an object and its name, i.e., when
evaluating the declaration, memory space for the object is allocated, and
up from now, it can be accessed under the name given in the
declaration.

At the module level or in a procedure or task, an object may be declared
only once. If an identifier X is declared as object both at module level
and in a procedure or task, two objects are introduced: In the
respective procedure or task, identifier X refers to the object
(locally) declared there, outside the procedure or task it refers to the
object declared at module level (cf. section \ref{sec_block}, Block Structure).

Example:

\begin{tabbing}
P:~~ \= DECLARE x FLOAT xxx \=   \kill

{\bf PROBLEM}; \>            \> \\
   \> {\bf DECLARE x FLOAT}; \> ! 1st declaration at module level\\
   \> {\bf DECLARE x FIXED}; \> ! 2nd declaration at module level (wrong)\\
   \>                        \> \\
P: \> {\bf PROCEDURE};       \> \\
   \> {\bf DECLARE x FIXED}; \> ! declaration in procedure P (permitted)\\
   \> ...                    \> \\
   \> x := 3;                \> ! assignment to the local variable x\\
   \> ...                    \> \\
   \> {\bf END};             \> ! P \\
   \>                        \> \\
   \> x := 5;                \> ! assignment to variable x declared at module level\\
   \> ...                    \> \\
   \> {\bf END};             \> ! T \\
   \>                        \> \\
... \>                       \>
\end{tabbing}

The different declaration forms are treated with the various objects in
the subsequent chapters.

In procedures and tasks,
\begin{itemize}
\item tasks
\item interrupts
\item signals
\item synchronising variables
\item data stations, as well as
\item formats
\end{itemize}
must not be declared or specified.

Table~\ref{objekte} shows where which objects may be declared or
specified.

\begin{table}
\caption{Permissibility of declarations\label{objekte}}
\vspace{5mm}

\begin{tabular}{|l|c|c|c|c|}
\hline
                       & \multicolumn{4}{|c|}{declaration possible on} \\
object                 & module level & task level & procedure level & block level \\ \hline
variable, constant     & x            & x    & x         & x  \\
label                  & --           & x    & x         & x  \\
%%%%%procedure              & x            & x    & x         & -- \\
procedure              & x            & --    & --         & -- \\
task                   & x            & --   & --        & -- \\
block                  & --           & x    & x         & x  \\
Sema, Bolt variable    & x            & --   & --        & -- \\
data station, format   & x            & --   & --        & -- \\
type                   & x            & x    & x         & x \\
\hline
\end{tabular}
\end{table}

\subsection{Specification (SPC) and Identification (SPC IDENT)}  % 4.2.2

With a specification, an already declared object is
referred to.  This is meaningful for objects which are declared in a
module, and which shall be used in other modules (cf. \ref{sec_references_module},
 References
between Modules), but also for introducing additional names for already
declared objects in general.
\newpage

Example:

Object c of type FIXED (15) shall get xx as 2nd name --- or formulated
in a different way: Object x shall be {\it identified} with name xx:

{\bf PROBLEM};\\
\x ...\\
\x {\bf DECLARE x FIXED};\\
\x ...\\
\x {\bf SPECIFY xx FIXED IDENT} (x);\\
\x ...\\
\x xx := 7; \x ! assignment to object x

In general, the form of identification reads as follows:

%%%\begin{front}
%%%Identification ::=\\
%%%\x \{ {\bf SPECIFY $\mid$ SPC} \} Identifier [ AllocationProtection ] Type IdentificationAttribute ;

\input{Identification.bnf}

\input{IdentificationAttribute.bnf}

The given type has to correspond to the type of the named object. More
details are defined when presenting the various objects.

\section{Block Structure, Validity of Objects}   % 4.3
\label{sec_block}

Blocks are used to structure task or procedure bodies and to influence
the scope and the life-span of PEARL objects. A block is a summary of
declarations and statements:

\input{Block.bnf}

Blocks are regarded as statements and may thus only occur in tasks and
procedures, but there even nested. The entry into a block takes
place when executing BEGIN. A block is left by the corresponding END or
by a branch to an statement outside the block, e.g., by the exit
statement (cf. 7.5). Jumps into a block are not allowed.

Within the blocks, no procedures may be declared!

Memory space is not allocated to the (local) objects declared in a block
until the block is entered; it is abandoned when leaving the block. Like
tasks, procedures and repetitions, blocks can introduce and remove
objects dynamically and thus provide the opportunity to use the
available memory space repeatedly.

Thus, some rules for the life-span and the scope range of these objects
have to be established:

The life-span of an object is the (processing) time between the
evaluation of its declaration and the execution of the end of the block
(or of the repetition, procedure, task or module), where the declaration
takes place.

% Klammern einfuegen !!!!!!!!!!!!!!!!!!!!!!!!!!!!!!!!!!!!!!!!!!!
%%%\begin{discuss}
%%%Das Systemgeraet PRINTER macht heutzutage wohl keinen Sinn mehr.
%%%Was soll damit dann passieren? Wie sollte so etwas in Linux aussehen?
%%%Wie soll so etwas bei Linux gehen? Drcukjob bei Programmende - aber das 
%%%endet ja nicht unbedingt ...
%%%\end{discuss}
%%%
%%%{\bf MODULE};\\
%%%\x {\bf SYSTEM};\\
%%%\x \x printer: STDPRINT;\\
%%%\x {\bf PROBLEM};\\
%%%\x \x {\bf SPC} printer {\bf DATION} ... ;\\
%%%\x \x T: {\bf TASK};\\
%%%\x \x \x {\bf DCL} a {\bf FIXED};\\
%%%\x \x \x {\bf BEGIN}\\
%%%\x \x \x \x ...\\
%%%\x \x \x \x {\bf DCL} x {\bf FLOAT};\\
%%%\x \x \x \x ...\\
%%%\x \x \x {\bf END};\\
%%%\x \x \x ...\\
%%%\x \x \x {\bf END}; ! T\\
%%%\x \x ...\\
%%%{\bf MODEND};

The scope of an object are all parts of the program where the object can
be used. The following rules are to be obeyed:
\begin{itemize}
\item An object declared at module level is usable at the module level
and in all tasks and procedures of this module (however, see 
\ref{sec_references_module}), even
in all encapsulated procedures, blocks and repetitions with the
following exception: The scope range is restricted, if in one of the
tasks or procedures another object is declared under the same name.
\item An object declared in a task, procedure, repetition or block is
usable in this task, procedure, repetition or block and all
encapsulated procedures, repetitions and blocks with the following
exception: The scope is restricted, if in one of the encapsulated
procedures, repetitions or tasks another object is declared under the
same name.
\end{itemize}

Example:

% Klammern einfuegen !!!!!!!!!!!!!!!!!!!!!!!!!!!!!!!!!!!!!!!!!!!!!!!!!!!!!!!!!!

{\bf PROBLEM};\\
\x {\bf DCL} x {\bf FIXED};\\
\x ...\\
\x {\bf SPC} xx {\bf FIXED IDENT} (x);\\
\x T: {\bf TASK};\\
\x \x ...\\
\x \x {\bf END}; ! T\\
\x P: {\bf PROC};\\
\x \x {\bf DCL} y {\bf FIXED};\\
\x \x x := 2;\\
\x \x {\bf BEGIN}\\
\x \x \x {\bf DCL} x {\bf FLOAT};\\
\x \x \x x :=3 ;\\
\x \x \x ...\\
\x \x \x {\bf BEGIN}\\
\x \x \x \x {\bf DCL} x {\bf DUR};\\
\x \x \x \x ...\\
\x \x \x {\bf END};\\
\x \x \x ...\\
\x \x {\bf END};\\
\x \x y := x; ! y = 2\\
\x \x ...\\
\x \x {\bf END}; ! P\\
\x ...\\
{\bf MODEND};

After END, blocks can be provided with identifiers, so that encapsulated
blocks can be left deliberately with the help of the exit statement
(cf. 7.5).

\section{References between Modules}  % 4.4
\label{sec_references_module}

If a PEARL program consists of several modules, it can be necessary to
use objects declared in a module and occupying memory space there (data,
procedures, etc.) in other modules as well. For this reason, these
(global) objects are declared with the global attribute at module level
in the module where they are to occupy memory space, and specified with
the global attribute at the module level in the other modules. In this
way, the scope of objects declared at module level can be extended.

Example:

\begin{tabbing}
\x {\bf DCL} x {\bf FIXED GLOBAL} \hspace{2cm} \= \kill

{\bf MODULE} (a);                  \> {\bf MODULE} (b); \\
{\bf PROBLEM};                     \> {\bf PROBLEM};\\
\x ...                             \> \x ...\\
\x {\bf DCL} x {\bf FIXED GLOBAL}; \> \x {\bf SPC} x {\bf FIXED GLOBAL} (a);\\
\x ...                             \> \x ...\\
\x x := 2;                         \> \x x := 3;\\
\x ...                             \> \x ...\\
{\bf MODEND};                      \> {\bf MODEND};
\end{tabbing}

All data stations, interrupts and signals given in the system part of a
certain module are regarded as declared (implicitly) with the global
attribute.
\begin{discuss}
Diese Sonderbehandlung ist schwer nachvollziehbar. 
Systemnamen sind immer GLOBAL. Daher sollte GLOBAL immer 
bei deren SPC-Anweisungen angegeben sein.

Dies ist Thema in der AK-Sitzung im November 2016.

\begin{removed}
Thus, they are only specified in the problem parts of the
program; here the global attribute can be omitted in the problem part of
the same module, in all other modules it must be defined.
\end{removed}
\end{discuss}

The general form of the global attribute reads:

\input{GlobalAttribute.bnf}
\begin{discuss}
in der deutschen Fassung des Sprachreports von 1995 ist noch dargelegt,
dass Identifier(Module) nur dokumentatorischen Character hat.
In dieser Fassung ist diese Anmerkung entfallen.

Dies ist Thema in der AK-Sitzung im November 2016.

Klaerungsmoeglichkeit 1:\\
Identifer(Module) entfaellt in der Sprachdefinition

Klaerungsmoeglichkeit 2: Modulangabe ist notwendig\\
\begin{added}
The identifer$\S $Module denotes the module name of the corresponding
declaration. Multiple declarations of the same identifier in different modules
are distinct.
\end{added}  
\end{discuss}

%%%%%\begin{discuss}
%%%%%Formulierung bezueglich Laengenangabe ist schwer  verst"andlich.

%%%%%\begin{removed}
%%%%%When specifying an object, all attributes have to be taken over from its
%%%%%declaration, except for a given priority, precision or length. In the
%%%%%latter exceptional case, the precision or length defined in the
%%%%%corresponding length declaration is applied for the range of the program
%%%%%where the specification is valid.
%%%%%\end{removed}

%%%%%Vorschlag:\\
%%%%%\begin{added}
When specifying an object, all attributes have to be taken over from its
declaration, except for a given priority. 
The effective precission or length 
must fit exactly. Maybe the literal specification and declaration differ 
due to different length declarations
 in the corresponding section of the program.
%%%%%\end{added} 

%%%%%\end{discuss}

Example:

% ??? Einrueckungen im Original unuebersichtlich und unklar

%%%%% SPC in rechter Spalte mit , statt ;

\begin{tabbing}
\x P: {\bf PROC} (A(,) {\bf FIXED IDENT})\=  \kill

{\bf MODULE};                     \> {\bf MODULE};\\
{\bf PROBLEM};                    \> {\bf PROBLEM};\\
\x T: {\bf TASK PRIO 3 GLOBAL};   \> \x {\bf SPC} T {\bf TASK GLOBAL},\\
\x \x \x ! task body              \> \x \x P {\bf ENTRY} ((,) {\bf FIXED IDENT}) {\bf GLOBAL};\\
\x {\bf END}; ! T              \> \x INIT: {\bf TASK};\\
                                  \> \x \x {\bf DCL} TAB (10,20) {\bf FIXED;}\\
\x P: {\bf PROC} (A(,) {\bf FIXED IDENT}) \> \x \x ...\\
\x \x \x \x {\bf GLOBAL};         \> \x \x {\bf CALL} P (TAB);\\
\x \x \x ! procedure body         \> \x \x ... \\
\x {\bf END}; ! P              \> \x \x {\bf ACTIVATE} T;\\
\x ...                            \> \x {\bf END}; ! INIT\\
{\bf MODEND};                     \> \x ...\\
                                  \> {\bf MODEND};
\end{tabbing}

The different forms of specifying global objects are defined when
presenting the objects.

\section{Execution of a Program}   % 4.5
\label{sec_program_execution}

After the loading of a PEARL program, the PEARL run time system
automatically starts all tasks marked by the attribute MAIN according
to their priority.  
%%%begin difference
%%%All tasks provided with MAIN have to be declared in
%%%the same module.
OpenPEARL supports tasks with attribute MAIN in multiple modules.
%%%end difference

Example:

\begin{tabbing}
{\bf MODULE} (Main); \= \\
{\bf SYSTEM}; ...    \> \\
{\bf PROBLEM};       \> \\
start:               \> {\bf TASK MAIN};\\
                     \> ! activating and scheduling of other tasks\\
                     \> {\bf END}; start\\
                     \> \\
measuring:           \> {\bf TASK PRIO 1};\\
                     \> ! task body \\
                     \> {\bf END}; ! measuring\\
...                  \> \\
{\bf MODEND};        \>
\end{tabbing}

After the loading, the task start is started first.

