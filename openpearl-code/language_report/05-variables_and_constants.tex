\chapter{Variables and Constants}  % 5

\section{Declaration and Specification of Variables and Constants}  % 5.1

Upon its execution, a PEARL program uses and modifies integers, floating
point numbers, bit strings, character strings, times and durations.
These data occur in form of constants or as values of variables.
Constants are identified by their notations and keep their values during
the entire program execution. Variables denote data (their values) which
can change during the program execution.

Generally, the value range of a variable is limited to a single kind of data,
e.g. bit strings, which determines the type of the variable (Bit string
variables have only bit strings as values, e.g.). This type has to be
defined with its identifier, when declaring or specifying a
variable.

Example:

\begin{tabbing}
SPECIFY Status BIT (16) GLOBAL; xx \= \kill

{\bf DECLARE} switch {\bf BIT}; \> /* declaration of a variable
                                   switch of type bit string of  \\
                                \> length 1 */ \\
                                \> \\
{\bf SPECIFY} status {\bf BIT}(16) {\bf GLOBAL};
                                \> /* specification of a global
                                   variable status of type bit string \\
                                \> of length 16 */
\end{tabbing}

A variable denotes {\it one} data element, e.g. {\it one} integer, {\it
one} bit string, etc.; in the following, such scalar variables are
treated. The possibilities to summarise scalar variables to arrays and
structures are described in 5.10 and 5.11.

When declaring a variable, its type has to be given in a type attribute;
if different variables with the same type attribute are to be declared,
this can take place in form of a list in a declaration, e.g., by

{\bf DECLARE} (x,y,z) {\bf FLOAT};

the three variables x, y, and z are declared with the type attribute
FLOAT.  For simpler notation, different declarations may be formulated
in a declaration by separating them by commas:

\begin{tabbing}
{\bf DECLARE} \= x {\bf FLOAT,}\\
              \> i {\bf FIXED};
\end{tabbing}

Summarised, scalar variables can be declared as follows:

\begin{front}
ScalarVariableDeclaration ::=\\
\x \{ {\bf DECLARE $\mid$ DCL} \} VariableDenotation [ , VariableDenotation ] $^{...}$ ;

VariableDenotation ::=\\
\x IdentifierDecnotation [ AllocationProtection ] TypeAttribute [ GlobalAttribute ]\\
\x [ InitialisationAttribute ]

IdentifierDenotation ::=\\
\x Identifier $\mid$ (Identifier [ , Identifier ] $^{...}$ )

TypeAttribute ::=\\
\x SimpleType $\mid$ TypeReference $\mid$ Identifier\S ForType

SimpleType ::=\\
\x TypeInteger $\mid$ TypeFloatingPointNumber $\mid$\\
\x TypeBitString $\mid$ TypeCharacterString $\mid$\\
\x TypeTime $\mid$ TypeDuration
\end{front}
\begin{grammar}
\input{SimpleType.bnf}
\end{grammar}

The general form of specifying scalar variables reads as follows:

\begin{front}
ScalarVariableSpecification ::=\\
\x \{ {\bf SPECIFY $\mid$ SPC} \} VariableDenotationS [ , VariableDenotationS ] $^{...}$;

VariableDenotationS ::=\\
\x IdentifierDenotation [ AllocationProtection ] TypeAttribute\\
\x \{ GlobalAttribute $\mid$ IdentificationAttribute \}
\end{front}
\begin{grammar}
not in grammar
\end{grammar}

Allocation protection and initialisation attribute are described in 5.14,
identification and global attributes were already defined in 4.2 and
4.4.

Subsequently, the different type possibilities are presented; variables
of type Sema, Bolt, Interrupt or Signal are considered in 9.3, 9.4 and
10.

\section{Integers (FIXED)}   % 5.2

Integers can be written in decimal or dual representation. The decimal
representation of an integer consists of a sequence of decimal digits. The dual
representation consists of a sequence of the digits 0 and 1, terminated
with character B.

Examples:

\begin{tabular}{rrr}
integer & decimal pres. & dual pres. \\
      6 &             6 &     110B \\
    123 &           123 & 1111011B
\end{tabular}

Furthermore, the precision of the representation can be defined for a
numerical constant by noting the number of bit positions, in which it
shall be represented computer internally without signs, in brackets behind
the numerical constant.

Example: 123(31) \x Integer 123 is presented in 31 bit positions.

If no precision is given, the precision is derived from the constants.

Variables for integers are declared with the type attribute FIXED.

\begin{front}
TypeInteger ::=\\
\x {\bf FIXED} [ (Precision) ]

Precision ::=\\
\x IntegerWithoutPrecision\S GreaterZero
\end{front}
\begin{grammar}
\input{Integer.bnf}

\input{IntegerWithoutPrecision.bnf}

\input{Precision.bnf}
\end{grammar}

The precision gives the number of bit positions in which the
corresponding value of the variable (without signs) is represented. If
the precision declaration is lacking, the precision defined in a length
specification (see 5.6) is inserted. 
\removedtext{Usual precisions are 15 or 31.}

\begin{added}
\OpenPEARL{} supports all lengths from 1 to 63.
All operations are monitored for successful operation.
In case of a range violations the signal FixedRangeSignal is induced.
In case of divide by zero or reminder by zero the signal FixedDivideByZeroSignal
is induced.
\end{added}

Example:

{\bf DCL} counter {\bf FIXED}(31),\\
\x (i,j,k) {\bf FIXED};\\
...\\
i := 2;

\section{Floating Point Numbers (FLOAT)}  % 5.3

Constants for floating point numbers can be represented as a sequence of
\begin{enumerate}
\item a point, an integer and possibly an exponent to the basis 10,
where an exponent consists of a sequence of the character E, possibly a
plus or minus sign, and an integer

\begin{tabbing}
Example: \= .123E-1 xx \=  \kill
Example: \> .123 \> (corresponds to 0.123)\\
 \> .123E2 \> (corresponds to 12.3)\\
 \> .123E-1 \> (corresponds to 0.0123)
\end{tabbing}

\item an integer followed by a sequence given in 1.

Example: 3.123E2 (corresponds to 312.3)

\item an integer, a point, and possibly an exponent

Example: 3. (corresponds to 3.0)

\item an integer and an exponent

Example: 3E-2 (corresponds to 0.03)
\end{enumerate}

Analogously to integers, even constants for floating point numbers can
be defined with precision.

Variables of type floating point number (with integers or floating point
numbers as values) are declared with type attribute FLOAT.

\begin{front}
TypeFloatingPointNumber ::=\\
\x {\bf FLOAT} [ (Precision) ]
\end{front}
\begin{grammar}
not in grammar
\end{grammar}

The statements of 5.2 are valid; 
\removedtext{usual precisions are here 23 and 53, respectively.}
\begin{added}
\OpenPEARL{} uses the supports the lengths 24 and 53.
All operations are monitored for successful operation.
In case of range violations or
divide by zero the result is set to INF (infinity) and
FloatIsINFSignal is induced.
In case of operations leading to NaN (not a number) the signal
FloatIsNaNSignal is induced.
The default preset value of {\bf FLOAT} values is NaN, which will produce
the FloatIsNaNSignal when used without prio assignment.

The precision for float literals is derived in the priority from: 
\begin{enumerate}
\item the given precision like 1.0(53)
\item the {\bf LENGTH} statement
\item is set to 24 as default value in \OpenPEARL
\end{enumerate}
\end{added}

Example:

{\bf DCL} (x,y,z) {\bf FLOAT},\\
\x Koeff {\bf FLOAT}(53);\\
... \\
x:=3.5; y:=1; Koeff:=3.14E-10;

\section{Bit Strings (BIT)}  % 5.4

A bit string constant can be given in binary form (B1), in form of
tetrades (B2), octades (B3), or in hexadecimal form (B4).

The form Bi (i=1,...,4) consists of an apostrophe, a sequence of
\begin{itemize}
\item digits 0 and 1 in case B1
\item digits 0 to 3 in case B2
\item digits 0 to 7 in case B3
\item digits 0 to 9 and letters A to F in case B4
\end{itemize}
followed by an apostrophe and the corresponding attribute B1 or B2 or
B3 or B4.

Example:

\begin{tabular}{rl}
'110010100111'B1 & corresponds to\\
      '302213'B2 & corresponds to\\
        '6247'B3 & corresponds to\\
         'CA7'B4 & \\
\end{tabular}

Instead of B1, B can be written in a shorter way. The following tables
show the assignment between the binary form and the other forms:

\begin{tabular}{cc@{\hspace{2cm}}cc@{\hspace{2cm}}cc}

B2 & B1 & B3 & B1  & B4 & B1 \\
0  & 00 & 0  & 000 & 0  & 0000 \\
1  & 01 & 1  & 001 & 1  & 0001 \\
2  & 10 & 2  & 010 & 2  & 0010 \\
3  & 11 & 3  & 011 & 3  & 0011 \\
   &    & 4  & 100 & 4  & 0100 \\
   &    & 5  & 101 & 5  & 0101 \\
   &    & 6  & 110 & 6  & 0110 \\
   &    & 7  & 111 & 7  & 0111 \\
   &    &    &     & 8  & 1000 \\
   &    &    &     & 9  & 1001 \\
   &    &    &     & A  & 1010 \\
   &    &    &     & B  & 1011 \\
   &    &    &     & C  & 1100 \\
   &    &    &     & D  & 1101 \\
   &    &    &     & E  & 1110 \\
   &    &    &     & F  & 1111 \\
\end{tabular}

Variables for bit strings are to be declared with type attribute BIT.

\begin{front}
TypeBitString ::=\\
\x {\bf BIT} [ (Length) ]

Length ::=\\
\x IntegerWithoutPrecision\S GreaterZero
\end{front}
\begin{grammar}
not in grammar
\end{grammar}

Length gives the number of elements of the bit string. If the length
option is lacking, it had either to be defined in a length
definition (see 5.6), or 1 is assumed as length.

Example:

{\bf DCL} X\_coord {\bf BIT}(2),\\
\x Y\_coord {\bf BIT}(8);\\
...\\
X\_coord:='01'B;\\
Y\_coord:= 'A9'B4;

The addressing of parts of bit strings is treated in 6.1, Expressions.

\begin{added}
\OpenPEARL{} supports all lengths from 1 to 64.
In case of selecting a slice which is not part of the bit string, the
signal BitIndexOutOfRangeSignal is induced.
\end{added}

\section{Character Strings (CHARACTER)}  % 5.5

A character string constant consists of an apostrophe, a sequence of any
characters (besides of apostrophe) and an apostrophe.

Example: 'fault no.: '

If, however, the character string has to contain an apostrophe, it has
to be represented by two subsequent apostrophes.

Example: 'fault''no.: '

Control characters can be inserted into character string constants with
the help of the switch symbols `` `$\backslash$ '' and `` $\backslash$'
''.  The control characters are given as pairs of sedecimal digits.

Examples:

'This string '$\backslash$ 0D 0A $\backslash$' contains two control characters'\\
''$\backslash$1B$\backslash$' control character at the start of the string'\\
'control character at the end of the string'$\backslash$00$\backslash$''\\
''$\backslash$00$\backslash$'' /* string consisting of one single control character */

Example:

'By changing over to the control character sequence'$\backslash$20\\
$\backslash$' very long character string constants'$\backslash$20\\
$\backslash$' can be created (independently from the used editor).'

Variables for character strings are declared by the type attribute
CHARACTER.

\begin{front}
TypeCharacterString ::=\\
\x \{ {\bf CHARACTER $\mid$ CHAR} \} [ (Length) ]
\end{front}
\begin{grammar}
not in grammar
\end{grammar}

Length gives the number of characters. If no length is indicated, it had
either to be defined in a length specification (see 5.6) or length 1 is
assumed.

Examples:

{\bf DCL} ArticleIdentificator {\bf CHAR}(6);\\
...\\
ArticleIdentificator:='BCD/27';

The addressing of parts of character strings is treated in 6.1,
Expressions.

\begin{added}
\OpenPEARL{} supports all lengths from 1 to 32767.
All operations are monitored:
In case defining a {\bf CHAR} variable with more than 32767 elements the
CharacterTooLongSignal is induced.
In case of selecting a slice which is not part of the {\bf CHAR} variable the 
CharacterIndexOutOfRangeSignal is induced.
\end{added}

\begin{tobedone}
\section{The Length Definition}   % 5.6

With the length definition, the precisions and lengths for such
number and string objects are defined, whose precisions and lengths are
not determined by notation (for constants) or declaration (for
variables).

\begin{front}
\begin{tabbing}
\x {\bf LENGTH} \= \kill
LengthDefinition::=\\
\x {\bf LENGTH} \{ \{ {\bf FIXED $\mid$ FLOAT} \} (precision)\\
              \x \> $\mid$ \{ {\bf BIT $\mid$ CHARACTER $\mid$ CHAR} \} (length) \};
\end{tabbing}
\end{front}
\begin{grammar}
\input{LengthDefinition.bnf}
\end{grammar}
Example:

\begin{tabbing}
\x \x Y {\bf FLOAT}(23); \x \= \kill
{\bf PROBLEM}; \> \\
\x {\bf LENGTH FIXED}(15); \> \\
\x {\bf LENGTH FLOAT}(53); \> \\
\x {\bf DCL} A {\bf FIXED}, \> /* A is of type {\bf FIXED}(15) */ \\
\x \x X {\bf FLOAT},        \> /* X is of type {\bf FLOAT}(53) */ \\
\x \x Y {\bf FLOAT}(23);    \> /* Y is of type {\bf FLOAT}(23) */ \\
\x ... \> \\
\end{tabbing}

For a length definition, the same validity rules are valid as for
variable declarations (see 4.3, Block Structure).
\end{tobedone}

\section{Times (CLOCK)}   % 5.7

A time constant consists of a positive integer to state the hour, an
integer between 0 and 59 for the minute, and a floating point number
between 0 and 59.999... for the second --- separated by a colon, each.
The hour is interpreted modulo 24.

Example:

\begin{tabbing}
11:30:00 \x \= means 11.30\\
15:45:3.5   \> means 15.45 and 3.5 seconds\\
25:00:00    \> means 1.00
\end{tabbing}

Variables for times are declared with type attribute CLOCK.

\begin{front}
TypeClock ::=\\
\x {\bf CLOCK}
\end{front}
\begin{grammar}
missing in grammar
\end{grammar}

Example:

{\bf DCL} time {\bf CLOCK};\\
...\\
time:=12:30:00;

\begin{added}
\OpenPEARL{} provides a resolution for CLOCK values of $1\mu s$
\end{added}

\section{Durations (DURATION)}   % 5.8
\label{sec_type_duration}

A duration constant is composed of stating an hour, a minute, and a
second, however, single pieces of specifications may be lacking. An hour
is given by an integer and the character sequence HRS, a minute by an
integer and the character sequence MIN, a second by a floating point
number and the character sequence SEC.

Examples:

\begin{tabbing}
5 {\bf MIN} 30 {\bf SEC} \= means 5 minutes and 30 seconds \\
.05 {\bf SEC}            \> means 50 milliseconds \\
5 {\bf HRS} 10 {\bf SEC} \> means 5 hours and 10 seconds
\end{tabbing}

The integers and floating point numbers in time constants must not
contain precision specifications.

Variables for durations have to be declared with the type attribute
DURATION.

\begin{front}
TypeDuration ::=\\
\x {\bf DURATION $\mid$ DUR}
\end{front}
\begin{grammar}
not in grammar
\end{grammar}

Example:

{\bf DCL} delay {\bf DUR};\\
...\\
delay:=0.1 {\bf SEC};

\begin{added}
\OpenPEARL{} provides a resolution for DURATION values of $1\mu s$ and
a range of $\approx 100 days$.
In case of an arithmetic overflow in calculations with durations, the
DurationValueSignal is induced.
In case of division by a zero duration or 0 FLOAT, 
the DurationDivideByZeroSignal is induced.
\end{added}

\begin{tobedone}
\section{References (REF)}  % 5.9

For the indirect addressing, so-called reference variables (pointer
variables, pointers) are provided in PEARL. Reference variables have the
names of variables as values (they point at variables). Analogously to
the variables introduced up to now, the value range of a reference
variable is limited to a type of variables given when declaring the
reference variables (see, however, 5.13, Interference of Data
Structures).

The value of a reference variable, the referenced variable, is referred
to with the help of the monadic operator CONT (``content''); this
process is called ``dereferentiation''.

Examples:

\begin{tabbing}
{\bf DCL} (rk1, rk2) {\bf REF FIXED}, \= \kill

{\bf DCL} (k,l) {\bf FIXED}, \> \\
\x        x {\bf FLOAT}      \> \\
\x        (rk1,rk2) {\bf REF FIXED}, \> /* fixed reference variable */ \\
\x        rx {\bf REF FLOAT}; \> /* float reference variable */ \\
rk1:=k;   \> /* rk1 points at k */ \\
rk1:=l;   \> /* rk1 points at l */ \\
rk2:=rk1; \> /* rk2 points at l, pointer assignment */ \\
rx:=x;    \> /* rx points at x */ \\
rx:=k;    \> /* wrong, type unequal */ \\
rx:=rk1;  \> /* wrong, type unequal */ \\
l:=2;     \> \\
k:={\bf CONT} rk1; \> /* k obtains value 2 */ \\
rk2:=3;   \> /* wrong, 3 is no variable */ \\
{\bf CONT} rk2:=3; \> /* l obtains value 3 */ \\
{\bf CONT} rk2:=k; \> /* l obtains value 2 */ \\
\end{tabbing}

Instead of k:=CONT rk1, k:=rk1 can be written in a simpler way; i.e.,
operator CONT may be omitted on the right side of an assignment
(``implicit dereferentiation'').

In general, it holds:

\begin{front}
TypeReference ::=\\
\x {\bf REF} [ VirtDimensionList ] [ AssignmentProtection ]\\
\x \{ SimpleType $\mid$ Identifier\S ForType $\mid$ TypeStructure $\mid$\\
\x \x TypeDation $\mid$ {\bf SEMA $\mid$ BOLT $\mid$ INTERRUPT $\mid$ IRPT $\mid$ SIGNAL $\mid$} \\
\x \x TypeProcedure $\mid$ {\bf TASK} $\mid$ TypeVOID $\mid$ {\bf CHAR}()\\
\x \}

VirtDimensionList ::=\\
\x ( [ , $^{...}$ ] )

Dereferentiation ::=\\
\x {\bf CONT} Name\S ReferenceVariable

\end{front}
\begin{grammar}
\input{TypeReference.bnf}
\input{VirtualDimensionList.bnf}
\input{Dereferentiation.bnf}
\end{grammar}

The definition of the language form TypeReference shows, which type a
variable may have, at which a reference variable shall point.
Particularly, reference variables may also point at arrays and
structures (cf. 5.10 and 5.11). The number n of commas in the virtual
dimension list states that the referenced variable is an (n + 1)
dimensional array. On the other hand, reference variables must not point
at reference variables; referenced arrays and structures, however, may
possess reference variables as elements and components. This can be
used, e.g., to interlink structures, or to build up lists in general.
Section 5.12, Type Definition, gives an example for that.

TypeStructure and TypeDation are defined in 5.11 and 10.2, TypeProcedure and 
TASK in 8.3 and 9.11, TypeVOID and REF CHAR() in 6.4.2
and 5.9.1

If a reference variable R points at an array F with elements F
(i,j,k,...) or at a structure S with components S.Ki, the elements of F
or the components of S are addressed (taking R as starting-point) with R
(i,j,k,...) or R.Ki without using CONT.

Furthermore, a reference variable is dereferentiated implicitly,
\begin{itemize}
\item if it is the actual parameter of a procedure call, and if the
formal parameter belonging to it is no reference variable.
\item if it is used as operand of a dyadic operator, as far as the
latter is not defined for values of reference variables like IS (see
below).
\end{itemize}

Example:

{\bf DCL} rk {\bf REF FIXED},\\
\x k {\bf FIXED};\\
\x ...\\
\x rk:=k; k:=2;\\
\x k:=rk+1; /* equivalent     k:=k+1; */

E.g., to mark the end of a string, a reference variable NIL (zero
pointer) is provided, possessing a certain, constant value.

To compare the values of reference variables, the dyadic operators IS or
ISNT can be used; IS (ISNT) provides the result '1'B ('0'B), if both the
given reference variables possess the same (different) value(s),
otherwise it provides the result '0'B ('1'B).

Example:

{\bf DCL} NextOrder {\bf REF} type\_order;\\
...\\
{\bf IF} NextOrder {\bf IS NIL THEN}\\
\x ...\\
{\bf FIN;}

\subsection{Variable Character String Pointers (REF CHAR())}  % 5.9.1

Variably long character strings can easily be processed with special
pointer objects, so-called variable character string pointers.

Example: Declarations of variable character string pointers

\begin{tabbing}
{\bf SPC} print {\bf ENTRY (REF CHAR()) GLOBAL}; \=  \kill
{\bf DCL} str\_ptr {\bf REF CHAR()}; \> /* pointer declaration */       \\
{\bf SPC} print {\bf ENTRY (REF CHAR()) GLOBAL}; \>  /* pointer as parameter */
\end{tabbing}

A pointer of type ``REF CHAR ()'' contains, besides of the address of a
character string, two counters storing the maximum length and the
actually used length of the referenced character string variable. When
assigning the address of a character string variable to a variable
character string pointer, both counters are initialised with the length
of the referenced character string variable.

Example: Initalisation of a variable character string pointer

\begin{tabbing}
{\bf DCL} str\_ptr \= {\bf REF CHAR}(); \= \kill
{\bf DCL} s1 \> {\bf CHAR}(20); \> \\
{\bf DCL} s2 \> {\bf CHAR}(100); \> \\
{\bf DCL} str\_ptr \> {\bf REF CHAR}(); \> \\
str\_ptr:=s1; \> \> /* adr=s1, length=20, max=20 */ \\
str\_ptr:=s2; \> \> /* adr=s2, length=100, max=100 */
\end{tabbing}

Thus, the variable pointer differs at first not from a character string
pointer with constant length. When assigning to the character string
variable via this pointer, the length of the assigned character string,
however, is stored in the counter for the actual length, and the
otherwise usual filling of the character string with blanks up to the
maximum length is omitted.

Example: Assignment to the referenced variable

\begin{tabbing}
{\bf CONT} str\_ptr:='Werum GmbH'; \= \kill
str\_ptr:=s1; \> /* adr=s1, length=20, max=20 */ \\
{\bf CONT} str\_ptr:='PEARL 90'; \> /* adr=s1, length=8, max=20 */ \\
 \> \\
str\_ptr:=s2; \> /* adr=s2, length=100, max=100 */ \\
{\bf CONT} str\_ptr:='Werum GmbH'; \> /* adr=s2, length=10, max=100 */
\end{tabbing}

For the subsequent accesses to the character string via this pointer,
only the actual length is used.

Example:

\begin{tabbing}
{\bf DCL} str\_ptr x \= {\bf REF CHAR}(); xxxxxxxxxxx \= \kill
{\bf DCL} s1 \> {\bf CHAR}(20); \> \\
{\bf DCL} s2 \> {\bf CHAR}(100); \> /* filler */ \\
{\bf DCL} str\_ptr \> {\bf REF CHAR}(); \> /* variable pointer */       \\
 \> \> \\
str\_ptr:=s2: \> \> /* adr=s2, length=100, max=100 */ \\
{\bf CONT} str\_ptr:='D21337 ' {\bf CAT} 'Lueneburg'; \> \> /* adr=s2, length=16, max=100 */
\end{tabbing}

After this assignment, the variable ``s2'' contains the characters
``D21337 Lueneburg'' in the first 16 positions, the other characters
were not changed by the assignment. The length index of ``str\_ptr'' was
set to the actual length 16.

s1 := {\bf CONT} str\_ptr;

For this assignment, ``CONT str\_ptr'' generates the character string
value 'D21337 Lueneburg' (with type ``CHAR(16)''). Due to the assignment
rules of PEARL, the character string is extended by four blanks, before
this value is assigned to variable ``s1'' (with type ``CHAR(20)'').

Variable character string pointers turn out particularly useful as
formal parameters of procedures. Apart from the address of a character
string variable, the compiler passes its actual length at the call
position.  Thus, e.g. an error routine can give out error texts of
different length.

Example: Character string pointer as parameters

{\bf SPC} print\_error {\bf ENTRY (REF CHAR()) GLOBAL};\\
{\bf DCL} s1 {\bf CHAR}(20)  {\bf INIT}('...');\\
{\bf DCL} s2 {\bf CHAR}(100) {\bf INIT}('...');

...\\
{\bf CALL} print\_error (s1);\\
{\bf CALL} print\_error (s2);

Even character string constants may be passed, if the formal parameter
was defined with type ``REF INV CHAR()''.

Example:

{\bf SPC} print\_message {\bf ENTRY (REF INV CHAR()) GLOBAL};

...\\
{\bf CALL} print\_message ('text 1');\\
{\bf CALL} print\_message ('longer text 2');

A variable character string pointer can be used in all expressions
analogously to constant character string pointers. The length check
belonging to the type check takes first place at run time.

A variable character string pointer is not allowed as result value of a
function. Type ``CHAR()'' may only be used in combination with ``REF'',
i.e., parameter type ``CHAR()IDENT'' is not allowed.
\end{tobedone}

\section{Arrays}   % 5.10

If possible, objects of the same kind are summarised in
one identifier, and the various objects are addressed with indices.

\begin{added}
The value of the index must be within the range.
If the value of any index is out of the defined index range
an \verb|IndexOutOfRangeSignal| is induced.
Each index value must fit into \verb|FIXED(31)|. 

\end{added}

Example:

An embedded controller controls three devices G(1), G(2), and G(3). For
the output of the bit string '0001'B to the controller, G(1) shall be
switched on, device G(2) for '0010'B, and device G(3) for '0100'B. Here it
suggests itself to summarise the three switch-on signals under one
identifier, e.g., switch\_on, and to address them with indices:

{\bf DCL} switch\_on(1:3) {\bf BIT}(4),\\
\x i {\bf FIXED};\\
switch\_on(1):='0001'B;\\
switch\_on(2):='0010'B;\\
switch\_on(3):='0100'B;\\
...

/* taking over value of i (index of the device to be switched on)
from another program part\\
output of switch\_on (i) to the embedded controller */\\

In general, scalar variables of the same type can be combined into
n-dimensional arrays (n=1,2,3,4).

When declaring the array, an identifier is assigned to it; the various
array elements (scalar variables) are addressed under this identifier
and the definition of their positions within the array (the index).

Thus, with the help of

{\bf DCL} Koeff(1:2, 0:3) {\bf FIXED};

a 2-dimensional array is declared, where the first dimension has the
lower boundary 1 and the upper boundary 2, i.e., length 2, whereas the
second dimension has the lower boundary 0 and the upper boundary 3,
i.e., length 4.  This means that Koeff consists of the 8 scalar fixed
variables

Koeff(1,0) \x  Koeff(1,1) \x  Koeff(1,2) \x  Koeff(1,3) \\
Koeff(2.0) \x  Koeff(2,1) \x  Koeff(2,2) \x  Koeff(2,3)

The general form of a array declaration reads as follows:

\begin{front}
ArrayDeclaration ::=\\
\x \{ {\bf DECLARE $\mid$ DCL} \} ArrayDenotation [ , ArrayDenotation ] $^{...}$ ;

ArrayDeclaration ::=\\
\x IdentifierDenotation DimensionAttribute [ AssignmentProtection ] TypeAttribute\\
\x [ GlobalAttribute ] [ InitialisationAttribute ]

DimensionAttribute ::=\\
\x (BoundaryDenotation\S FirstDimension [ , BoundaryDenotation\S FurtherDimension ] $^{...}$)

BoundaryDenotation ::=\\
\x [ Boundary\S Lower: ] Boundary\S Upper

Boundary ::=\\
\x [ - ] IntegerWithoutPrecision $\mid$ Identifier\S NamedConstant $\mid$ \\
\x ConstantExpression
\end{front}
\begin{grammar}
\input{DimensionAttribute.bnf}
\input{DimensionBoundaries.bnf}
\end{grammar}

Thus, even negative integers are allowed as dimension boundary. The
upper boundary of a dimension, however, has always to be greater or
equal its lower boundary.

If the lower boundary is not given, value 1 is assumed for it. By a
named constant, a variable of type FIXED(15) is understood, declared
with assignment protection (INV) and initial value (INIT).

Examples:

{\bf DCL} switch\_on(3) {\bf BIT}(4),\\
\x Koeff(2, 0:3) {\bf FIXED},\\
\x message(20) {\bf CHAR}(12);

The one-dimensional array message contains e.g. 20 error texts, so that,
in the case of error i, a program can pass the message (i) to a console
(i=1,...,20).

Arrays can also be declared mixed with scalar variables in a declaration,
e.g.:

{\bf DCL} numb\_mess {\bf INV FIXED}(15) {\bf INIT}(20);\\
\x ...

{\bf DCL}  message (numb\_mess) {\bf CHAR}(12),\\
\x (i,j,k) {\bf FIXED},\\
\x (switch\_on, switch\_off)(3) {\bf BIT}(4);

The general specification form of an array reads:

\begin{front}
ArraySpecification ::=\\
\x \{ {\bf SPECIFY $\mid$ SPC} \} ArrayDenotationS [ , ArrayDenotationS ] $^{...}$;

ArrayDenotationS ::=\\
\x IdentifierDenotation VirtualDimensionAttribute [ AssignmentProtection ] \\
\x TypeAttribute \{ GlobalAttribute $\mid$ IdentificationAttribute \}

VirtualDimensionAttribute ::=\\
\x ( [ , $^{...}$ ] )
\end{front}
\begin{grammar}
not in grammar
\end{grammar}

The number n of commas in the virtual dimension attribute informs that
the specified array has n+1 dimensions.

In Section 1 of the Appendix it is described in tabular form, which
objects may be combined to arrays.


\section{Structures}   % 5.11

Arrays allow to combine scalar variables of the {\it same} type under
one identifier; the various variables are addressed by this identifier
and their indices.

For many automation problems, especially for those with dispositive
character, however, data structures have to be described, whose
components have {\it different} types.

Example:

The contributions for the TV news of one day are stored on magnetic
tapes (VTR); the contribution sequence of a certain news broadcast of
that day is composed by the editors with the help of a computer
controlling the corresponding VTR. For this, a data set is necessary for
each contribution with, e.g. the following structure:

\begin{itemize}
\item identity label of the contribution
\item archive number
\item notes whether the contribution already was broadcast in the first,
second or third news broadcast of that day
\item initial position of the contribution on the tape
\item end position of the contribution on the tape
\item notes whether original sound is available
\item length of the text to be stored, if needed
\item a text to be stored, if needed
\end{itemize}

Such data structures can be described in a problem oriented way as
structures; the above structure ``Contribution'', e.g., is declared as
follows:

\begin{tabbing}
{\bf DCL} \= {\bf [} \= \kill
{\bf DCL} \>         \> Contribution {\bf STRUCT} \\
          \> {\bf [} \> (IdentNo, Archive) {\bf FIXED},\\
          \>         \> already\_broadcast(3) {\bf BIT}(1),\\
          \>         \> (Start, End) {\bf FIXED},\\
          \>         \> OriginalSound {\bf BIT}(1),\\
          \>         \> TextLength {\bf FIXED},\\
          \>         \> Text {\bf CHAR}(200) {\bf ]};
\end{tabbing}

(A structure is encapsulated in square brackets; they are printed boldly in
the text to distinguish them from the meta-linguistic characters ``[''
and ``]''. When writing down the program, only square brackets or their
substitutional symbols ``(/'' and ``/)'' are used.)

In contrast to array elements, the components of a structure are not
addressed via the joint identifier and an index, but via the joint
identifier and the identifier denoted in their declaration, where both
identifiers are separated by a full-stop.

Example:

By the statement

Contribution.Start := 1027;

value 1027 is assigned to the component Start of structure Contribution.

As the declaration of contribution shows, components of structures may be
arrays; as usual, the array elements are then addressed with their
indexed identifier preceded by the structure identifier, such as in the
form

{\bf IF} Contribution.already\_broadcast(i) {\bf THEN...FIN};

A structure component, however, can also be a structure itself, so that not only
linear, but also hierarchical data structures can be modeled.

Example:

A staff file shall contain descriptions of the staff members; these
descriptions each have the structure shown in Figure~\ref{mitarbeiter}.
\newpage

\begin{figure}
\setlength{\unitlength}{0.00087500in}%
\begin{picture}(4377,2735)(46,-2103)
\thicklines
\put(676,-736){\line( 1, 0){900}}
\put(1351,389){\line( 0,-1){2475}}
\put(1351,-2086){\line( 1, 0){225}}
\put(1351,-1636){\line( 1, 0){225}}
\put(1351,389){\line( 1, 0){225}}
\put(2386,389){\line( 1, 0){765}}
\put(3376,614){\line(-1, 0){225}}
\put(3151,614){\line( 0,-1){450}}
\put(3151,164){\line( 1, 0){225}}
\put(2566,-736){\line( 1, 0){810}}
\put(3376,-286){\line(-1, 0){225}}
\put(3151,-286){\line( 0,-1){900}}
\put(3151,-1186){\line( 1, 0){225}}
\put( 46,-826){\makebox(0,0)[lb]{\smash{\SetFigFont{12}{14.4}{rm}Person}}}
\put(1711,-826){\makebox(0,0)[lb]{\smash{\SetFigFont{12}{14.4}{rm}Address}}}
\put(1711,-1726){\makebox(0,0)[lb]{\smash{\SetFigFont{12}{14.4}{rm}Sex}}}
\put(3466,524){\makebox(0,0)[lb]{\smash{\SetFigFont{12}{14.4}{rm}FirstName}}}
\put(3466,
74){\makebox(0,0)[lb]{\smash{\SetFigFont{12}{14.4}{rm}SecondName}}}
\put(3511,-376){\makebox(0,0)[lb]{\smash{\SetFigFont{12}{14.4}{rm}Street}}}
\put(3466,-826){\makebox(0,0)[lb]{\smash{\SetFigFont{12}{14.4}{rm}PostCode}}}
\put(3511,-1276){\makebox(0,0)[lb]{\smash{\SetFigFont{12}{14.4}{rm}Town}}}
\put(1666,-2076){\makebox(0,0)[lb]{\smash{\SetFigFont{12}{14.4}{rm}\ldots}}}
\put(1711,299){\makebox(0,0)[lb]{\smash{\SetFigFont{12}{14.4}{rm}Name}}}
\end{picture}
\caption{\label{mitarbeiter}}
\end{figure}

The components Name and Address are structures themselves, namely
substructures of the main structure Person which can be declared as follows:


\begin{tabbing}
{\bf DCL} Person \= {\bf [} \= Address \= {\bf [} \= \kill
{\bf DCL} Person \>         \> {\bf STRUCT} \> \> \\
                 \> {\bf [} \> Name    \>         \> {\bf STRUCT}\\
                 \>         \>         \> {\bf [} \> FirstName {\bf CHAR}(10),\\
                 \>         \>         \>         \> SecondName {\bf CHAR}(15),\\
                 \>         \>         \> {\bf ]},\> \\
                 \>         \> Address \>         \> {\bf STRUCT}\\
                 \>         \>         \> {\bf [} \> str {\bf CHAR}(15);\\
                 \>         \>         \>         \> PostCode {\bf FIXED},\\
                 \>         \>         \>         \> Town {\bf CHAR}(15)\\
                 \>         \>         \> {\bf ]},\> \\
                 \>         \> Sex     \>         \> {\bf CHAR}(1)\\
                 \>         \> ...     \>         \> \\
                 \> {\bf ]};\>         \>         \> \\
...              \>         \>         \>         \>
\end{tabbing}

Therefore, substructures can be addressed via their identifier and the
preceding identifiers of the respective higher level structures.

The type of a structure is determined by the arrangement of its components
and their types. Structures of the same type can be combined into arrays:

{\bf DCL} Contribution(20) {\bf STRUCT}...;  /* see above */

Thus, any component of a structure is generally addressed with its (if
needed, indexed) identifier and the (if needed, indexed) identifiers of
all higher level structures, where the names are separated by a full-stop,
each.

Example:

Person.Name.FirstName := 'OTTO';\\
{\bf IF} Contribution(i).already\_broadcast(j) {\bf THEN...FIN};

The general form of the structure declaration reads:

\begin{front}
StructureDeclaration ::=\\
\x \{ {\bf DECLARE $\mid$ DCL} \} StructureDenotation [ , StructurDenotation ] $^{...}$ ;

StructureDenotation ::=\\
\x IdentifierDenotation\S MainStructure [ DimensionAttribute ][ AllocationProtection ]\\
\x TypeStructure [ GlobalAttribute ][ InitialisationAttribute ]

TypeStructure ::=\\
\x {\bf STRUCT [} StructureComponent [ , StructureComponent ] $^{...}$ {\bf ]}

StructureComponent ::=\\
\x IdentifierDenotation TypeAttributeInStructureDeclaration

TypeAttributeInStructureDeclaration ::=\\
\x [ DimensionAttribute ] \\
\x \{ SimpleType $\mid$ TypeReference $\mid$ Identifier\S ForType $\mid$ TypeStructure \}
\end{front}
\begin{grammar}
\input{TypeStructure.bnf}

\input{StructureComponent.bnf}

\input{TypeAttributeInStructureComponent.bnf}

\end{grammar}

The general form of the structure specification reads:

\begin{front}
StructureSpecification ::=\\
\x \{ {\bf SPECIFY $\mid$ SPC} \} StructureDenotationS [ , StructureDenotationS] $^{...}$ ;

StructureDenotationS ::=\\
\x IdentifierDenotation\S MainRecord [ VirtualDimensionAttribute ]\\
\x [ AllocationProtection ] TypeRecord \{ GlobalAttribute $\mid$ IdentificationAttribute\}
\end{front}
\begin{grammar}
missing in grammar
\end{grammar}

\section{Type Definition (TYPE)} % 5.12

A certain structure, e.g., can occur as parameter, as substructure in other
structures, or as type of the transmission data of a data station.  Each
time, the type of this structure must be stated, which might cause
extensive writing effort for complex structures. For that reason, and to
increase the readability of the program, the type of a structure can be
defined as new data type under a freely selectable identifier, and used
under this identifier, e.g., to declare variables of this data type.

Examples:
\begin{enumerate}
\item
\begin{tabbing}
\x \= {\bf TYPE} \= {\bf [} \= \kill
{\bf PROBLEM}; \> \> \> \\
  \> ... \> \> \\
  \> {\bf TYPE} \>         \> Message {\bf STRUCT}\\
  \>            \> {\bf [} \> (IdentNo, Archive) {\bf FIXED},\\
  \>            \>         \> already\_broadcast(3) {\bf BIT}(1),\\
  \>            \>         \> (Start, End) {\bf FIXED},\\
  \>            \>         \> OriginalSound {\bf BIT}(1),\\
  \>            \>         \> TextLength {\bf FIXED},\\
  \>            \>         \> Text {\bf CHAR}(200) {\bf ]};\\
  \>            \>         \> \\
  \> {\bf DCL}  \>         \> Content\_VTR {\bf DATION INOUT} Message...;\\
  \>            \>         \> \\
  \> Coord:     \>         \> {\bf TASK};\\
  \>            \>         \> {\bf DCL} Contribution Message;\\
  \>            \>         \> ...\\
  \>            \>         \> {\bf READ} Contribution {\bf FROM} Content\_VTR;\\
  \>            \>         \> ...\\
  \>            \>         \> {\bf END}; ! Coord\\
  \> ...        \>         \>
\end{tabbing}
\item A task controls certain devices; it receives the control orders
from another task. Since occasionally more than one order can be
existing and the orders can be of different urgency, they are buffered
in form of a queue. Let the structure of the orders be of the (re-defined)
type OrderType; let this even contain the urgency criteria for
arranging new orders. The reference variables Next\_Order and Free,
respectively, as shown in Figure~\ref{kette}, point at the order to be treated next or the next free position
in the queue order string, respectively; starting with Free, let the free
positions be interlinked as well. Let the queue be not longer than 10
elements.


\begin{figure}
\setlength{\unitlength}{0.0007500in}%
\begin{picture}(7539,2889)(124,-2110)
\thicklines
\put(451,-1411){\framebox(1035,1350){}}
\put(2071,-1411){\framebox(1035,1350){}}
\put(3646,-1411){\framebox(1035,1350){}}
\put(6121,-1411){\framebox(1035,1350){}}
\thinlines
\put(451,-961){\line( 1, 0){1035}}
\put(946,-1276){\makebox(0,0)[b]{\smash{\SetFigFont{10}{12.0}{rm}Succ}}}
\put(901,-421){\makebox(0,0)[b]{\smash{\SetFigFont{10}{12.0}{rm}Order}}}
\put(901,-781){\makebox(0,0)[b]{\smash{\SetFigFont{10}{12.0}{rm}\ldots}}}
\put(2071,-961){\line( 1, 0){1035}}
\put(2566,-1276){\makebox(0,0)[b]{\smash{\SetFigFont{10}{12.0}{rm}Succ}}}
\put(2521,-421){\makebox(0,0)[b]{\smash{\SetFigFont{10}{12.0}{rm}Order}}}
\put(2521,-781){\makebox(0,0)[b]{\smash{\SetFigFont{10}{12.0}{rm}\ldots }}}
\put(3646,-961){\line( 1, 0){1035}}
\put(4141,-1276){\makebox(0,0)[b]{\smash{\SetFigFont{10}{12.0}{rm}Succ}}}
\put(4096,-421){\makebox(0,0)[b]{\smash{\SetFigFont{10}{12.0}{rm}Order}}}
\put(4096,-781){\makebox(0,0)[b]{\smash{\SetFigFont{10}{12.0}{rm}\ldots }}}
\put(6121,-961){\line( 1, 0){1035}}
\put(6616,-1276){\makebox(0,0)[b]{\smash{\SetFigFont{10}{12.0}{rm}Succ}}}
\put(6571,-421){\makebox(0,0)[b]{\smash{\SetFigFont{10}{12.0}{rm}Order}}}
\put(6571,-781){\makebox(0,0)[b]{\smash{\SetFigFont{10}{12.0}{rm}\ldots }}}
\put(4681,-1186){\line( 1, 0){270}}
\put(4951,-1186){\line( 0,-1){450}}
\put(4951,-1636){\line(-1, 0){4815}}
\put(136,-1636){\line( 0, 1){1260}}
\put(136,-376){\vector( 1, 0){315}}
\put(3106,-1186){\line( 1, 0){270}}
\put(3376,-1186){\line( 0, 1){1575}}
\put(3376,389){\line( 1, 0){2250}}
\put(5626,389){\line( 0,-1){765}}
\put(5626,-376){\line( 1, 0){495}}
\put(1486,-1186){\line( 1, 0){315}}
\put(1801,-1186){\line( 0,-1){675}}
\put(1801,-1861){\vector( 1, 0){450}}
\put(2566,614){\vector( 0,-1){675}}
\put(4186,614){\line( 0,-1){180}}
\put(4186,344){\vector( 0,-1){405}}
\put(7156,-1186){\line( 1, 0){495}}
\put(7651,-1186){\line( 0, 1){1575}}
\put(7651,389){\vector(-1, 0){450}}
\put(2566,659){\makebox(0,0)[b]{\smash{\SetFigFont{10}{12.0}{rm}Next\_Order}}}
\put(4186,659){\makebox(0,0)[b]{\smash{\SetFigFont{10}{12.0}{rm}Free}}}
\put(2476,-1861){\makebox(0,0)[b]{\smash{\SetFigFont{10}{12.0}{rm}\ldots }}}
\put(5401,-736){\makebox(0,0)[b]{\smash{\SetFigFont{10}{12.0}{rm}\ldots }}}
\put(7021,389){\makebox(0,0)[b]{\smash{\SetFigFont{10}{12.0}{rm}\ldots }}}
\put(946,-2086){\makebox(0,0)[b]{\smash{\SetFigFont{10}{12.0}{rm}1}}}
\put(2521,-2086){\makebox(0,0)[b]{\smash{\SetFigFont{10}{12.0}{rm}2}}}
\put(4141,-2086){\makebox(0,0)[b]{\smash{\SetFigFont{10}{12.0}{rm}3}}}
\put(6616,-2086){\makebox(0,0)[b]{\smash{\SetFigFont{10}{12.0}{rm}10}}}
\end{picture}
\caption{}
\label{kette}
\end{figure}


The order string can be declared as follows:

\begin{tabbing}
{\bf TYPE} \= \x \= \x \= \kill
... \> \> \> \\
{\bf TYPE} \> StringElement {\bf STRUCT} \> \> \\
           \>   \> {\bf [} \> Order  OrderType,\\
           \>   \>         \> ...,\\
           \>   \> {\bf ]};\\
{\bf DCL}  \> OrderString (10) StringElement, \> \> \\
           \> (Next\_Order, Free) {\bf REF} StringElement;\\
... \> \> \>
\end{tabbing}

After initialising and arranging various orders, a finished order can be
removed from the queue as follows (let Help be a variable of type REF
string element):

...\\
Help := Next\_Order;
Next\_Order := Next\_Order.Succ;\\
Help.Succ := Free\\
Free := Help\\
...

These statements are also further examples for implicit
dereferentiation. The order string can be initialised as follows:

{\bf FOR} i {\bf TO} 9\\
\x {\bf REPEAT} \\
\x \x OrderString(i).Succ:=OrderString(i+1);\\
\x {\bf END};\\
OrderString(10).Succ:={\bf NIL};
\end{enumerate}

Generally, a new data type is defined as follows:

\input{TypeDefinition.bnf}

The definition of a new data type has to take place before its use.
Besides, data types defined at module level, can be used in all tasks
and procedures of the module; the usability of locally defined data
types depends on the rules of the block structure.

In contrast to global objects, re-defined data types can only be used
in the module where they are defined. Thus, type definitions for
global objects have to be repeated in all used modules.

Example:

\begin{tabbing}
\x {\bf TYPE} T {\bf STRUCT} [...]; \hspace{2cm} \= \kill

{\bf MODULE} (ModuleA);            \> {\bf MODULE} (ModuleB);\\
{\bf PROBLEM;}                      \> {\bf PROBLEM;}\\
                                    \> \\
\x {\bf TYPE} T {\bf STRUCT} [...]; \> \x {\bf TYPE} T {\bf STRUCT} [...];\\
\x ...                              \> \x ... \\
\x {\bf DCL} x T {\bf GLOBAL;}      \> \x {\bf SPC} x T {\bf GLOBAL}(ModuleA); \\
\x ...                              \> \x ...\\
{\bf MODEND;}                       \> {\bf MODEND;}
\end{tabbing}


\section{The Initialisation Attribute (INITIAL)}   % 5.13

The initialisation attribute allows to assign initial values to
variables upon their declaration. For variables with assignment
protection (see 5.15), this is the only possibility of allocating
values. The initialisation attribute is given as last attribute of the
respective declaration:

\input{InitialisationAttribute.bnf}

\input{InitElement.bnf}

\begin{front}
Constant ::=\\
\x Integer\\
\x $\mid$ FloatingPointNumber\\
\x $\mid$ BitStringConstant\\
\x $\mid$ TimeConstant\\
\x $\mid$ DurationConstant\\
\x $\mid$ {\bf NIL}
\end{front}
\begin{grammar}
\input{Constant.bnf}
\end{grammar}

The type of the given init element has to be compatible with the type of
the declared variable in accordance with the rules for assignment (cf.
6.3):

Example:

...
{\bf DCL} number\_devices {\bf FIXED INIT}(12),\\
\x (UGR, OGR) {\bf FIXED INIT}(2,15);\\
...\\
{\bf FOR} i {\bf FROM} 1 {\bf TO} number\_devices\\
...

Here, the variables number\_devices, UGR, and OGR have the values 12, 2
and 15, i.e., the elements of a declared constant list are assigned in
the sequence of writing down the declared identifier list.

Arrays and structures can also be initialised at the position of
declaration. For all structure components, one initial value has to be
given, each. For arrays, the initialisation list may be shorter than the
number of array elements. In this case, the last initial value for the
remaining elements is used repeatedly.

Example:

\begin{tabbing}
{\bf DCL} \= all\_sixes \= {\bf INIT}(21337, \= HouseNumber \= \kill
... \> \> \> \> \\
{\bf DCL} \> Address \> {\bf STRUCT}[ \> PostNumber  \> {\bf FIXED},\\
          \>         \>               \> Town        \> {\bf CHAR}(20),\\
          \>         \>               \> Street      \> {\bf CHAR}(20),\\
          \>         \>               \> HouseNumber \> {\bf FIXED \hspace{1cm} ]}\\
          \>         \> {\bf INIT}(21337, 'Lueneburg', 'Erbstorfer Landstr.', 14); \> \> \\
{\bf DCL} \> colours \> (3)           \> {\bf CHAR}(7) \> {\bf INIT} ('black\ \ ', 'white\ \ ', 'red\ \ \ \ ');\\
{\bf DCL} \> all\_sixes \> (6)        \> {\bf FIXED}  \> {\bf INIT} (6);\\
... \> \> \> \>
\end{tabbing}

The initialisation list can, besides constants, even contain identifiers
of named constants as well as expressions calculable by the compiler
(cf. 5.15.1 and 5.15.2).

\begin{tobedone}
\section{Assignment Protection (INV)}    % 5.14

Variables can be declared with an assignment protection,i.e. the attribute
INV (from ``invariant''), in order to forbid --- apart from
initialisation --- assignments to these variables. The attribute INV is
noted immediately before the type attribute.

Example:

...\\
{\bf DCL} Pi {\bf INV FLOAT INIT}(3.141);\\

Pi := 3;  /* results in error message of the compiler */

An assignment protection which once is associated with an object in its
declaration cannot be revoked; thus it must be considered in
specifications of this object, but also when transmitting it as actual
procedure parameter or when using it as value of reference variables. On
the other hand, in one of this ways an assignment protection may arise
which was not associated in the declaration.

Example:

\begin{tabbing}
{\bf CALL} \= pi {\bf FLOAT IDENT} (Pi); \= \kill
... \> \> \\
P:  \> {\bf PROC}(A (,) {\bf INV FIXED IDENT}, x {\bf FLOAT IDENT}); /* procedure body */ \> \\
    \> {\bf END}; ! P \> \\
    \> \> \\
{\bf DCL} \> Tab(10,20) {\bf FIXED}, \> \\
          \> Pi {\bf INV FLOAT INITIAL}(3.141), \> \\
          \> R1 {\bf REF FLOAT}, \> \\
          \> R2 {\bf REF INV FIXED}; \> \\
... \> \> \\
    \> \> \\
{\bf SPC} \> pi {\bf FLOAT IDENT} (Pi); \> /* wrong */ \\
... \> \> \\
    \> \> \\
{\bf CALL} \> P (Tab, Pi); \> /* wrong */ \\
R1:=Pi;    \>              \> /* wrong */ \\
R2:=Tab(5,7); \> \>
\end{tabbing}

The specification of pi is wrong, because assignments to pi are allowed,
and thus the assignment protection of Pi is circumvented.

The call of P is wrong, because the formal parameter x belonging to Pi
is specified without assignment protection, and thus a (forbidden) value
change of Pi could happen in the body of P via an assignment to x.

In contrast, in the same call, the parameter transmission from Tab to A
is correct, because only an assignment protection for Tab arises which
was not introduced yet.

The assignment R1:=Pi is not permitted, because otherwise the assignment
protection for Pi via (allowed) assignments to CONT R1 could be revoked.
The assignment R2:=Tab(5,7), however, is permitted; a value modification
of the tabular element via CONT is not possible.

\subsection{Named Constants (INV Constants)}   % 5.14.1

The compiler treats objects of type FIXED, BIT, CHAR(1), CLOCK and
DURATION declared with attribute INV as (named) constants. The names of
these objects may be used after their declaration (with initialisation)
for declaring and initialising further objects. For this, they may be
applied as dimension boundary denotation in array declarations as well
as in initialisation lists.

Example:

A surveillance program shall be adapted for various projects with
different numbers of embedded controllers. The number of message
buffers and synchronising variables belonging to them (cf. 9.3) can be
adapted easily by changing the constant NumberControls:

...\\
{\bf DCL} NumberControls {\bf INV FIXED INIT}(11);\\

{\bf DCL} MessageBuffer (NumberControls) {\bf CHAR}(100); \\
{\bf DCL} BufferAccess (NumberControls) {\bf BOLT};\\
...

For a further project with another number of controls, only the named
constant NumberControls has to be adapted to the actual conditions. The
memory areas dependent on that are adapted by the compiler.

Since the compiler does not generate any run time objects for named
objects, they may not be assigned to a reference variable. Likewise,
they cannot be used as actual parameters of a procedure, if the formal
parameter was specified with the IDENTICAL attribute or as reference
variable (cf. 8.2).

\subsection{Constant Expressions (of Type FIXED)}   % 5.14.2

In all positions where the compiler expects constants, also constant
expressions may occur. These expressions may only contain constants
(also named constants) as operands. They are evaluated by the compiler,
and the results are inserted in the respective positions.

\input{ConstantExpression.bnf}

\input{Constant-FIXED-Expression.bnf}

\input{Term.bnf}

\begin{front}
Factor ::=\\
\x [ + $\mid$ - ] \\
\x \{ integer \\
\x \x $\mid$ (Constant-FIXED-Expression)\\
\x \x $\mid$ {\bf TOFIXED} \{ CharacterStringConstant\S OfLength1 $\mid$ BitStringConstant \}\\
\x \x $\mid$ Identifier\S NamedConstant\\
\x \} \\
\x [ {\bf FIT} constant-FIXED-Expression ]
\end{front}
\begin{grammar}
\input{Factor.bnf}
\end{grammar}



Constant expressions of type FIXED can, e.g., be applied in the
following positions:
\begin{itemize}
\item denotation of dimension boundaries in array declarations,
\item precision declaration (for FIXED and FLOAT variables),
\item length definition (for CHAR and BIT variables),
\item in initialisation lists (also for PRESET),
\item priority definition in task declarations and task statements
\item in the constant lists of the CASE statement
\item on the right side of an assignment
\end{itemize}

Often, several objects of a program (variables, constants) are dependent
from one another. If this dependence can be described by a simple
formula, a program can probably be configured by just adapting few
constants.

Example:

A warning shall be given, if a buffer system is filled up to the last 10
\%:

\begin{tabbing}
{\bf DCL} WarnBoundary \= {\bf INV} \= \kill
... \> \> \\
{\bf DCL} MaxBuffer    \> {\bf INV} \> {\bf FIXED INIT}(1000); \\
{\bf DCL} WarnBoundary \> {\bf INV} \> {\bf FIXED INIT}(MaxBuffer - MaxBuffer // 10);\\
{\bf DCL} Buffer (MaxBuffer) \>     \> {\bf FIXED};\\
{\bf DCL} ActualBufferIndex \>     \> {\bf FIXED INIT}(1);
\end{tabbing}
... \\
ActualBufferIndex := ActualBufferIndex+1;\\
{\bf IF} ActualBufferIndex $>$ WarnBoundary\\
{\bf THEN} /* output of warning */ \\
{\bf FIN};\\
...
\end{tobedone}

