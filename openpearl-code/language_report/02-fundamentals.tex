\chapter{Fundamental Features of PEARL}   % 2

\section{Multitasking Characteristics} % 2.1


A data processing program for on-line control or on-line analysis of a
technical process must be able to react as soon as possible to
spontaneous requests of the process or temporal events. Thus, in most
cases it is not sufficient to arrange and to carry out the individual
programming steps sequentially, i.e. in temporally unchangeable order.
The more or less complex automation problem has rather to be split up into
problem related parts with varying urgency, and the program
structure has to be adapted to this problem structure. Upon this,
independent program components for problem parts are created that can be
worked on sequentially within other components (e.g. procedures);
but also independent program components for problem parts arise that
have immediately to be executed in parallel to all other components because
of a temporally not necessarily fixed stimulus (e.g. a fault indication
from the process). Such a program component is called
task; in order to fix their urgency, tasks can be provided with
priorities.

For the declaration and the working together of tasks with each other and
with the technical process, PEARL offers the following possibilities:

\begin{itemize}
\item declaration of tasks, e.g.\\
\code{supplies:} \kw{TASK PRIORITY} \code{2;}\\
\code{~~~!   task object (declarations, statements)}\\
\kw{END};
\item starting (activation), e.g.\\
      \kw{ACTIVATE}\code{ supplies;}
\item terminating, e.g.\\
      \kw{TERMINATE}\code{ printing;}
\item suspending, e.g.\\
      \kw{SUSPEND}\code{ statistics;}
\item continuing, e.g.\\
      \kw{CONTINUE}\code{ statistics;}
\item resuming, e.g.\\
      \kw{AFTER} \code{5} \kw{SEC RESUME}\code{;}
\end{itemize}

According to the requests of automation tasks, some of these
statements can be scheduled for (repeated) execution, e.g. if a time,
the end of a duration, or a message arrives:

\begin{lstlisting}[frame=off]
WHEN ready ACTIVATE supplies;
\end{lstlisting}
(meaning: every time when the interrupt ready occurs, the task \code{supplies}
has to be activated)

A temporal periodical start can also be scheduled:
\begin{lstlisting}[frame=off]
AT 12:0:0 ALL 1 SEC UNTIL 12:15:0 ACTIVATE measuring;
\end{lstlisting}

Different tasks execute their statements independently of one another,
provided relevant measures do not prevent that. However, sometimes a
synchronisation of two or more tasks is necessary, e.g. if a task
produces data and stores them into a buffer. Here the producer must not
work faster than the consumer. More complex synchronisation problems
occur, if e.g. a task has to access to a file exclusively (because
writing), whereas others can access simultaneously (because reading). In
order to solve these synchronisation problems, PEARL contains the data
types \code{SEMA} and \code{BOLT} 
with corresponding statements, such as

\begin{itemize}
\item \kw{REQUEST}\code{ conveyor\_belt;}        %%%conveyor-belt;
\item \kw{RELEASE}\code{ comminication\_buffer;} %%%communication-buffer;
\end{itemize}

%\begin{removed}
%It is fundamental that these multitasking statements are {\it computer
%independent}, i.e. PEARL programs can run on iRMX, OS/2, UNIX or VAX/VMS
%systems without any changes.
%\end{removed}

%\begin{added}
It is fundamental that these multitasking statements are {\it computer 
plattform independent}, i.e. \OpenPEARL{} programs can run on Linux and 
Microcontrollers enhanced with FreeRTOS without any changes.
%\end{added}

In addition to this, there is the great advantage that the PEARL
programmer can program his multitasking statements in a {\it problem
oriented way} with high comfort without becoming deeply involved in the
peculiarities of the various operating systems --- e.g. in the handling
of fork and message queue mechanisms of Unix. The conversion of problem
oriented multitasking statements of PEARL into the mechanisms of
multitasking operating systems is taken over by the PEARL compiler.

\section{Possibilities of Input and Output}   % 2.2

The transmission of data to or from devices of the standard peripherals (printer, hard disk etc.) 
or process peripherals (sensors, 
actuators, etc.), respectively, as well as the control of files take place in PEARL with 
the help of computer independent statements.

Devices and files are summarised by the term data station. Two kinds of
data transmission are distinguished essentially:

\begin{itemize}
\item The transmission of data without format control, i.e. without
      conversion into an external representation:

      This kind of data transmission is used for file communication that
      allows for sequential and direct access as well as for the transmission
      of process data.

      Examples:
\begin{lstlisting}
READ record FROM file BY POS (10);
WRITE data_set TO logbook;
TAKE measured_value FROM TemperatureSensor;
SEND on  TO motor;
\end{lstlisting}

\item The transmission of data with format control, i.e. with conversion
      between internal format and external representation with those
      possibilities available at the data station:

      This means e.g. the representation in the characters of a character set
      of the data station.

      Examples:

\begin{lstlisting}
PUT event TO printer BY F(5);
GET receipt FROM terminal;
\end{lstlisting}
\end{itemize}

The names of the data stations can be chosen freely. This is reached by the
partitioning of a PEARL program into computer dependent and mostly
computer independent parts.

In order to address special devices, the compiler offers a driver
interface, to which the PEARL programmer himself can connect device
drivers.

\section{Program Structure}   % 2.3

Program systems for the solution of complex automation tasks should be
structured in a modular way. PEARL meets this requirement, because a
PEARL program consists of one or several separately compilable
modules.

Connections between modules are possible by means of so-called global
objects (e.g. variables, procedures, tasks).

In order to be able to program the statements for data transmission
and to schedule the reactions to the events from the technical process
(interrupts) or of the computer hardware (signals) computer
independently, a module is usually structured into a system part and a
problem part.

In the system part, the used hardware configuration is described.
Particularly freely chosen names can here be assigned to the devices and
their connections, the interrupts and signals. 

%\begin{removed}
%Thus, the following
%example means that a valve is connected to the connector 3 of a digital
%output unit called by the computer specific system name DIGOUT (1). The
%valve, i.e. connector 3, is to be called by the freely selectable
%computer independent user name ``valve'':
%
%valve: DIGOUT (1) * 3;
%\end{removed}

%%%\begin{discuss}
%%%wir nutzen Userdations auf fuer die ProzessE/A. Das wäre aber hier 
%%%zu frueh... daher nun ein Interrupt

Thus, the following
example means that a button may emit an interrupt when pressed.
The button is connected to the computer at location specified
by the system name \code{GpioInterrupt(2)}.

The button, i.e. GpioInterrupt 2, is to be called by the freely selectable
computer independent user name \code{start}:

\code{start: GpioInterrupt(2);}

Using the user name introduced in the system part, the actual algorithm
for the solution of the automation task is programmed computer
independently in the problem part, e.g.:

\kw{WHEN}\code{ start} \kw{ACTIVATE}\code{ measurement\_task;}
%%%\end{discuss}

In order to structure the algorithms, (named) blocks, procedures and
tasks (parallel activities) are available.

\section{Data Types}   % 2.4

The following data types are available in PEARL:

\begin{itemize}
\item \kw{FIXED} and \kw{FLOAT} with specifyable precision
\item \kw{BIT} and \kw{CHARACTER} strings with specifyable length
\item \kw{CLOCK} and \kw{ DURATION} for times and durations
\item references (\kw{REF}) for indirect addressing
\item devices and files (\kw{(DATION}) for standard and process input and
      output
\item interrupts (\kw{INTERRUPT}) for external interrupts
\item signals (\kw{SIGNAL}) for internal exceptional situations
\item semaphores (\kw{SEMA}) and bolt variables (\kw{BOLT}) for
      coordinating the access of tasks to shared object
\end{itemize}

From this, the user himself can build up more complex data structures
like arrays, hierarchical structures (\kw{STRUCT}) and lists; with the help
of the \kw{TYPE} definition, they can be declared as new, problem oriented
data types as well.
%%%%%removed
%%%%% Furthermore, PEARL allows the introduction of new,
%%%%%problem oriented operators for any data structures with the help of the
%%%%%OPERATOR definition.
%%%%%end removed

\section{Control Structures}   % 2.5

The following control structures are available:

\begin{itemize}
\item conditional statement\\

      \kw{IF} expression\\
      \kw{THEN} statement$^{...}$\\
      \kw{ELSE} statement$^{...}$\\
      \kw{FIN};

\item statement selection\_1\\

      \kw{CASE} expression\\
      \kw{ALT} (alternative 1) statement$^{...}$\\
      \kw{ALT} (alternative 2) statement$^{...}$\\
      ...\\
      \kw{OUT} statement${...}$\\
      \kw{FIN};

\item statement selection\_2\\

      \kw{CASE} Case\_Index\\
      \kw{ALT} (Case\_List) statement$^{...}$\\
      \kw{ALT} (Case\_List) statement$^{...}$\\
      ...\\
      \kw{OUT} statement${...}$\\
      \kw{FIN};

\item loops\\

      \kw{FOR} ControlVariable \kw{FROM} start \kw{BY} StepLength \kw{TO} end\\
      \kw{REPEAT}\\
      statement$^{...}$\\
      \kw{END};\\

      \kw{WHILE} condition\\
      \kw{REPEAT}\\
      statement$^{...}$\\
      \kw{END};

\item exit statement\\

      \kw{EXIT} block;
\end{itemize}

