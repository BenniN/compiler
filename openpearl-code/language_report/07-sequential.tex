% for loop diagramm updated

\chapter{Statements for the Control of Sequential Execution}   % 7

A task or procedure declaration defines a sequence of statements which
are processed sequentially in the defined order when executing the task or
the procedure, unless control statements designated for this influence the
order of processing.

Such control statements are
\begin{itemize}
\item the conditional statement
\item the statement selection
\item the empty statement
\item the repetition
\item the jump statement
\item the exit statement
\end{itemize}

\input{SequentialControlStatement.bnf}

%%%\begin{added}
\section{Empty Statement}
The empty statement  consists of one semicolon only and has no effects.

The general form of the empty statement reads:

\input{EmptyStatement.bnf}

%%%\end{added}

\section{Conditional Statement (IF)}   % 7.1

With the help of the conditional statement, it is determined depending
on the result of an expression with which statement the program
execution is to continued.

\input{IfStatement.bnf}
The result of the expression must be of type BIT(1). If the expression
provides value '1'B (true), the statements following THEN are evaluated;
otherwise the statements following ELSE are evaluated.

If the execution of the statements following THEN or ELSE, resp., does
not result in a jump out of the conditional statement, the statement
following FIN is evaluated subsequently.

Example:

{\bf IF} gradient $>$ degree\_bound \\
\x {\bf THEN} alarm;\\
\x {\bf ELSE IF} gradient $>$ degree\_threshold \\
\x \x \x {\bf THEN ALL} 1 {\bf SEC ACTIVATE} measurement; ! measure more often\\
\x \x {\bf FIN};\\
{\bf FIN};\\
...

\section{Statement Selection (CASE)}  % 7.2

Let us assume that a (function) procedure control shall be used for
controlling several devices of the same kind, after each call returning
a number between 1 and 4 meaning:
\begin{itemize}
\item returned value = 1: request carried out
\item returned value = 2: request data wrong
\item returned value = 3: device not addressable
\item returned value = 4: device does not work correctly
\end{itemize}

The task supply shall perform a measure planned for each case.

For the programming of such case distinctions, the statement selection
provided in two (historically caused) forms is suitable: The older form
allows only integers as distinction criteria; the newer one permits also
characters. At first, the older form is described:

\input{CaseStatement.bnf}
\input{CaseStatement1.bnf}

The number 1 is assigned to the statement sequence following the first
ALT (alternative 1), the number 2 to the statement sequence following
the second ALT (alternative 2), etc.

Upon execution of the case statement, the given expression is
evaluated; it must result in a value of type {\bf FIXED}. If the integer value
is between 1 and the number of given alternatives, the associated
statement sequence is executed; otherwise the statement sequence
following OUT (if given) is executed.

If the selected statement sequence does not contain a jump out of the
statement selection, the statement following FIN is evaluated
subsequently.

Example:

The above problem can be programmed as follows:

\begin{tabbing}

control: \= {\bf TASK} \= order {\bf BIT}(8)) \x \= \kill

supply:  \> {\bf TASK} \> {\bf PRIO} 7;       \> \\
 \> \> \> \\
control: \> {\bf PROC} \> (No {\bf FIXED},    \> ! device \\
 \>             \> order {\bf BIT}(8)) \> ! order inf.\\
 \> {\bf RETURNS} ({\bf FIXED});   \>  \> \\
 \> ! procedure body for carrying out the control order \> \> \\
 \> {\bf END}; ! control \> \> \\
 \> ... \> \> \\
 \> ! creating an order for the device with index no \> \> \\
 \> \> \> \\
again:   \> {\bf CASE} \> control (no, order) \> \\
 \> {\bf ALT}   \> ! order carried out \> \\
 \>             \> ; \> \\
 \> {\bf ALT}   \> ! order inf. wrong \> \\
 \>             \> {\bf CALL} error(2); {\bf GOTO} end; \> \\
 \> {\bf ALT}   \> ! device dead \> \\
 \>             \> {\bf ACTIVATE} device\_breakdown {\bf PRIO} 2; \> \\
 \>             \> {\bf CALL} error(3); {\bf GOTO} end; \> \\
 \> {\bf ALT}   \> ! device works incorrectly \> \\
 \>             \> {\bf CALL} device\_check; {\bf GOTO} again; \> \\
 \> {\bf OUT}   \> ! result out of range \> \\
 \>             \> {\bf CALL} error(5); \> \\
 \> {\bf FIN}; \> \> \\
 \> ...         \> \> \\
end:     \> {\bf END}; \> ! supply \>
\end{tabbing}

%%%In this example, the empty statement is used. It consists of one
%%%semicolon only and has no effects.
The keyword ALT must be followed by
one statement at least; this may also be an empty statement. It has no
effects and is only of interest in conditional statements and statement
selections.

In the example, the empty statement results in the immediate execution
of the statement following FIN in the case of success (``request carried
out'').


The second form of the statement selection has the following form:

\input{CaseStatement2.bnf}
\input{CaseIndex.bnf}

%%%\begin{front}
%%%CaseList ::= \\
%%%\x IndexRange [ , Index-Range ]$^{...}$
%%%
%%%IndexRange ::= \\
%%%\x Constant [ : Constant ]
%%%\end{front}

\input{CaseList.bnf}
\input{IndexRange.bnf}

All given constants must be of type CaseIndex expression; CHAR(1) or
FIXED are allowed.

When executing the StatementSelection2, the CaseIndex is evaluated.
If the value is contained in one of the given case lists, the associated
statement sequence is executed; otherwise the statement sequence
following OUT (if given) is executed.

If the selected statement sequence does not contain a jump out of the
statement selection, the statement following FIN is evaluated
subsequently.

Examples:

\begin{tabbing}
{\bf CASE} \= {\bf ALT} ('A':'Z') \= \kill

{\bf DCL}  \> (Operator, chr)CHAR(1), (x, y) {\bf FIXED}; \> \\
...        \> \> \\
{\bf CASE} \> Operator \> \\
    \> {\bf ALT} ('+')    \> x := x + y; \\
    \> {\bf ALT} ('-')    \> x := x - y; \\
    \> {\bf ALT} ('*')    \> x := x * y; \\
    \> {\bf ALT} ('/')    \> {\bf CASE} y \\
    \>                    \> {\bf ALT} (0) {\bf CALL} Error;\\
    \>                    \> {\bf OUT} x := x//y;\\
    \>                    \> {\bf FIN};\\
{\bf FIN}; \> \> \\
    \> \> \\
{\bf CASE} \> chr \> \\
    \> {\bf ALT} ('A':'Z') \> {\bf CALL} uppercase;\\
    \> {\bf ALT} ('a':'z') \> {\bf CALL} lowercase;\\
{\bf FIN}; \> \> \\
    \> \> \\
{\bf CASE} \> chr \> \\
    \> {\bf ALT} ('A', 'E', 'I', 'O', 'U', 'a', 'e', 'i', 'o', 'u') {\bf CALL} Vocal(chr); \> \\
    \> ... \> \\
{\bf FIN}; \> \>
\end{tabbing}

It is not intended to mix both forms of the statement selection. Thus,
e.g., the following is not correct:

\begin{tabbing}
{\bf CASE} \= {\bf ALT} /* 1 */ \= \kill

{\bf CASE} \> ErrNum \> \\
    \> {\bf ALT} /* 1 */  \> {\bf CALL} ok;\\
    \> {\bf ALT} (0)      \> {\bf CALL} nothing\_done;\\
    \> {\bf ALT} (-99:-1) \> {\bf CALL} ErrorMsg (ErrNum);\\
    \> ... \>
\end{tabbing}

The selection must be deterministic.

Jumps into a case statement are forbidden (refer chapter \ref{sec_goto}).

\section{Repetition (FOR -- REPEAT)}   % 7.3

Often, a statement sequence must be executed repeatedly, with only one
parameter changing. E.g., various devices are to be checked; let
num\_devices be the number of devices:

{\bf FOR} i {\bf FROM} 1 {\bf BY} 1 {\bf TO} num\_devices\\
{\bf REPEAT}\\
\x checking of device(i)\\
{\bf END};

In general, such ``program loops'' are constructed like this:

%%%\begin{front}
%%%repetition ::= \\
%%%\x [ {\bf FOR} Identifier\S ControlVariable ] \\
%%%\x [ {\bf FROM} Expression\S InitialValue ] \\
%%%\x [ {\bf BY} Expression\S Increment ] \\
%%%\x [ {\bf TO} Expression\S FinalValue ] \\
%%%\x [ {\bf WHILE} Expression\S Condition ] \\
%%%\x {\bf REPEAT} \\
%%%\x [ Declaration ]$^{...}$ [ Statement ]$^{...}$ \\
%%%\x {\bf END} [ Identifier\S Loop ] ;
%%%\end{front}
%%%\begin{grammar}
\input{LoopStatement.bnf}
%%%\end{grammar}
The declarations and statements following REPEAT, i.e., the loop body,
are run so often as specified by the clauses given in front of them; the
statement following END is carried out subsequently. However, it is also
possible to leave the loop body prematurely by a jump statement (cf. \ref{sec_goto})
or the exit statement (cf. \ref{sec_exit}). Jumps into the loop body are not permitted.

In the loop body, all statements are permitted; thus, particularly
repetitions can be nested:

{\bf FOR} i {\bf TO} 10\\
{\bf REPEAT} \\
\x {\bf FOR} k {\bf TO} 10 \\
\x {\bf REPEAT} \\
\x \x c (i,k) := a (i,k) + b(i,k);\\
\x {\bf END};\\
{\bf END};

If InitialValue or Increment are missing, they are assumed as 1. If FinalValue is lacking, the loop body can be repeated unrestrictedly.

The ControlVariable may neither be declared nor changed; it has
implicitly type FIXED. The values of the expressions for InitialValue,
Increment and FinalValue have to be of type FIXED, the value of the
expression for the Condition must be of type BIT(1).

The ControlVariable may not be used in the given expressions, except
for Expression\S Condition, but in the statements to be repeated.

Besides, all rules for blocks are valid for the loop body (cf. \ref{sec_block}).

%%%\begin{added}
The precision of the control variable is derived from the
required precision defined by the initial and final expression if both are
specifed.
The default precision is 31.
%%%\end{added}

The flow chart depicted in Figure~\ref{schleife} is an equivalent representation of the
statement

{\bf FOR} Indicator\S ControlVariable \\
{\bf FROM} Expression\S InitialValue \\
{\bf BY} Expression\S Increment \\
{\bf TO} Expression\S FinalValue \\
{\bf WHILE} Expression\S Condition \\
{\bf REPEAT} \\
\x LoopBody \\
{\bf END};

%%%\begin{added}
Regard that the expression for the increment and the final
value are calculated once. Modifications of these expressions
in the loop body does not effect the loop behavior.

The increment of the control variable does not cause
a range exception.
%%%\end{added}

\begin{figure}[htb]
\centering
\setlength{\unitlength}{0.00087500in}%
\begin{picture}(4299,9740)(214,-8900)
\thicklines
\put(1126,-736){\framebox(2475,1125){}}
   \put(2341,119){\makebox(0,0)[b]{\smash{\SetFigFont{10}{14.4}{rm}
	a:=Expression\S InitialValue}}}
   \put(2341,-181){\makebox(0,0)[b]{\smash{\SetFigFont{10}{14.4}{rm}
	s:=Expression\S Increment}}}
   \put(2341,-481){\makebox(0,0)[b]{\smash{\SetFigFont{10}{14.4}{rm}
	e:=Expression\S FinalValue}}}

\put(1576,-1411){\line( 1, 0){1575}}
   \put(3151,-1411){\line( 1,-1){450}}
   \put(3601,-1861){\line(-1,-1){450}}
   \put(3151,-2311){\line(-1, 0){1575}}
   \put(1576,-2311){\line(-1, 1){450}}
   \put(1126,-1861){\line( 1, 1){450}}
   \put(2251,-1681){\makebox(0,0)[b]{\smash{\SetFigFont{10}{14.4}{rm}
	s $>$ 0 AND a $<=$ e OR}}}
   \put(2251,-1951){\makebox(0,0)[b]{\smash{\SetFigFont{10}{14.4}{rm}
	s $<$ 0 AND a $>=$ e OR}}}
   \put(2251,-2221){\makebox(0,0)[b]{\smash{\SetFigFont{10}{14.4}{rm}
	s == 0 ?}}}

\put(1126,-3211){\framebox(2475,450){}}
   \put(2251,-3076){\makebox(0,0)[b]{\smash{\SetFigFont{10}{14.4}{rm}
	Indicator\S ControlVariable := a}}}
\put(1126,-5011){\framebox(2475,450){}}
   \put(2251,-4876){\makebox(0,0)[b]{\smash{\SetFigFont{10}{14.4}{rm}LoopBody}}}
\put(2386,-5011){\vector( 0,-1){450}}

%%%%
\put(1151,-5661){\line( 1, 1){200}}
\put(1151,-5661){\line( 1,-1){200}}
\put(1351,-5461){\line(1, 0){2025}}
\put(1351,-5861){\line(1, 0){2025}}
\put(3576,-5661){\line(-1,-1){200}}
\put(3576,-5661){\line(-1, 1){200}}
\put(1151,-5661){\vector(-1,0){925}}
\put( 901,-5615){\makebox(0,0)[b]{\smash{\SetFigFont{10}{14.4}{rm}yes}}}

   \put(2350,-5661){\makebox(0,0)[b]{\smash{\SetFigFont{10}{14.4}{rm}
	s == 0 ?}}}
\put(2386,-5861){\vector( 0,-1){500}}
\put(2600,-6111){\makebox(0,0)[b]{\smash{\SetFigFont{10}{14.4}{rm}no}}}
%%%%
\put(3601,-6586){\vector( 1, 0){900}}
\put(3916,-6550){\makebox(0,0)[b]{\smash{\SetFigFont{10}{14.4}{rm}no}}}
%%% 
\put(2386,-6811){\vector( 0,-1){500}}
\put(2600,-7061){\makebox(0,0)[b]{\smash{\SetFigFont{10}{14.4}{rm}yes}}}
%%% a  + s < e....
\put(1126,-6586){\line( 1, 1){225}}
   \put(1126,-6586){\line( 1,-1){225}}
   \put(1351,-6361){\line( 1, 0){2025}}
   \put(1351,-6811){\line( 1, 0){2025}}
   \put(3601,-6586){\line(-1, 1){225}}
   \put(3601,-6586){\line(-1,-1){225}}
   \put(2251,-6511){\makebox(0,0)[b]{\smash{\SetFigFont{10}{14.4}{rm}
	s $>$ 0 AND a $<=$ e - s OR}}}
   \put(2251,-6761){\makebox(0,0)[b]{\smash{\SetFigFont{10}{14.4}{rm}
	s $<$ 0 AND a $>=$ e - s ?}}}
%%%

%%%%
\put(1126,-7761){\framebox(2475,450){}}
  \put(2341,-7626){\makebox(0,0)[b]{\smash{\SetFigFont{10}{14.4}{rm}
	a := a + s}}}

\put(2386,-7761){\line( 0,-1){270}}
\put(2386,-8031){\line(-1, 0){2160}}
\put(226,-8031){\line( 0, 1){5535}}
\put(226,-2496){\vector( 1, 0){2160}}

\put(1126,-3886){\line( 1, 1){225}}
\put(1351,-3661){\line( 1, 0){2025}}
\put(3376,-3661){\line( 1,-1){225}}
\put(3601,-3886){\line(-1,-1){225}}
\put(3376,-4111){\line(-1, 0){2025}}
\put(1351,-4111){\line(-1, 1){225}}
   \put(2350,-3906){\makebox(0,0)[b]{\smash{\SetFigFont{10}{14.4}{rm}
	Expression\S Condition == '1'B ?}}}

\thinlines
\put(2386,839){\vector( 0,-1){450}}
\put(2386,-736){\vector( 0,-1){675}}
\put(2386,-2311){\vector( 0,-1){450}}
\put(2386,-3211){\vector( 0,-1){450}}
\put(2386,-4111){\vector( 0,-1){450}}
\put(3601,-1861){\line( 1, 0){900}}
% rechts aussen nach unten
\put(4501,-1861){\line( 0,-1){6420}}
\put(4501,-8281){\line(-1, 0){2115}}
\put(2386,-8281){\vector( 0,-1){270}}
\put(2386,-8776){\makebox(0,0)[b]{\smash{\SetFigFont{10}{14.4}{rm}LoopEnd}}}

\put(3601,-3886){\vector( 1, 0){900}}

\put(3916,-1771){\makebox(0,0)[b]{\smash{\SetFigFont{10}{14.4}{rm}no}}}
\put(3916,-3796){\makebox(0,0)[b]{\smash{\SetFigFont{10}{14.4}{rm}no}}}
\put(2600,-2570){\makebox(0,0)[b]{\smash{\SetFigFont{10}{14.4}{rm}yes}}}
\put(2600,-4325){\makebox(0,0)[b]{\smash{\SetFigFont{10}{14.4}{rm}yes}}}

\end{picture}

\caption{Flow chart for evaluating a repetition statement}
\label{schleife}
\end{figure}

\section{GoTo Statement (GOTO)}   % 7.4
\label{sec_goto}

\input{GoToStatement.bnf}

This statement has the result that program processing is continued at
the program position determined by the label identifier.
%%% This program
%%%position must be a statement and may not be outside the body of the
%%%task or procedure executing the GoTo statement.
The label for a goto statement must be defined in the same block or in a outer block 
in the same procedure or task.

Example:

\begin{tabbing}
measure: read: \= \kill
        \> ...\\
measure: read: \> {\bf READ} value {\bf FROM} device;\\
        \> ... \\
        \> {\bf GOTO} read;
\end{tabbing}

In general, statements can (several times) be marked with lables; i.e.,
the label is given immediately before the (possibly already marked)
statement, separated by a colon.


\section{Exit Statement (EXIT)}    % 7.5
\label{sec_exit}

The exit statement serves to exit blocks and loops deliberately. With
EXIT, also blocks and loops nested several times can be exited
deliberately, which must have an identifier (the jump target) at the
corresponding end.

\input{ExitStatement.bnf}

If the identifier is missing, program processing is continued with the
statement following the end of the block or the loop containing the exit
statement.

If the identifier is given, program processing is continued with the
statement following the end of the indicated block or loop, the exit
statement being in an internal block or loop.

%%%The exit statement may not serve for exiting procedures or tasks.

Example:

\begin{tabbing}
\hspace*{7cm} \= \kill

... \> \\
{\bf BEGIN} \> /* analysis */ \\
\x ... \> \\
\x {\bf TO} number {\bf REPEAT} \> /* comparison */ \\
\x \x ... \> \\
\x \x {\bf IF} MeasuredValue $<$ BoundaryValue \> \\
\x \x {\bf THEN EXIT} analysis; \> \\
\x \x {\bf ELSE} ... \> \\
\x \x {\bf FIN}; \> \\
\x \x ... \> \\
\x {\bf END} comparison; \> \\
\x ... \> \\
{\bf END} analysis; \> \\
{\bf RETURN} (OK);\> \\
\x ... \>
\end{tabbing}

The execution of ``EXIT analysis;'' would immediately be followed by
``RETURN (OK);''.

