\chapter{Procedures}   % 8

When solving an automation problem, in the sense of structured
programming an independent program part is formulated and named for a
logically independent algorithm, particularly if the processing of the
algorithm is needed in several parts of the program, eventually only
changing the arguments of the algorithm, its parameters. The execution
of such a program part is initiated by calling its name --- possibly
provided with actual parameters.

If this call shall have the same effect as executing the proper program
part instead of it, in PEARL this program part is declared and called
as a procedure. Otherwise --- namely, if the statements following the
call are to be executed simultaneously with the called program part ---
the program part is declared and started as task. Tasks are treated in
Chapter 9, Parallel Activities.

Procedures returning a result to their call position are called function
procedures, all other ones are called subprogram procedures.

Example for a subprogram procedure:

Let the procedure Output convert a position indication of type FIXED
into a bit string BinPos and pass it to a machine to be positioned and
marked by number Mach\_No of type FIXED. Let Output be called by
the task Control.

{\bf PROBLEM};\\
\x Output: {\bf PROC} ((Position, Mach\_No) {\bf FIXED});\\
\x \x {\bf DCL} BinPos {\bf BIT}(8);\\
\x \x ! transmission of Position into BinPos\\
\x \x ! output of BinPos to machine Mach\_No \\
\x \x {\bf END}; ! declaration of Output

\x Control: {\bf TASK};\\
\x \x {\bf DCL} (Pos, /* actual nominal position */ \\
\x \x \x         No   /* no of the device */ ) {\bf FIXED};\\
\x \x ... \\
\x \x /* assignments to Pos and No */ \\
\x \x {\bf CALL} Output (Pos, No); \\
\x \x ... \\
\x \x {\bf END}; ! declaration of Control \\
\x ...

Position and Mach\_No are the formal parameters of Output; Pos and No
are actual parameters. BinPos is a local variable of Output only known
within this procedure.

Example for a function procedure:

Due to a schedule Occ\_Plan, procedure Next\_Machine shall determine the
number of the machine to be occupied next among all available machines.
Occ\_Plan shall not be passed on as parameter. Let the number to be
returned be of type FIXED. Next\_Machine shall be declared and called
within the task Supply.

{\bf PROBLEM};\\
\x {\bf DCL} Occ\_Plan ...;

\x Supply: {\bf TASK};\\
\x \x \x {\bf DCL} Mach\_No {\bf FIXED};\\
\x \x \x ... \\
\x \x \x Next\_Machine: {\bf PROCEDURE RETURNS (FIXED)};\\
\x \x \x \x \x {\bf DCL} No {\bf FIXED}; ! No of the next machine\\
\x \x \x \x \x \x ! establishing of No with the help of Occ\_Plan\\
\x \x \x \x \x {\bf RETURN} (No);\\
\x \x \x \x {\bf END} ! Declaration of Next\_Machine\\
\x \x \x ...

\x \x \x Mach\_No := Next\_Machine;\\
\x \x \x ... \\
\x \x {\bf END}; ! Declaration of Supply\\
\x ...

Since the variable Occ\_Plan is declared at module level, it can be used
and, if needed, changed by all procedures and tasks of the module.

\section{Declaration and Specification of Procedures (PROC)}  % 8.1

The statement sequence to be executed when calling a procedure is
prescribed in a procedure declaration, defining a procedure identifier.
The statements of the procedure can use data
\begin{itemize}
\item which are declared at module level or in a higher level relativ to the procedure
(see 4.1),
\item which are specified as formal parameters, i.e., as representatives
for those expressions or variables, which, upon calling, are passed
to the procedure as actual parameters, or
\item which are locally declared in the procedure.
\end{itemize}

The local declarations and the statements of the procedure form the
procedure body.

\begin{front}
ProcedureDeclaration ::= \\
\x Identifier: \{ {\bf PROCEDURE $\mid$ PROC} \} [ ListOfFormalParameters ]\\
\x [ ResultAttribute ]\\
\x [ GlobalAttribute ] ;\\
\x ProcedureBody\\
\x {\bf END};

ProcedureBody ::= \\
\x [ Declaration$^{...}$ ] [ Statement$^{...}$ ]

\end{front}
\begin{grammar}
\input{ProcedureDeclaration.bnf}
\input{ProcedureBody.bnf}
\end{grammar}

\input{ListOfFormalParameters.bnf}

\begin{front}
FormalParameter ::=\\
\x Identifier\_or\_IdentifierList [ VirtualDimensionList ] [ AssignmentProtection ]\\
\x ParameterType [ {\bf IDENTICAL $\mid$ IDENT} ]
\end{front}
\begin{grammar}
\input{FormalParameter.bnf}
\end{grammar}

\input{VirtualDimensionList.bnf}

\begin{front}

ParameterType ::= \\
\x SimpleType $\mid$ TypeReference $\mid$ TypeStructure\\
\x $\mid$ Identifier\S ForType $\mid$ TypeDation $\mid$ TypeRealTimeObject
\end{front}
\begin{grammar}
\input{ParameterType.bnf}
\end{grammar}

\input{ResultAttribute.bnf}

\begin{front}
TypeRealTimeObject ::=\\
\x {\bf SEMA $\mid$ BOLT $\mid$ IRPT $\mid$ INTERRUPT $\mid$ SIGNAL}

ResultType ::= \\
\x SimpleType $\mid$ TypeReference $\mid$ TypeStructure $\mid$ Identifier\S ForType
\end{front}
\begin{grammar}
\input{TypeRealTimeObject.bnf}
\input{ResultType.bnf}
\end{grammar}

The general form of the specification of a procedure reads as follows:

\begin{front}
ProcedureSpecification ::=\\
\x \{ {\bf SPECIFY $\mid$ SPC} \} Identifier\S Procedure \{ {\bf ENTRY $\mid$ [ : ] PROC} \} \\
\x [ ListOfParametersFor-SPC ] [ResultAttribute ] GlobalAttribute ;

ListOfParametersFor-SPC ::=\\
\x (ParameterSpecification [ , ParameterSpecification ]$^{...}$)

ParameterSpecification ::=\\
\x [ Identifier ] [ VirtualDimensionList ] [ AssignmentProtection ] \\
\x ParameterType [ {\bf IDENTICAL $\mid$ IDENT} ]
\end{front}
\begin{grammar}
not in grammar
\end{grammar}

In the procedure specification, the (optionally) definable list of
parameters is only of documentary importance; however, it is thus
possible to copy the head of a procedure declaration into another module
and to generate a correct specification of the procedure by adding the
keyword SPECIFY.

Subprogram procedures are declared without and function procedures with
result attribute. The result type determines the type of the calculated
result which is returned to the call position. With the help of the
return statement, this return takes place in the form

{\bf RETURN} (Expression);

Thus, the value of the expression must have the type specified by the
result attribute.

Execution of the procedure body of a function procedure is terminated by
executing a return statement. Function procedures may only be terminated
this way and no other.

Execution of a subprogram procedure is terminated by
\begin{itemize}
\item executing the return statement in the form\\
      {\bf RETURN};\\
\item executing the last statement of the procedure body.
\end{itemize}

The procedure body can contain declarations, e.g., declarations of local
variables which are only known within the procedure body. However,
further procedures, so-called nested procedures, may also be declared; the
occurring aspects of unambiguity of names, also occurring when declaring
procedures in task bodies, are described in 4.3 within the context of
blocks.

Due to the call, variables or expressions are associated with the
specified formal parameters of the procedure as actual parameters. How
this association takes place (two possibilities), is determined by the
fact whether the attribute IDENTICAL is given or not. Both ways are
explained in 8.2, Call of Procedures.

The number n of commas in the virtual dimension list indicates that the
parameter is an (n+1) dimensional array. Formal array parameters
(virtual dimension list is present) may only be specified together with
the IDENTICAL attribute. If, e.g., the one-dimensional array ``A(10)
FIXED'' shall be passed on to a procedure P with the corresponding
formal parameter Array, Array is to be specified like this: ``Array()
FIXED IDENTICAL''.

Procedures declared at module level, are translated by the compiler with
the {\em re-entrancy} capability, so that they can be used
simultaneously by several tasks (see 9). The recursive call of
procedures is allowed for all --- even for nested --- procedures.
However, since for each task only limited memory space for the local
data of the called procedures (stack) is provided, the programmer
should avoid (or suitably restrict) recursion in the sense of safe
programs.

\section{Calls of Procedures (CALL)}   % 8.2

Subprogram procedures are called with the help of the keyword CALL or
only with their identifiers:

\begin{front}
CallStatement ::=\\
\x [ {\bf CALL} ] Name\S SubprogramProcedure [ ListOfActualParameters ] ;
\end{front}
\begin{grammar}
\input{ProcedureCall.bnf}
\end{grammar}

\input{ListOfActualParameters.bnf}

Example:

{\bf SPC} Output {\bf PROC} (P {\bf FIXED}, N {\bf FIXED}) {\bf GLOBAL};\\
{\bf DCL} (Pos, No) {\bf FIXED};\\
...\\
! Assignments to Pos and No\\
{\bf CALL} Output (Pos, No);

The call statement results in associating the given actual parameters
to the formal parameters of the indicated procedure in the order of
writing down them, and then executing the procedure body. Subsequently,
the statement following the call statement is executed.

The call of a function procedure does not take place as an independent
statement, but within expressions upon stating the identifier and the
actual parameters:

\begin{front}
FunctionCall ::= \\
\x Name\S FunctionProcedure [ ListOfActualParameters ]
\end{front}
\begin{grammar}
\input{FunctionCall.bnf}
\end{grammar}
Example:

The function procedure Ari shall calculate the arithmetic average of an
array of n FLOAT variables. This average shall then be printed together
with the text ``Arith.Average''.

Ari: {\bf PROC} (Array() {\bf FLOAT IDENTICAL}) {\bf RETURNS} ({\bf FLOAT});\\
\x \x {\bf DCL} Sum {\bf FLOAT};\\
\x \x {\bf DCL} (LowerBound, UpperBound) {\bf FIXED};\\
\x \x Sum := 0;\\
\x \x LowerBound := LWB Array; \\
\x \x UpperBound := UPB Array; \\
\x \x {\bf FOR} i {\bf FROM} LowerBound {\bf BY} 1 {\bf TO} UpperBound\\
\x \x {\bf REPEAT}\\
\x \x \x Sum := Sum + Array(i);\\
\x \x {\bf END}; ! loop\\
\x \x {\bf RETURN} (Sum/(UpperBound - LowerBound + 1));\\
\x {\bf END}; ! Ari

\x {\bf DCL} MeasuredValue(10) {\bf FLOAT};\\
\x ...
\x \x /* Acquisition of the measured values */ \\
\x {\bf PUT} Ari (MeasuredValue), 'Arith.Average' {\bf TO} Printer {\bf BY} LIST;\\
\x ...

When evaluating a function call, the given actual parameters are
associated with the formal parameters of the indicated function
procedure in the order of writing down them; then the procedure body is
executed. Subsequently, the evaluation of the expression where the
function call took place is continued --- in the above example the
evaluation of the expression 'Arith.Average' in the put statement.

Both in the call statement and in the function call, the types of the
actual parameters must match the types of the formal parameters
corresponding to them.

The association of the actual parameters with the formal parameters can
take place in two ways: If the specification of a formal parameter has
the attribute IDENTICAL or IDENT, the association takes places with the
help of identification, otherwise by value transmission.

In the case of value transmission (also called {\em call by value}), a
new object, having the type of the formal parameter and being local to
the procedure body, is declared for each defined formal parameter when
invoking the procedure; i.e., the formal parameters become local
variables of specified types. Then, the values of the actual parameters
are assigned to the corresponding formal parameters. An assignment to a
formal parameter by a statement in the procedure body, hence, does not
result in a change of the actual parameter. Furthermore, in this case
any expressions may be passed as actual parameters.

When associating with the help of identification (also called {\em call
by reference}), a formal parameter is identified with the corresponding
actual parameter; i.e., in the procedure body, the data of the actual
parameter are referred to under the name of the formal parameter. An
assignment to a formal parameter in the procedure body means, thus, an
assignment to the variable passed as corresponding actual parameter.
Hence, in this case, not expressions, but only names (of variables) may
be passed as actual parameters.

Example:

\begin{tabbing}
P2: \= \kill

{\bf PROBLEM}; \> \\
    \> \\
P1: \> {\bf PROC} (pi {\bf FIXED}, pj {\bf FLOAT IDENT});\\
    \> ... \\
    \> pi := 3; pj := 5.0;\\
    \> {\bf END}; ! P1\\

P2: \> {\bf PROC} ...;\\
    \> {\bf DCL} (i, j) {\bf FIXED}, a(100) {\bf FLOAT};\\
    \> ... \\
    \> i := 2 ; a(i) := 2.5; \\
    \> {\bf CALL} P1 (i, a(i));\\
    \> ... \\
    \> {\bf END}; ! P2 \\
... \>
\end{tabbing}

After the call of P1 in P2 i (still) has the value 2, but a(i) has the
value 5.0.

As the language form of the procedure declaration already shows (see
8.1), the values of the actual parameters may be of type
\begin{itemize}
\item Integer or FloatingPointNumber, or
\item BitString or CharacterString, or
\item Time or Duration, or
\item Structure or Identifier\S For\_Type, or
\item TypeReference.
\end{itemize}
No explicit values are assigned to objects of types
\begin{itemize}
\item DATION, SEMA, BOLT, INTERRUPT and SIGNAL.
\end{itemize}
Such objects may only be passed to a procedure via identification, i.e.,
the formal parameter may only be defined with the IDENTICAL attribute.

\section{References to Procedures (REF PROC)}  % 8.3

The opportunity to use procedure reference variables is a first step
towards object oriented programming. With the help of it, e.g., data
structures and the necessary procedures for the controlled manipulation of
these structures can be combined into new, abstract data structures.

A declaration of reference variables for procedures contains the
description of all parameter types, as well as the type of the result.

\begin{front}
ProcedureReferenceDeclaration ::= \\
\x \{ {\bf DECLARE $\mid$ DCL} \} Identifier\_or\_IdentifierList [ DimensionAttribute ] [ {\bf INV} ]\\
\x {\bf REF} TypeProcedure [ GlobalAttribute ] [ InitialisationAttribute ] ;

TypeProcedure ::=\\
\x {\bf PROC} [ ListOfParametersFor-SPC ] [ ResultAttribute ]
\end{front}
\begin{grammar}
\input{TypeProcedure.bnf} 
\end{grammar}

The general form of the specification of procedure reference variables
reads:

\begin{front}
ProcedureReferenceSpecification ::= \\
\x \{ {\bf SPECIFY $\mid$ SPC} \} Identifier\_or\_IdentifierList [ VirtualDimensionList ] [ {\bf INV} ]\\
\x {\bf REF} TypeProcedure GlobalAttribute;
\end{front}
\begin{grammar}
not in grammar
\end{grammar}

The value assignment to a procedure reference variable takes place with
the assignment:

\begin{front}
Assignment ::=\\
\x Name\S RefProcVariable \{ := $\mid$ = \} Identifier\S Procedure;
\end{front}
\begin{grammar}
\input{Assignment.bnf}
\end{grammar}

Only procedures declared at module level may be assigned here.

The required matching of types means in this case, that the number of
parameters, all parameter types and also the type of the result
attributes match.

Calling a procedure via a procedure reference variable takes simply
place by giving the reference variable, followed by a list of the actual
parameters. For procedures without parameters, CALL, or in the case of
function procedures, also CONT, can be used.

Examples:
\begin{enumerate}
\item {\bf DCL} ProcPointer {\bf REF PROC} (a {\bf FIXED}, b {\bf FIXED}, c {\bf FIXED IDENT});\\
      add: {\bf PROC} (a {\bf FIXED}, b {\bf FIXED}, c {\bf FIXED IDENT});\\
      \x   c := a + b;\\
      \x   {\bf END};\\
      \x   {\bf DCL} (A, B, C) {\bf FIXED};\\
      ProcPointer := add;\\
      ProcPointer (A, B, C);
\item {\bf DCL} FuncPointer {\bf REF PROC RETURNS(CLOCK)};\\
      time: {\bf PROC RETURNS(CLOCK)};\\
      \x    {\bf RETURN}(NOW);\\
      \x    {\bf END};

      {\bf DCL}(A, B) {\bf CLOCK};\\
      FuncPointer := time;\\
      A := FuncPointer;\\
      B := {\bf CONT} FuncPointer;
\end{enumerate}

