\chapter{Expressions and Assignments}   % 6

\section{Expressions}   % 6.1

Expressions are used for various language forms, e.g.,

\begin{itemize}
\item as index when addressing array elements:\\
      Identifier\S Array(Expression,Expression)\\
      e.g., Tab(K,2*i)
\item as reference value for the return statement in function procedures
      (see 8.1):\\
      {\bf RETURN}(Expression);\\
      e.g., {\bf RETURN}(No);
\item as parameter when calling procedures (see 8.2):\\
      ListOfActualParameters ::= (Expression [ , Expression ] $^{...}$ )\\
      e.g., {\bf CALL} P(A, Tab (K, 2*i));
\item as start condition for tasks (see 9.2.1):\\
      {\bf AT} Expression\S Time\\
      e.g., {\bf AT} 12:00:00 {\bf ACTIVATE} T;
\end{itemize}

These examples show that an expression at least can be

\begin{itemize}
\item a constant
\item an identifier
\item an indexed identifier or
\item an arithmetical expression (e.g., 2*i).
\end{itemize}

Identifiers, indexed identifiers and names of structure components are
generally summarised under the notion ``Name'':

\begin{front}
Name ::= Identifier [ (Index [ , Index ] $^{...}$ ) ] [ . Name ] $^{...}$
\end{front}
\begin{grammar}
\input{Name.bnf}
\end{grammar}
\input{Index.bnf}

Examples:

A, A(3), A(i,j,2*k), A.B, A.B.C, A(3).B.C(i,j)

The identifiers of the components of a structure can be selected
independently from the identifiers outside the structure:

{\bf DCL} Person {\bf STRUCT}..., /* see above */ \\
\x Town {\bf CHAR}(15);\\
\x ...\\
Person.Address.Town := Town;

Generally, the names denoted in expressions have to be names of scalar
variables; however, in the expression lists of the output statements
(cf. 10.4, 10.5, 10,6, 10.7) even identifiers of arrays may be given.

In general, an expression has the form

\input{Expression.bnf}

Monadic operators have only one operand, dyadic operators have two
operands.

\begin{front}
MonadicOperator ::=\\
\x + $\mid$ - {\bf $\mid$ NOT $\mid$ ABS $\mid$ SIGN $\mid$ LWB $\mid$ UPB $\mid$ SIZEOF $\mid$ CONT \\
\x $\mid$ TOFIXED $\mid$ TOFLOAT $\mid$ TOBIT $\mid$ TOCHAR $\mid$ ENTIER $\mid$ ROUND} \\
\x $\mid$ Identifier\S UserDefinedMonadicOperator

DyadicOperator ::=\\
\x + $\mid$ - $\mid$ * $\mid$ / $\mid$ // $\mid$ {\bf REM} $\mid$ ** \\
\x $\mid$ $<$ $\mid$ {\bf LT} $\mid$ $>$ $\mid$ {\bf GT} $\mid$ $<$= $\mid$ {\bf LE} $\mid$ $>$= $\mid$ {\bf GE} $\mid$ == $\mid$ {\bf EQ} $\mid$ /= $\mid$ {\bf NE} \\
\x $\mid$ {\bf AND $\mid$ OR $\mid$ EXOR} $\mid$ $><$ $\mid$ {\bf CAT} $\mid$ $<>$ $\mid$ {\bf CSHIFT $\mid$ SHIFT $\mid$ LWB $\mid$ UPB \\
\x $\mid$ IS $\mid$ ISNT $\mid$ FIT} \x $\mid$ Identifier\S UserDefinedDyadicOperator

Operand ::=\\
\x Constant $\mid$ Name $\mid$ FunctionCall $\mid$ ConditionalExpression $\mid$ Dereferentiation $\mid$\\
\x StringExcerpt $\mid$ (Expression)
\end{front}
\begin{grammar}
\input{MonadicOperator.bnf}
\input{DyadicOperator.bnf}
\input{Operand.bnf}

\end{grammar}

Examples:

\begin{itemize}
\item -A + B * C - D/E**2
\item (A - B) / (A + B)
\item F(Tab(K, 2*i)) / (F(i) - 3)
\item A $<$ B {\bf OR} A $<$ C
\item ProcessImage\_new {\bf AND NOT} ProcessImage\_old\\
      (the result has 1 in {\it that} bit place where ProcessImage\_old
      has 0 and ProcessImage\_new has 1)
\item Xcoord $><$ Ycoord $><$ Zcoord\\
      (the three bit strings are concatenated to one bit string)
\end{itemize}

The conditional expression can be useful, e.g., in assignments or
function procedures (see 8); it has the following form:

ConditionalExpression ::=\\
\x {\bf IF} Expression\S WithValueOfType-BIT(1)\\
\x \x {\bf THEN} Expression {\bf ELSE} Expression {\bf FIN}

If the calculation of the expression following IF yields value '1'B
(true), the expression following THEN is executed; if the value equals '0'B (wrong), the expression after ELSE
is executed.

Examples:

\begin{enumerate}
\item The function procedure max shall state and return the greater one
      of two floating point numbers.

      max: {\bf PROC}((X,Y){\bf FLOAT}) {\bf RETURNS} ({\bf FLOAT});\\
      \x {\bf RETURN}({\bf IF} X$>$Y {\bf THEN} X {\bf ELSE} Y {\bf FIN});\\
      \x {\bf END};\\
      ... \\
      A := max(B,C)/2.0 ;

\item Equivalent with this assignment is the following one:

      A := ( {\bf IF} B$>$C {\bf THEN} B {\bf ELSE} C {\bf FIN})/2.0;
\end{enumerate}

The importance of dereferentiation was treated in 5.6, References. The
general form reads:

\input{Dereferentiation.bnf}

The i-th bit of a bit string can be addressed with the help of the
standard attribute BIT(i) which --- separated by a full-stop --- is mentioned
behind the name of the bit string. Therefore, a bit string B of length
lg is considered as structure B, which as only component has a
one-dimensional array BIT of length lg, whose elements are of type
BIT(1). Analogously, the i-th character of a character string Z can be
addressed with Z.CHAR(i) or Z.CHARACTER(i). The bits or characters of a
bit or character string are numbered from the left to the right, starting with 1.

Example:

\begin{tabbing}
{\bf DCL} Byte {\bf BIT}(8), \= \kill
... \> \\
{\bf DCL} Byte {\bf BIT}(8), \> \\
\x Bt {\bf BIT}(1),   \> \\
\x i {\bf FIXED};     \> \\
... \> \\
Byte:='11101111'B;            \> \\
Bt:=Byte.{\bf BIT}(4);        \> /* Bt obtains value '0'B */\\
Byte.{\bf BIT}(2):='0'B;      \> /* Byte has value '10101111'B */\\
i:=8; \> \\
Byte.{\bf BIT}(i):=Bt;        \> /* Byte has value '10101110'B */
\end{tabbing}

Furthermore, segments of bit or character strings containing several
bits or characters can be addresses analogously; the general form of
addressing is:

\begin{front}
StringPart::=\\
\x Name\S String. \{ {\bf BIT $\mid$ CHAR $\mid$ CHARACTER} \} (Expression\S Start [ : Expression\S End ] )
\end{front}
\begin{grammar}
\input{StringSelection.bnf}
\end{grammar}


The expressions Start and End must have integer values. End must be
greater than or equal to Start, and both values must be within the
declared string length.
\begin{added}
If Start or End is outside of the boundaries of  the CHAR string
the CharacterIndexOutOfRangeSignal ins induced.
If Start or End is outside of the boundaries of  the BIT string
the BitIndexOutOfRangeSignal is induced.
\end{added}

The result of such a string segment expression has type BIT(lg) or
CHAR(lg), resp., where ``lg := End - Start + 1'' is holds. For bit
strings, the compiler must be able to calculate length ``lg'', so that
further type checks are possible when using the string segment
expression.

The following cases are distinguished by the compiler:
\begin{enumerate}
\item Name.\{ {\bf BIT $\mid$ CHAR} \}(Expression)
\item Name.\{ {\bf BIT $\mid$ CHAR} \}(constant-FIXED-Expression: constant-FIXED-Expression)
\item Name.\{ {\bf BIT $\mid$ CHAR} \}(Expression1: Expression2 + FIXED-Constant)
\item Name.{\bf CHAR}(Expression\S Start: Expression\S End)
\end{enumerate}

In case (1), the length of the string segment equals 1, ``Expression''
denotes the bit number or the character position, respectively.

In case (2), the compiler calculates the values of the two constant
expressions and determines the length ``lg := Constant\S End -
Constant\S Start + 1''.

In case (3), Expression1 and Expression2 must be equal. Thus, the length
results from the additive constant: ``lg := FIXED-Constant + 1''.

In case (4), the compiler cannot calculate the length of the string
segment! This is not allowed for bit strings. Case (4) provides a
variable character string, i.e., the length is not calculable until run
time. Variable character strings can be used anywhere, besides in
conjunction with the CAT operator.

Example:

Let a transport vehicle in a warehouse be connected to a digital input
via 16 consecutive positions. Let the meaning of the connectors be as
follows:

\begin{tabular}{l@{ : }l}
connector 1--8   & X coordinate\\
connector 9--12  & Y coordinate\\
connector 13     & Z coordinate\\
connector 14--16 & further parameters
\end{tabular}

When positioning has been completed, the actual position shall be
retrieved and checked.

\begin{removed}
{\bf MODULE};\\
{\bf SYSTEM};\\
\x RFZ : DIGE(1)*0*1, 16;\\
\x ...\\
{\bf PROBLEM};\\
\x {\bf SPC} RFZ {\bf DATION IN BIT}(16);\\
\x {\bf DCL} Rfz {\bf DATION IN BIT}(16) {\bf CREATED}(RFZ);\\
\x ...\\
\x Control:{\bf TASK};\\
\x \x {\bf DCL} Rfz\_State {\bf BIT}(16),\\
\x \x \x Xcoord {\bf BIT}(8), Ycoord {\bf BIT}(4), Zcoord {\bf BIT}(1);\\
\x \x ...\\
\x \x \x /* positioning */ \\
\x \x {\bf TAKE} Rfz\_State {\bf FROM} Rfz;\\
\x \x Xcoord:=Rfz\_State.{\bf BIT}(1:8);\\
\x \x Ycoord:=Rfz\_State.{\bf BIT}(9:12);\\
\x \x Zcoord:=Rfz\_State.{\bf BIT}(13);\\
\x \x \x /* checking of the actual position */ \\
\x \x ...\\
\x {\bf END}; ! Control\\
\x ...
\end{removed}

\begin{added}
{\bf MODULE};\\
{\bf SYSTEM};\\
\x RFZ : DIGE(1)*0*1, 16;\\
\x ...\\
{\bf PROBLEM};\\
\x {\bf SPC} RFZ {\bf DATION};\\
\x ...\\
\x Control:{\bf TASK};\\
\x \x {\bf DCL} Rfz\_State {\bf BIT}(16),\\
\x \x \x Xcoord {\bf BIT}(8), Ycoord {\bf BIT}(4), Zcoord {\bf BIT}(1);\\
\x \x ...\\
\x \x \x /* positioning */ \\
\x \x {\bf TAKE} Rfz\_State {\bf FROM} Rfz;\\
\x \x Xcoord:=Rfz\_State.{\bf BIT}(1:8);\\
\x \x Ycoord:=Rfz\_State.{\bf BIT}(9:12);\\
\x \x Zcoord:=Rfz\_State.{\bf BIT}(13);\\
\x \x \x /* checking of the actual position */ \\
\x \x ...\\
\x {\bf END}; ! Control\\
\x ...
\end{added}


This program part can be programmed more flexibly by additionally
recording the initial positions of the partial strings Xcoord, Ycoord
and Zcoord in variables Init\_X, Init\_Y, Init\_Z.

Control: {\bf TASK};\\
\x ... \\
\x {\bf DCL} (Init\_X, Init\_Y, Init\_Z) {\bf INV FIXED INIT}(1, 9, 13);\\
\x ... \\
\x Xcoord := Rfz\_State.{\bf BIT}(Init\_X:Init\_X+7);\\
\x Ycoord := Rfz\_State.{\bf BIT}(Init\_Y:Init\_Y+3);\\
\x Zcoord := Rfz\_State.{\bf BIT}(Init\_Z);\\
\x ... \\
{\bf END}; ! Control

Assignments to string segments are possible, too; furthermore, they may
occur as operands in expressions.

Parts of an expression may be written in brackets to influence the order
of the expression calculation (cf. 6.3), e.g.

A - ( B + C )

\subsection{Monadic Operators}   % 6.1.1

Table~\ref{tab71} describes for each listed monadic operator
\begin{itemize}
\item which type the operand may have
\item which type the result (of the operation) has, and
\item the meaning of the operator.
\end{itemize}
In Tables~\ref{tab71} to \ref{tab74}, ``a'' stands for any operand, g for the precision, and lg for
the length of the operand.

\begin{table} % 7.1 
\begin{center}
\caption{Monadic operators for numerical, temporal and bit values}
\label{tab71}
\vspace{5mm}
\begin{tabular}{|l|l|l|l|} 
\hline
{\bf Expression} & {\bf Operand type} & {\bf Result type} & {\bf Meaning of operator}\\ \hline

+a               & FIXED(g)          & FIXED(g)         & identity\\
                 & FLOAT(g)          & FLOAT(g)         & \\
                 & DURATION          & DURATION         & \\ \hline

-a               & FIXED(g)          & FIXED(g)         & changing the sign of a\\
                 & FLOAT(g)          & FLOAT(g)         & \\
                 & DURATION          & DURATION         & \\ \hline

NOT a            & BIT(lg)           & BIT(lg)          & inverting all bit positions of a\\ \hline

ABS a            & FIXED(g)          & FIXED(g)         & absolute value of a\\
                 & FLOAT(g)          & FLOAT(g)         & \\
                 & DURATION          & DURATION         & \\ \hline

SIGN a           & FIXED(g)          & FIXED(1)         & determining the sign of a\\ 
                 & FLOAT(g)          &                  & 1 for a $>$ 0\\
                 & DURATION          &                  & 0 for a = 0\\   
                 &                   &                  & -1 for a $<$ 0\\
\hline
\end{tabular}
\end{center}
\end{table}

Examples (Table~ref{tab71}):

\begin{tabbing}
{\bf DCL} (X, Y) \= {\bf FLOAT},  \= \\
\x        B      \> {\bf BIT}(4); \> \\
X := 3;            \> \> \\
B := '1001'B       \> \> \\
Y := -X;           \> \> /* Y has value -3 */\\
B := {\bf NOT} B;  \> \> /* B has value '0110'B */\\
\end{tabbing}

In general, in an assignment, the type of the variable given to the left
of the assignment character must match the type of the value of the
assigned expression (cf. section 6.3). Particularly, only values of type
integer may be assigned to variables of integers. Furthermore, the types
of operands of dyadic operators must be compatible.

Thus, several operators are provided which cause a possibly needed
conversion of the type of objects.

\begin{table} % 7.2
\begin{center}
\caption{Monadic operators for explicit type conversions}
\label{tab72}
\vspace{5mm}
\begin{tabular}{|l|l|l|l|}
\hline
{\bf Expression} & {\bf Operand type} & {\bf Result type} & {\bf Meaning of operator}\\ \hline

TOFIXED a        & CHARACTER(1)      & FIXED(7)         & ASCII code for operand character\\ \cline{2-4}
                     
                 & BIT(lg)           & FIXED(g)         & interpreting the bit pattern of the\\
                 &                   &                  & operand as an integer, with g = lg\\ \hline

TOFLOAT a        & FIXED(g)          & FLOAT(g)         & converting the operand into a\\
                 &                   &                  & floating point number\\ \hline
    
TOBIT a          & FIXED(g)          & BIT(lg)          & interpreting the operand as bit\\
                 &                   &                  & pattern, with lg = g\\ \hline

TOCHAR a         & FIXED             & CHARACTER(1)     & character for the ASCII code of the\\
                 &                   &                  & operand\\ \hline

ENTIER a         & FLOAT(g)          & FIXED(g)         & greatest integer less or equal than\\
                 &                   &                  & the operand\\ \hline
 
ROUND a          & FLOAT(g)          & FIXED(g)         & next integer according to DIN\\ \hline

CONT a           & REF type          & type             & explicit de-referentiation\\
\hline
\end{tabular}  
\end{center} 
\end{table}

Examples (Table \ref{tab72}):
\begin{modified}
\begin{tabbing}
A1 := {\bf ROUND} 3.4; \x \x \=  /* delivers 4 */ \kill 

{\bf DCL} A {\bf FIXED}(31),\\
\x A1 {\bf FIXED}(31);\\
\x (B, C) {\bf BIT}(15);\\
...\\
A1 := {\bf ENTIER} X//2; \> /* corresponds to A := (ENTIER X)//2; */ \\
A1:= {\bf ENTIER} (X/2.0);\\
A1 := {\bf ENTIER} 3.4; \> /* delivers 3 */ \\
A1 := {\bf ROUND} 3.4;  \>  /* delivers 4 */ \\
A1 := {\bf ENTIER} 3.5; \> /* delivers 3 */ \\
A1 := {\bf ROUND} 3.5;  \>  /* delivers 4 */ \\
A1 := {\bf ENTIER} -3.4; \> /* delivers -3 */ \\
A1 := {\bf ROUND} -3.4; \> /* delivers 4 */ \\
A1 := {\bf ENTIER} -3.5; \> /* delivers -3 */ \\
A1 := {\bf ROUND} -3.5; \> /* delivers -4 */ \\
C := {\bf TOBIT} A {\bf AND} B;\\
A := {\bf ROUND} X + {\bf TOFIXED} B;
\end{tabbing}
\end{modified}


\begin{table} % 7.3
\begin{center}
\caption{Monadic arithmetical operators}
\label{tab73}
\vspace{5mm}
\begin{tabular}{|l|l|l|l|}
\hline 
{\bf Expression} & {\bf Operand type} & {\bf Result type} & {\bf Meaning of operator}\\ \hline
SQRT a           & FIXED(g) or        & FLOAT(g)          & square root of operand \\ \cline{1-1} \cline{4-4}
SIN a            & FLOAT(g)           &                   & sine of operand \\ \cline{1-1} \cline{4-4}
COS a            &                    &                   & cosine of operand \\ \cline{1-1} \cline{4-4}
EXP a            &                    &                   & $e^{a}$, with e=2.718281828459\\ \cline{1-1} \cline{4-4}
LN a             &                    &                   & natural logarithm of operand\\ \cline{1-1} \cline{4-4}
TAN a            &                    &                   & tangent of operand \\ \cline{1-1} \cline{4-4}
ATAN a           &                    &                   & arcus tangent of operand\\ \cline{1-1} \cline{4-4}
TANH a           &                    &                   & tangent hyperbolicus of operand\\
\hline
\end{tabular}
\end{center}
\end{table}

\begin{table} % 7.4 Hat noch keinen Textbezug!!!!!!!!!!!!!!!!!!!!!!!!!!!!!
\begin{center}
\caption{Other monadic operators}
\label{tab74}
\vspace{5mm}
\begin{tabular}{|l|l|l|l|}
\hline
{\bf Expression} & {\bf Operand type} & {\bf Result type} & {\bf Meaning of operand}\\ \hline

LWB a            & array              & FIXED(31)        & lower boundary of the first\\
                 &                    &                  & dimension of the operand array\\ \hline

UPB a            & array              & FIXED(31)        & upper boundary of the first\\
                 &                    &                  & dimension of the operand array\\ \cline{2-4}
                 & CHARACTER(lg)      & FIXED(15)        & result := Lg \\ \hline
 
SIZEOF a         & identifier         & FIXED(31)        & memory size of the identified\\
                 &                    &                  & object at run time in bytes\\ \cline{2-2} \cline{4-4}
                 & simple type        &                  & \\ \hline

SIZEOF a MAX     & REF CHAR( )        & FIXED(31)        & maximum length of the referenced\\
                 &                    &                  & character string variable\\ \hline

SIZEOF a LENGTH  & REF CHAR( )        & FIXED(31)        & actual length of the referenced\\
                 &                    &                  & character string variable\\ \hline

TRY a            & SEMA               & BIT(1)           & try of a REQUEST, '1'B\\
                 &                    &                  & if successful\\
\hline
\end{tabular}
\end{center}
\end{table}

\subsection{Dyadic Operators}   % 6.1.2

Tables~\ref{tab75}, \ref{tab76}, \ref{tab77} and \ref{tab78} describe for each listed dyadic operator
\begin{itemize}
\item which type the operands may have
\item which type the result (of the operation) has, and
\item the meaning of the operator,
\end{itemize}
where
\begin{itemize}
\item ``op1'' and ``op2'', denote the first and second operand, resp.
\item g1, g2, ..., and lg1, lg2, ..., denote the precision and
lengths,resp., of the operands and the results.
\end{itemize}
Since the precision of the result of an addition, subtraction,
multiplication or division equals the maximum of the precisions of both
operands, it may happen that the overflows arising in these operations
are cut off. Hence, the dyadic operator FIT is provided.
It causes the conversion from the precision of operand ``a'' into the
precision of operand b. FIT has rank 1.

\begin{table} % 7.5
\begin{center}
\caption{Dyadic operators for numerical and temporal values}
\label{tab75}
\vspace{5mm}
\begin{tabular}{|l|l|l|l|l|}
\hline
{\bf Expression} & {\bf Type}      & {\bf Type}      & {\bf Result type} & {\bf Meaning of operator}\\
                 & {\bf operand 1} & {\bf operand 2} &                   & \\ \hline
op1 + op2        & FIXED(g1)       & FIXED(g2)       & FIXED(g3)         & addition of the values of\\

                 & FIXED(g1)       & FLOAT(g2)       & FLOAT(g3)         & the operands op1 and op2\\
                 & FLOAT(g1)       & FIXED(g2)       & FLOAT(g3)         & \\
                 & FLOAT(g1)       & FLOAT(g2)       & FLOAT(g3)         & g3 = max (g1, g2)\\
                 & DURATION        & DURATION        & DURATION          & \\
                 & DURATION        & CLOCK           & CLOCK             & \\
                 & CLOCK           & DURATION        & CLOCK             & \\ \hline

op1 - op2        & FIXED(g1)       & FIXED(g2)       & FIXED(g3)         & subtraction of the values of\\
                 & FIXED(g1)       & FLOAT(g2)       & FLOAT(g3)         & the operands op1 and op2\\
                 & FLOAT(g1)       & FIXED(g2)       & FLOAT(g3)         & \\
                 & FLOAT(g1)       & FLOAT(g2)       & FLOAT(g3)         & g3 = max (g1, g2)\\
                 & DURATION        & DURATION        & DURATION          & \\
                 & CLOCK           & DURATION        & CLOCK             & \\
                 & CLOCK           & CLOCK           & DURATION          & \\ \hline

op1 * op2        & FIXED(g1)       & FIXED(g2)       & FIXED(g3)         & multiplication of the values of\\
                 & FIXED(g1)       & FLOAT(g2)       & FLOAT(g3)         & the operands op1 and op2\\
                 & FLOAT(g1)       & FIXED(g2)       & FLOAT(g3)         & \\
                 & FLOAT(g1)       & FLOAT(g2)       & FLOAT(g3)         & g3 = max (g1, g2)\\
                 & FIXED(g1)       & DURATION        & DURATION          & \\
                 & DURATION        & FIXED(g2)       & DURATION          & \\
                 & FLOAT(g1)       & DURATION        & DURATION          & \\
                 & DURATION        & FLOAT(g2)       & DURATION          & \\ \hline

op1 / op2        & FIXED(g1)       & FIXED(g2)       & FLOAT(g3)         & division of the values of\\
                 & FLOAT(g1)       & FIXED(g2)       & FLOAT(g3)         & the operands op1 and op2,\\
                 & FIXED(g1)       & FLOAT(g2)       & FLOAT(g3)         & if op2 $\neq$ 0\\
                 & FLOAT(g1)       & FLOAT(g2)       & FLOAT(g3)         & \\
                 & DURATION        & FIXED(g2)       & DURATION          & g3 = max (g1, g2)\\
                 & DURATION        & FLOAT(g2)       & DURATION          & g4 = 31\\
                 & DURATION        & DURATION        & FLOAT(g4)         & (dependent on implementation)\\ \hline

op1 // op2       & FIXED(g1)       & FIXED(g2)       & FIXED(g3)         & integer division of the values of\\
                 &                 &                 &                   & the operands op1 and op2\\
                 &                 &                 &                   & \\
                 &                 &                 &                   & g3 = max (g1, g2)\\ \hline

op1 REM op2      & FIXED(g1)       & FIXED(g2)       & FIXED(g3)         & remainder of the integer division\\
                 &                 &                 &                   & of the values of the operands op1\\
                 &                 &                 &                   & and op2\\ \hline

op1 ** op2       & FIXED(g1)       & FIXED(g2)       & FIXED(g1)         & exponentiation of the values\\
                 & FLOAT(g1)       & FIXED(g2)       & FLOAT(g1)         & of the operands op1 and op2\\ \hline

op1 FIT op2      & FIXED(g1)       & FIXED(g2)       & FIXED(g2)         & changing the precision of\\
                 & FLOAT(g1)       & FLOAT(g2)       & FLOAT(g2)         & operand op1 into the precision\\
                 &                 &                 &                   & of operand op2\\  
\hline
\end{tabular}
\end{center}
\end{table}

Example:

\begin{modified}
\begin{tabbing}
C:=(A {\bf FIT} C)*B; \x \x \= \kill

... \> \\
{\bf DCL} (A, B) {\bf FIXED}(15), \> \\
\x        C {\bf FIXED}(31);      \> \\
{\bf DCL} D {\bf BIT}(16);        \> \\
A:=32767;                         \> \\
B:=4;                             \> \\
C:=(A {\bf FIT} C)*B;    \> /* C obtains value 131068 */ \\
A:=A {\bf FIT} C;        \> /* assignment not permitted */ \\
A:=C {\bf FIT} A;        \> /* OK, but lack of information of C by C FIT A */ \\
A:= (-5) REM (-3);	 \> /* delivers -2 */ \\
A:= (-5) REM (+3);	 \> /* delivers -2 */ \\
A:= (+5) REM (-3);	 \> /* delivers +2 */ \\
A:= (+5) REM (+3);	 \> /* delivers +2 */ \\
\end{tabbing}
\end{modified}

/* conversion of FIXED and BIT objects */ \\
/* without conversion of the internal presentation */ \\
A:=({\bf TOFIXED} D) {\bf FIT} A;\\
D:={\bf TOBIT} (A {\bf FIT} 1(16));

\begin{table} % 7.6
\begin{center}
\caption{Dyadic comparison operators}
\label{tab76}
\vspace{5mm}
\begin{tabular}{|l|l|l|l|l|}
\hline
{\bf Expression} & {\bf Type}      & {\bf Type}      & {\bf Result type} & {\bf Meaning of operator}\\
                 & {\bf operand 1} & {\bf operand 2} &                   & \\ \hline

op1 $<$ op2      & FIXED(g1)       & FIXED(g2)       &   BIT(1)             & ``less than''\\
or               & FIXED(g1)       & FLOAT(g2)       &                      & If op1 is less than op2,\\
op1 LT op2       & FLOAT(g1)       & FIXED(g2)       &                      & the result has value '1'B,\\
                 & FLOAT(g1)       & FLOAT(g2)       &                      & otherwise '0'B.\\
                 & CLOCK           & CLOCK           &                      & \\
                 & DURATION        & DURATION        &                      & \\ \cline{2-3} \cline{5-5}
                 & CHAR(lg1)       & CHAR(lg2)       &                      & character string comparison\\
                 &                 &                 &                      & (for comparison algorithm,\\
                 &                 &                 &                      & cf. explanations)\\ \hline

op1 $>$ op2      & compare         & compare         & BIT(1)               & ``greater than'':\\
or               & op1 $<$ op2     & op1 $<$ op      &                      & If op1 is greater than op2,\\
op1 GT op2       &                 &                 &                      & the result has value '1'B,\\
                 &                 &                 &                      & otherwise '0'B.\\ \hline

op1 $<$= op2     & compare         & compare         & BIT(1)               & ``less or equal''\\
or               & op1 $<$ op2     & op1 $<$ op2     &                      & If op1 is less or equal op2,\\
op1 LE op2       &                 &                 &                      & the result has value '1'B,\\
                 &                 &                 &                      & otherwise '0'B.\\ \hline

op1 $>$= op2     & compare         & compare         & BIT(1)               & ``greater or equal''\\
or               & op1 $<$ op2     & op1 $<$ op2     &                      & If op1 is greater or equal op2,\\
op1 GE op2       &                 &                 &                      & the result has value '1'B,\\
                 &                 &                 &                      & otherwise '0'B.\\ \hline

op1 == op2       & FIXED(g1)       & FIXED(g2)       & BIT(1)               & ``equal''\\
or               & FIXED(g1)       & FLOAT(g2)       &                      & If op1 is equal op2,\\
op1 EQ op2       & FLOAT(g1)       & FIXED(g2)       &                      & the result has value '1'B,\\
                 & FLOAT(g1)       & FLOAT(g2)       &                      & otherwise '0'B.\\
                 & CLOCK           & CLOCK           &                      & If lg2 $\neq$ lg1, the shorter\\
                 & DURATION        & DURATION        &                      & character or bit string, resp.,\\
                 & CHAR(lg1)       & CHAR(lg2)       &                      & is padded with blanks or zeros,\\
                 & BIT(lg1)        & BIT(lg2)        &                      & resp., on the right side to match\\
                 &                 &                 &                      & the length of the longer string.\\ \hline

op1 /= op2       & compare         & compare         & BIT(1)               & ``not equal''\\
or               & op1 $<$ op2     & op1 $<$ op2     &                      & If op1 is not equal op2,\\
op1 NE op2       &                 &                 &                      & the result has value '1'B,\\
                 &                 &                 &                      & otherwise '0'B.\\
                 &                 &                 &                      & If lg2 $\neq$ lg1, the shorter\\
                 &                 &                 &                      & character or bit string, resp.,\\
                 &                 &                 &                      & is padded with blanks or zeros,\\
                 &                 &                 &                      & resp., on the right side to match\\
                 &                 &                 &                      & the length of the longer string.\\ \hline

op1 IS op2       & REF type        & REF type        & BIT(1)               & Comparison for euqality of the\\
                 &                 &                 &                      & values of references variables.\\
                 &                 &                 &                      & The result upon equality is '1'B.\\ \hline

op1 ISNT op2     & REF type        & REF type        & BIT(1)               & Comparison for euqality of the\\
                 &                 &                 &                      & values of references variables.\\
                 &                 &                 &                      & The result upon non-equality\\
                 &                 &                 &                      & is '1'B.\\
\hline
\end{tabular}
\end{center}
\end{table}

\begin{table} % 7.7
\begin{center}
\caption{Dyadic Boolean and shift operators}
\label{tab77}
\vspace{5mm}
\begin{tabular}{|l|l|l|l|l|}
\hline
{\bf Expression} & {\bf Type}      & {\bf Type}      & {\bf Result type} & {\bf Meaning of operator}\\
                 & {\bf operand 1} & {\bf operand 2} &                   & \\ \hline

op1 AND op2      & BIT(lg1)        & BIT(lg2)        & BIT(lg3)          & Boolean bitwise conjunction,\\  
                 &                 &                 &                   & disjunction, or antivalence,\\
                 &                 &                 &                   & resp., of the operands\\
                 &                 &                 &                   & lg3 = max (lg1, lg2).\\
                 &                 &                 &                   & The shorter operand is padded\\
                 &                 &                 &                   & with zeros on the right side to\\
                 &                 &                 &                   & match the length of the longer\\
                 &                 &                 &                   & operand.\\ \cline{1-4}
op1 OR op2       & BIT(lg1)        & BIT(lg2)        & BIT(lg3)          & \\ \cline{1-4}
op1 EXOR op2     & BIT(lg1)        & BIT(lg2)        & BIT(lg3)          & \\ \hline

op1 $<>$ op2     & BIT(lg)         & FIXED(g)        & BIT(lg)           & Cyclic shift of operand op1 for the\\
or               &                 &                 &                   & number of positions determined\\
op1 CSHIFT op2   &                 &                 &                   & by operand op2.\\
                 &                 &                 &                   & If operand op2 $>$ 0,\\
                 &                 &                 &                   & op1 is shifted to the left,\\
                 &                 &                 &                   & otherwise to the right.\\ \hline

op1 SHIFT op2    & BIT(lg)         & FIXED(g)        & BIT(lg)           & Shifting of operand op1 for the\\
                 &                 &                 &                   & number of positions determined\\
                 &                 &                 &                   & operand op2. If operand op2 $>$ 0,\\
                 &                 &                 &                   & op1 is shifted to the left,\\
                 &                 &                 &                   & otherwise to the right. In both\\
                 &                 &                 &                   & cases, zeros are padded.\\ 
\hline
\end{tabular}
\end{center}
\end{table}

\begin{table} % 7.8
\begin{center}
\caption{Dyadic character string operators}
\label{tab78}
\vspace{5mm}
\begin{tabular}{|l|l|l|l|l|}
\hline
{\bf Expression} & {\bf Type}      & {\bf Type}      & {\bf Result type} & {\bf Meaning of operator}\\
                 & {\bf operand 1} & {\bf operand 2} &                   & \\ \hline

op1 $><$ op2     & CHAR(lg1)       & CHAR(lg2)       & CHAR(lg3)         & Concatenation of the strings\\ 
or               & BIT(lg1)        & BIT(lg2)        & BIT(lg3)          & op1 and op2. Op2 is linked \\
                 &                 &                 &                   & to op1 on the right side.\\
op1 CAT op2      &                 &                 &                   & lg3 = lg1 + lg2.\\
\hline
\end{tabular}
\end{center}
\end{table}

In order to enable the user to reconstruct the result of a comparison
operation of two character strings, the comparison algorithm is given
here (the shorter character string is filled with blanks on the right
side).

Algorithm to compare two character strings, where it holds ``lg3 =
max(lg1, lg2)'':

{\bf TYPE} StringCheck
{\bf FIXED; DCL} (less, equal, greater) {\bf INV FIXED INIT}(-1, 0, 1);\\
\x string\_comparison: {\bf PROC}((string1, string2) {\bf REF INV CHAR}(lg3)){\bf RETURNS}(StringCheck);\\
\x \x {\bf FOR} i {\bf TO} lg3 {\bf REPEAT}\\
\x \x \x {\bf IF TOFIXED} string1.{\bf CHAR}(i) $<$ {\bf TOFIXED} string2.{\bf CHAR}(i)\\
\x \x \x {\bf THEN} {\bf RETURN}(less);\\
\x \x \x {\bf ELSE IF TOFIXED} string1.{\bf CHAR}(i) $>$ {\bf TOFIXED} string2.{\bf CHAR}(i)\\
\x \x \x \x {\bf THEN}\\
\x \x \x \x {\bf RETURN}(greater);\\
\x \x \x \x {\bf FIN};\\
\x \x \x {\bf FIN};\\
\x \x {\bf END}; ! loop\\
\x \x {\bf RETURN}(equal);\\
\x {\bf END}; ! PROC string\_comparison\\


Two further dyadic standard operators relating to array bounds are defined in Table~\ref{tab79}.

\begin{table} % 7.9 
\begin{center}
\caption{Other dyadic operators}
\label{tab79}
\vspace{5mm}
\begin{tabular}{|l|l|l|l|l|}
\hline
{\bf Expression} & {\bf Type}      & {\bf Type}      & {\bf Result type} & {\bf Meaning of operator}\\
                 & {\bf operand 1} & {\bf operand 2} &                   & \\ \hline

op1 LWB op2      & FIXED(g)        & array           & FIXED(31)         & lower boundary of the dimension\\
                 &                 &                 &                   & (given by op1) of the array\\
                 &                 &                 &                   & (determined by op2), if existing\\ \hline

op1 UPB op2      & FIXED(g)        & array           & FIXED(31)         & upper boundary of the dimension\\
                 &                 &                 &                   & (given by op1) of the array\\
                 &                 &                 &                   & (determined by op2), if existing\\ 
\hline
\end{tabular}
\end{center}
\end{table}

\newpage
Example:

...\\
P: {\bf PROC}(A(,) {\bf FIXED IDENT});\\
\x ... \\
\x {\bf FOR} i {\bf FROM LWB} A {\bf TO UPB} A\\
\x \x {\bf REPEAT} \\
\x \x \x {\bf FOR} k {\bf FROM} 2 {\bf LWB} A {\bf TO} 2 {\bf UPB} A \\
\x \x \x \x {\bf REPEAT} \\
\x \x \x \x \x ...\\
\x \x \x \x {\bf END}; ! for k\\
\x \x ... \\
\x \x {\bf END}; ! for i\\
\x ...\\
\x {\bf END}; ! proc p \\
...\\
{\bf DCL} Tab1(5,10) {\bf FIXED},\\
\x Tab2(-1:2, 3:5) {\bf FIXED};\\
... \\
{\bf CALL} P(Tab1) ;\\
...\\
{\bf CALL} P(Tab2);

\subsection{Evaluation of Expressions}   % 6.1.3

In the sequel, lower-case letters a, b, c ... denote constants or scalar
variables.

According to the rules of arithmetics, the order of calculating an
expression is dependent on the \mbox{(pre-)}rank of the various operators in
the expression. The dyadic operator ``*'', e.g., has a higher rank than
the dyadic operator ``+'': In the expression ``a+b*c'', hence, ``b*c''
is calculated first, and the product is added to a.

The ranking of dyadic operators is defined in Table~\ref{ranks}. Lower
numbers correspond to higher ranks.

\begin{table}
\begin{center}
\caption{Ranks of the operators defined in PEARL}
\label{ranks}
\vspace{5mm}
\begin{tabular}{|l@{\x}|l@{\x}|l|}
\hline
rank & dyadic operators       & evaluation order \\ \hline
1    & **, FIT, LWB, UPB      & from the right to the left\\
2    & *, /, $><$, //, REM    & from the left to the right\\
3    & +, -, $<>$, SHIFT      & from the left to the right\\
4    & $<$, $>$, $<=$, $>=$   & from the left to the right\\
5    & ==, /=, IS, ISNT       & from the left to the right\\
6    & AND                    & from the left to the right\\
7    & OR, EXOR               & from the left to the right\\
\hline
\end{tabular}
\end{center}
\end{table}

All monadic standard operators have rank 1.

The order of calculating an expression is furthermore influenced in the
common way of putting parentheses around parts of the expression; e.g.,
in the expression

a*(b-(c-d))

first c-d is calculated, this first intermediate result is subtracted
from b, and then this second intermediate result is multiplied by a.

In general, an expression is calculated according to following rules:
\begin{itemize}
\item The partial expression with the highest rank operator is
calculated first, unless one of the two following rules is violated.
\item If several operators of the same rank occur, the calculation takes
place
\begin{itemize}
\item from the left to the right in case 2$\ge$rank$\ge$7
\item from the right to the left in case rank=1\\
      Example: -a**b corresponds to -(a**b)
\end{itemize}
\item parenthesised partial expressions are completely calculated due to
the rules above, before they are combined with another partial
expression.
\end{itemize}

\begin{removed}
\section{Operator Definition (OPERATOR)}   % 6.2

The operator definition allows to define new operators with freely
selectable identifiers, or to extend the meaning of the previously
introduced standard operators.

\begin{front}
OperatorDefinition ::=\\
\x {\bf OPERATOR} OpName ( [ OpParameter , ] OpParameter )\\
\x {\bf RETURNS} (ResultType);\\
\x ProcedureBody\\
\x {\bf END};

OpName ::=\\
\x Identifier $\mid + \mid - \mid * \mid ** \mid / \mid // \mid == \mid /= \mid <= \mid >= \mid < \mid > \mid <> \mid ><$

OpParameter ::=\\
\x Identifier [ VirtDimensionList ] ParameterType [ {\bf IDENTICAL $\mid$ IDENT} ]

ResultType ::=\\
\x SimpleType $\mid$ StructuredType $\mid$ TypeReference
\end{front}
\begin{grammar}
\input{OperatorDefinition.bnf}
\input{OpName.bnf}
\input{OpParameter.bnf}
\end{grammar}

The meaning is analogous to the one of a function procedure (cf. 8).


Example:

The standard operator + shall be extended for complex numbers.

{\bf PROBLEM};

{\bf TYPE} Complex {\bf STRUCT}\\
\x [ Real {\bf FLOAT}, Imag {\bf FLOAT} ];

{\bf OPERATOR} + (A Complex {\bf IDENT}, B Complex {\bf IDENT}) {\bf RETURNS} (Complex):\\
\x {\bf DCL} Sum Complex;\\
\x Sum.Real := A.Real + B.Real;\\
\x Sum.Imag := A.Imag + B.Imag;\\
\x {\bf RETURN} (Sum);\\
\x {\bf END}; ! Operator +

{\bf DCL} (XX, YY, ZZ) Complex,\\
\x (X, Y, Z) {\bf FLOAT};\\
...\\
ZZ := XX + YY;\\
Z := X + Y;\\
...

This example shows the possibility to define various operators with the
same operator name, if the operands are of different types. If an
expression is evaluated, in which such an operator name occurs, that
operation is executed where the parameter types are identical with the
operand types in the expression.

If an operator declaration serves for extending the meaning of a
standard operator, the re-declared operator has the rank of the standard
operator. For an operator declared with a new operator name not
introduced in a standard way, a rank between 1 and 7 can be determined
with the help of a rank declaration:

\input{PrecedenceDefinition.bnf}

If a precedence shall be defined for a new operator, this must take place
before the operator definition; if no precedence definition is stated, the
new operator is given precedence 7.

Example:

...
{\bf PRECEDENCE} INDEX(1);\\
{\bf OPERATOR} INDEX...;
\end{removed}


\section{Assignments}   % 6.3

Assignments are implemented for scalar variables and for structures, but
not for arrays.

\begin{front}
Assignment ::=\\
\x ScalarAssignment $\mid$ StructureAssignment
\end{front}
\begin{grammar}
\input{Assignment.bnf}
\end{grammar}

\subsection{Assignments for Scalar Variables}   % 6.3.1

Assignments for scalar variables are defined as follows:

\begin{front}
ScalarAssignment ::=\\
\x Name\S ScalarVariable \{ := $\mid$ = \} Expression;
\end{front}
\begin{grammar}
not in grammar
\end{grammar}

The statement causes that the value of the expression on the right side
is assigned to the variable given on the left of the assignment sign
(``:='' or ``=''), i.e., after the execution of the assignment, this
name can be used to refer to the value determined by this expression
and, if needed, calculated before the actual assignment:

Result(i) := Koeff * SIN((X(i+1) - X(i)) / X(i));

The type of the variable given to the left of the assignment sign has to
match the type of the value of the expression, with the following
exceptions:
\begin{itemize}
\item The value of a FIXED variable or an integer, resp., may be
assigned to a FLOAT variable.
\item The precision of a numeric variable to the left of an assignment
sign may be greater than the precision of the value of the expression.
\item A bit or character string, resp., to the left may have a greater
length than the value to be assigned; if needed, the latter is extended
by zeros or spaces, resp., on the right.
\end{itemize}

Operators for needed type conversions are described in 6.1.2.

Examples:

{\bf DCL} (I, J) {\bf FIXED}(15),\\
\x K {\bf FIXED}(31),\\
\x (X, Y) {\bf FLOAT},\\
\x Bit8 {\bf BIT}(8),\\
\x Bit12 {\bf BIT}(12),\\
\x Text4 {\bf CHAR}(4),\\
\x Text10 {\bf CHAR}(10),\\
\x Duration(2) {\bf DURATION},\\
\x Time(2) {\bf CLOCK};

\begin{tabbing}
Time(2) := Time(1) + \= {\bf FIN}); \= {\bf THEN} \= \kill

I := 2.0; \> \> \> ! wrong \\
J := 3;   \> \> \> \\
X := J+5; Y := 0; \> \> \> \\
K := J;   \> \> \> \\
J := K;   \> \> \> ! wrong\\
Text10 := 'Result'; \> \> \> ! Text10 has value 'Result\_\_' \\
Bit8 := 'A9F'B4;    \> \> \>! wrong, because too long \\
Duration(1) := 1 {\bf HRS}; \> \> \>\\
Duration(2) := 30 {\bf MIN}; \> \> \> \\
Time(1) := 11:00:00; \> \> \>\\
Time(2) := Time(1) + \> ({\bf IF} \> Time(1) $<$ 12:00:00 \> \\
       \>          \> {\bf THEN} Duration(1) \> \\
       \>          \> {\bf ELSE} Duration(2) \> \\
       \> {\bf FIN}); \> \> \\
Bit8 := '10001100'B; \> \> \> \\
Bit12 := Bit8 $><$ '11'B; \> \> \> /* Bit12 has value '100011001100'B */ \\
Bit8 := Bit8 {\bf CSHIFT} 3; \> \> \> /* Bit8 has value '01100100'B */ \\
Bit12 := Bit8 {\bf SHIFT} -6; \> \> \> /* Bit12 has value '000000100011'B */ \\
\end{tabbing}

It is possible to declare variables with an attribute for assignment
protection (cf. 5.15). Assignments to such variables result in error
messages.

\subsection{Assignments for Structures}   % 6.3.2

The values of all components of a structure can be assigned to another
structure in one single assignment:

\begin{front}
StructureAssignment ::=\\
\x Name\S Structure\_1 \{ := $\mid$ = \} ExpressionStructure\_2;
\end{front}
\begin{grammar}
not in grammar
\end{grammar}

The values of the components of Structure\_2 are assigned to the
corresponding components of Structure\_1. Both structures must have the same
type; i.e., the number of components and their types must match; an
implicit type adaptation like for scalar variables takes not place.

Example:

\begin{tabbing}
{\bf TYPE} \= {\bf [} \= \kill

{\bf TYPE} \>         \> Type\_Measurement {\bf STRUCT}\\
    \> {\bf [} \> Time\_Stamp, \\
    \>         \> Value {\bf FLOAT}(53){\bf ]};\\
    \>         \> \\
{\bf DCL}  \>         \> Workpiece {\bf STRUCT} \\
    \> {\bf [} \> Ident {\bf CHAR}(8),\\
    \>         \> Quality Type\_Measurement,\\
    \>         \> ... {\bf ]};\\
    \>         \> \\
{\bf DCL}  \>         \> Measurement Type\_Measurement;\\
...        \>         \> \\
Workpiece.Quality := Measurement;
\end{tabbing}

\begin{removed}
\section{Overloading of Data Structures}   % 6.4

To exclude programming errors to a large extent, the compiler checks the
use of variables and values in accordance with types.

Upon assignments, the types of the left and right side have to match
each other (see 6.3: Assignments), and when calling a procedure, the
actual parameters have to match the formal parameters in type (see 8.2:
Call of Procedures).

However, there are often situations where the required type
compatibility (``strong typing'') proves to be a hindrance. As an
example, we refer to a data base interface: the data base shall store
user data of different types with different length. For this, the
routines only need the address and the length of the structures to be
stored. For strong typing, a new data base interface (with the desired
user types) had to be programmed for each application.

PEARL 90 provides two different syntactic forms for the overloading of
different data types. Both forms allow exclusively the type conversion
of the objects' addresses when assigning or identifying, resp. A data
conversion does not take place in any case. For converting basic types,
there is a sufficient number of standard operations in PEARL (see 6.1:
Monadic and Dyadic Operators).

\subsection{The ``BY TYPE'' Operator} % 6.4.1

With the ``BY TYPE'' operator, the type of a variable address can be
converted to any other pointer type.

\begin{front}
TypeConvertingExpression ::=\\
\x Name {\bf BY TYPE} Type
\end{front}
\begin{grammar}
not in grammar
\end{grammar}

The type of the variable ``Name'' is converted into the given Type by
operator ``BY TYPE''. The result of the type converting expression is
the address of ``Name'' with type ``Type''. This type conversion is a
compiler internal action preventing an error message of the compiler. At
run time, no action is executed; especially the content of ``Name''
stays unchanged.
\end{removed}
\begin{removed}
Example:

Overloading of data objects upon assignment

{\bf DCL} var \x        TYPE\_A;\\
{\bf DCL} ptr {\bf REF} TYPE\_B;\\
...\\
ptr := var {\bf BY TYPE} TYPE\_B;

The address of variable ``var'' receives type ``TYPE\_B''; thus, the
assignment to pointer ``ptr'' is correct. It is now possible to access a
data object of type ``TYPE\_B'', which overloads variable ``var'', via
pointer ``ptr''.

Example:

Overloading of data objects upon procedure call

{\bf SPC} p1 {\bf ENTRY} ({\bf REF} TYPE\_B) {\bf GLOBAL};\\
{\bf SPC} p2 {\bf ENTRY} (TYPE\_B {\bf IDENT}) {\bf GLOBAL};\\
{\bf DCL} var TYPE\_A;\\
...\\
{\bf CALL} p1 (var {\bf BY TYPE} TYPE\_B);\\
{\bf CALL} p2 (var {\bf BY TYPE} TYPE\_B);

The type of the formal parameter ``var'' is converted to type
``TYPE\_B'' by operator ``BY TYPE''. In both cases, the address of
variable ``var'' is passed on to an object of type ``TYPE\_B'' to the
procedures as pointer. Thus, both calls are correct.

By switching off the type control, the compiler cannot indicate errors
any longer. On no account it is allowed to read or write beyond the end
of the basic memory space with the help of an overloaded data structure,
because severe errors can arise upon program execution (a data loss is
just the smallest problem).

The programmer should not make use of any assumptions about data storage
within the overloaded memory area, either. Storing data is generally
dependent on the target machine. Programs converting data with the help
of overloaded data structures are hardware dependent and only portable
with great effort. Access to data should always be executed via the same
data structure, both for writing and for reading later on. Since in
this case, the data are always processed in accordance with their types,
such programs are portable, if the space requirements of the overloading
data structure do not go beyond the supported filler.

\end{removed}
\begin{removed}
\subsection{The ``VOID'' Data Type}    % 6.4.2

In contrast to the active type conversion by the ``BY TYPE'' operator,
the ``VOID'' data type provides a passive possibility to overload
different data objects. Similar to the constant ``NIL'', which can be
assigned to pointer variables of any data types, the compiler accepts
the assignment of any data addresses to a pointer variable of type
``VOID''.

The ``VOID'' data type is written as structure description without
components, and may only be used in combination with REF.

\begin{front}
VOID-DataType ::=\\
\x {\bf STRUCT} [ ]
\end{front}
\begin{grammar}
\input{Type-VOID.bnf}
\end{grammar}

A reference variable with this type can admit any variable address, the
variable, however, cannot be altered directly. First, the address must
be assigned to a reference variable with the needed type, before the
data object can be manipulated via the reference variable.

Example:

General memory administration routine

\begin{tabbing}
{\bf DCL} memory\_block (c\_max\_buffer) \= {\bf FIXED}(31); \= \kill

{\bf DCL} c\_max\_buffer \x    {\bf INV} \> {\bf FIXED}(31)  \> {\bf INIT}(10000);\\
{\bf DCL} memory\_block (c\_max\_buffer) \> {\bf CHAR}(1);   \> \\
{\bf DCL} free                           \> {\bf FIXED}(31)  \> {\bf INIT}(0);\\
{\bf DCL} memory\_protection             \> {\bf SEMA}       \> {\bf PRESET}(1);
\end{tabbing}

Malloc: {\bf PROC}(size {\bf FIXED}(31)){\bf RETURNS}({\bf REF STRUCT}[ ]){\bf GLOBAL};\\
\x {\bf DCL} ptr {\bf REF CHAR}(1);

\x {\bf REQUEST} memory\_protection; \x /* synchronize access to global variables */

\x {\bf IF} free + size $>=$ c\_max\_buffer {\bf THEN}\\
\x \x ptr := {\bf NIL}; \x /* no space left */\\
\x {\bf ELSE} \\
\x \x ptr := memory\_block(free + 1);\\
\x \x free := free + size;\\
\x {\bf FIN};

\x {\bf RELEASE} memory\_protection;\\
\x {\bf RETURN}(ptr);\\
{\bf END}; ! Malloc

t: {\bf TASK};\\
\x {\bf DCL} ptr\_var\_a {\bf REF} TYPE\_A;\\
\x {\bf DCL} ptr\_var\_b {\bf REF} TYPE\_B;

\x /* require memory range for different types */\\
\x ptr\_var\_a := Malloc ({\bf SIZEOF} TYPE\_A);\\
\x ptr\_var\_b := Malloc ({\bf SIZEOF} TYPE\_B);\\
{\bf END}; ! t

\end{removed}
\begin{removed}
This second form is particularly suitable for polymorphic procedures,
i.e., procedures called with actual parameters of different types. The
``VOID'' data type serves on the one hand as indication in the procedure
specification that the procedure works with different parameter types,
on the other hand, an explicit type conversion with the ``BY TYPE''
operator can be omitted in the call position.

Example (Data base interface):

\begin{tabbing}
{\bf SPC} write\_record \= {\bf ENTRY}(adr {\bf REF INV} \= {\bf STRUCT}[ ], size {\bf FIXED}(31)) {\bf GLOBAL}; \kill
{\bf SPC} read\_record  \> {\bf ENTRY}(adr {\bf REF}     \> {\bf STRUCT}[ ], size {\bf FIXED}(31)) {\bf GLOBAL}; \\
{\bf SPC} write\_record \> {\bf ENTRY}(adr {\bf REF INV} \> {\bf STRUCT}[ ], size {\bf FIXED}(31)) {\bf GLOBAL};
\end{tabbing}

...\\
{\bf DCL} user\_data TYPE\_A;

/* add record with user data into data base */\\
{\bf CALL} write\_record (user\_data, {\bf SIZEOF} TYPE\_A);

/* read user record from data base */\\
{\bf CALL} read\_record (user\_data, {\bf SIZEOF} TYPE\_A);

The data base routines ``read\_record'' and ``write\_record'' are
called by various user programs. They can store and read records of
different length.  For this, the data base routines need only the
addresses and lengths of the records, but not the user specific data
types.
\end{removed}

