\appendix
\chapter{Appendix}

% --------
% enable hypertargets for syntax elements
\renewcommand{\htgt}{\htgtOn}

\section{Data Types and their usability}    % A1
\label{sec_data_types_usability}

The following overview shows for each of the available data types, whether
objects of this type may

\begin{itemize}
\item be summarised to arrays,
\item occur as structure components,
\item be formal procedure parameters,
\item be results of a function procedure,
\item be values of a reference variable,
\item be transmitted to or from data stations,
\item be provided with allocation protection,
\item be global, or
\item be provided with the initialisation attribute.
\end{itemize}

\begin{discuss}
IRPT als Parameter oder REF value zulaessig?

bei SIGNALs geht nur GLOBAL; die restlichen 'x' raus?

REF STRUCT[] entfernt

Dies ist Thema auf der AK-Sitzung im November 2016.


\end{discuss}
\begin{table}[htb]
\begin{tabular}{lccccccccc}
{\bf type}     & \multicolumn{9}{c}{{\bf usage}} \\ 
               & array & struct. & para- & result & ref.  & dation & INV & GLOBAL & INIT \\ 
               &       &         & meter & type   & value & class  &     &        &      \\ \hline
FIXED          & x     &  x      & x     &  x     &  x    &   x    &  x  &    x   &  x   \\
FLOAT          & x     &  x      & x     &  x     &  x    &   x    &  x  &    x   &  x   \\
BIT            & x     &  x      & x     &  x     &  x    &   x    &  x  &    x   &  x   \\
CHAR           & x     &  x      & x     &  x     &  x    &   x    &  x  &    x   &  x   \\
CLOCK          & x     &  x      & x     &  x     &  x    &   x    &  x  &    x   &  x   \\
DUR            & x     &  x      & x     &  x     &  x    &   x    &  x  &    x   &  x   \\
SEMA           & x     &  --     & x     &  --    &  x    &   --   &  -- &    x   &  --  \\
BOLT           & x     &  --     & x     &  --    &  x    &   --   &  -- &    x   &  --  \\
IRPT           & x     &  --     & x     &  --    &  x    &   --   &  -- &    x   &  --  \\
SIGNAL         & x     &  --     & x     &  --    &  x    &   --   &  -- &    x   &  --  \\
DATION         & x     &  --     & x     &  --    &  x    &   --   &  -- &    x   &  --  \\
array          & --    &  x      & x     &  --    &  x    &   --   &  x  &    x   &  x   \\
STRUCT         & x     &  x      & x     &  x     &  x    &   x    &  x  &    x   &  x   \\
new type       & x     &  x      & x     &  x     &  x    &   x    &  x  &    x   &  x   \\
REF            & x     &  x      & x     &  x     &  --   &   --   &  x  &    x   &  x   \\
procedure      & --    &  --     & --    &  --    &  x    &   --   &  -- &    x   &  --  \\
TASK           & --    &  --     & --    &  --    &  x    &   --   &  -- &    x   &  --  \\
FORMAT         & --    &  --     & --    &  --    &  --   &   --   &  -- &    --  &  --  \\
REF CHAR ( )   & x     &  x      & x     &  --    &  --   &   --   &  x  &    x   &  x   \\
REF PROC       & x     &  x      & x     &  --    &  --   &   --   &  x  &    x   &  x   \\
REF TASK       & x     &  x      & x     &  x     &  --   &   --   &  x  &    x   &  x   \\
%%% REF STRUCT [ ] & x     &  x      & x     &  x     &  --   &   --   &  x  &    x   &  x   \\
\end{tabular}
\end{table}
                                                                                 
Objects of type SEMA, BOLT, IRPT, SIGNAL, DATION or array may only be 
passed on by means of identification (IDENT) as procedure parameters.

\newpage
\section{Predefined Functions}   % A 2

This appendix describes the functions known to the PEARL
% compiler
System. They
can be used in the single modules 
%%%without
by
%%% ^^added
 specifying them before. 
%%%If one 
%%%of the functions is specified in a module, no object may exist with the 
%%%functions' name at module level.
%%%\begin{added}
A user supplied global function with the same signature substitutes
the predefined function in the whole program.
%%%\end{added}

\subsection{The Function NOW}    % A 2.1

The function procedure NOW passes back the actual time or system time, resp., 
as value of type CLOCK. The specification of the function looks like this:\\

{\bf SPC} NOW {\bf PROC RETURNS ( CLOCK ) GLOBAL} ; 


\subsection{The Function DATE}    % A 2.2

The actual date can be received by invoking the function procedure DATE.
The function result is a character string constant of length 10, containing
the date in the form ``year--month--day''. Here an example for December 5,
1991: ``1991--12-05''. The function must  be specified like this:\\

{\bf SPC} DATE {\bf PROC RETURNS ( CHAR(10) ) GLOBAL} ;

%%%\begin{added}
\section{CPP-Insertions}
The language translation is done in two steps.
Step 1 translates \OpenPEARL{} into C++.
Step 2 translates C++ into the plattform specific machine code.

It is possible to insert C++ statements into the generated C++ 
code. The position of the insertion is
\begin{discuss}
\begin{itemize}
\item in system part: {\em not defined yet}
\item on module level in the problem part 
   it is not clearly defined, since the first compilation
   steps adds some information for administration in between.
\item in tasks and procedures just between the output of the surrounding
statements.
\end{itemize}
\end{discuss}

The syntax is.

CPPInsertion ::=\\
\x {\bf \_\_cpp\_\_} ( CharacterStringConstant ... );

The CharacterStringConstants are expected in PEARL notation.
Some characters must be marked with the switch symbol as shown in the table
below:

\begin{tabular}{l|l}
C & PEARL \\
\hline
\verb|\| & \verb|\\| \\
\verb|'| & \verb|\'|  \\
\end{tabular}
%%%\end{added}

\newpage
\section{Syntax}    % A 3
\begin{discuss}
die Syntaxelemente sind ggf. noch besser zu ordnen.
Es koennen auch noch ueberzaehlige Elemente vorhanden sein!

ParameterListForSPC kann vermutlich durcgListOfFormalParameters ersetzt 
werden.

%%%Ist IdentifierDenotation das gleiche wie OneIdentifierOrList?
%%% ja --> wurde komplett ersetzt
%%% war so definiert:
%%%IdentifierDenotation ::=\\
%%%\x Identifier $\mid$ (Identifier [ , Identifier ] $^{...}$ )
%%%
%%% Indicator wurde durch Identifier ersetzt

\end{discuss}

Following meta characters are used in the syntax description:\\ 

\begin{tabular}{ll}
meta character & meaning \\ \hline
::=            & introduction of a Name (nonterminal symbol) for a language form\\
$[\ ]$         & bracketing of optional parts of a language form \\ 
$\mid$         & separation of alternative parts of a language form\\
\{ \}          & putting together several elements to a new element\\
$^{...}$       & one or multiple repetition of the preceding element\\
               & (or of several elements bracketed by \{ \} or [ ] ) \\
$\S $          & separates an explaining comment from a language form Name\\
$/^*$ $^*/$    & comment brackets: 
                 includes an explaining text, possibly explaining the\\
               & language form in detail instead of a formal description\\
\end{tabular}

All other elements occurring in the syntax rules are either Names of language
forms or terminal symbols. Examples for terminal symbols are the PEARL
keywords (printed boldly) or the characters semicolon ``;'', opening round
bracket ``('' and closing round bracket ``)'', 
% Klammer auf und zu ???
or the apostrophe `` ' ''; the terminal symbols opening square bracket ``[''
and closing square bracket ``]'' are printed boldly to distinguish them
from the meta symbols for optional parts. Attention: the round brackets are
no meta characters and have thus no grouping effect!\\

The symbol PEARL-Program is the initial symbol of the following syntax
description.

\subsection{Program}    % A 3.1

\input{Program.bnf}

\input{Module.bnf}

\subsection{System Part}   % A 3.2

\input{SystemPart.bnf}

\input{UserNameDeclaration.bnf}
\input{SystemDationDeclaration.bnf}
\input{SystemSignalDeclaration.bnf}
\input{SystemInterruptDeclaration.bnf}

\input{ConfigurationElement.bnf}
\input{ListOfConstants.bnf}
\input{ConstantParameter.bnf}
\input{Association.bnf}

%%%\begin{removed}
%%%nngz ::= \\
%%%\x IntegerWithoutPrecision$\S $NonNegative
%%%\end{removed}


\subsection{Basic Elements}    % A 3.3 

\htgt{Digit} ::=\\
\x 0 $\mid$ 1 $\mid$ 2 $\mid$ 3 $\mid$ 4 $\mid$ 5 $\mid$ 6 $\mid$ 7 $\mid$ 8 $\mid$ 9\\

\htgt{Letter} ::= \\
\x A $\mid$ B $\mid$ C $\mid$ ... $\mid$ Z $\mid$ a $\mid$ b $\mid$ c $\mid$ ... $\mid$ z\\

\htgt{Identifier} ::=\\
\x \hlink{Letter} [ \hlink{Letter} $\mid$ \hlink{Digit} $\mid$ \_ ] $^{...}$\\

\input{Constant.bnf}

\input{Integer.bnf} 

\input{IntegerWithoutPrecision.bnf}

\input{Precision.bnf}

\htgt{FloatingPointNumber} ::=\\
\x \hlink{FloatingPointNumberWithoutPrecision} [ ( \hlink{Precision} ) ]\\

\htgt{FloatingPointNumberWithoutPrecision} ::=\\
\x \{ \hlink{Digit} $^{...}$ . [ \hlink{Digit} $^{...}$ ] $\mid$ . \hlink{Digit} $^{...}$ \} [ \hlink{Exponent} ]\\
\x $\mid$ \hlink{Digit} $^{...}$ \hlink{Exponent}\\

\htgt{Exponent} ::=\\
\x E [ + $\mid$ - ] \hlink{Digit} $^{...}$\\

\htgt{BitStringConstant} ::=\\
\x ' \hlink{B1-Digit} $^{...}$ ' \{ {\bf B} $\mid$ {\bf B}1 \} $\mid$ ' \hlink{B2-Digit} $^{...}$ ' {\bf B}2 $\mid$ ' \hlink{B3-Digit} $^{...}$ ' {\bf B}3 $\mid$ ' \hlink{B4-Digit}$^{...}$ ' {\bf B4}\\

\htgt{B1-Digit} ::= 0 $\mid$ 1\\

\htgt{B2-Digit} ::= 0 $\mid$ 1 $\mid$ 2 $\mid$ 3\\

\htgt{B3-Digit} ::= 0 $\mid$ 1 $\mid$ 2 $\mid$ 3 $\mid$ 4 $\mid$ 5 $\mid$ 6 $\mid$ 7\\

\htgt{B4-Digit} ::= 0 $\mid$ 1 $\mid$ 2 $\mid$ 3 $\mid$ 4 $\mid$ 5 $\mid$ 6 $\mid$ 7 $\mid$ 8 $\mid$ 9 $\mid$ A $\mid$ B $\mid$ C $\mid$ D $\mid$ E $\mid$ F\\

\htgt{CharacterStringConstant} ::=\\
\x ' \{ \hlink{CharacterBesidesApostrophe} $\mid$ '' $\mid$ \hlink{ControlCharacterSequence} \} $^{...}$ '\\

\htgt{CharacterBesidesApostrophe} ::=\\
\x \hlink{Digit} $\mid$ \hlink{Letter} $\mid$ \x $\mid$ \_ $\mid$ + $\mid$ - $\mid$ $^*$ $\mid$ / $\mid$ $\backslash$ $\mid$ ( $\mid$ ) $\mid$ {\bf [} $\mid$ {\bf ]} $\mid$ : $\mid$ . $\mid$ ; $\mid$ , $\mid$ = $\mid$ $<$ $\mid$ $>$ $\mid$ !\\
\x $\mid$ $/^*$ more printable characters of the machine character set $*/$ \\

\htgt{ControlCharacterSequence} ::= \\
\x '$\backslash$ \{ \hlink{B4-Digit} \hlink{B4-Digit} \} $^{...}$ $\backslash$'\\

\htgt{TimeConstant} ::=\\
\x \hlink{Digit} $^{...}$ : \hlink{Digit} $^{...}$ : \hlink{Digit} $^{...}$ [ . \hlink{Digit} $^{...}$ ] \\

\htgt{DurationConstant} ::=\\
\x \hlink{Hours} [ \hlink{Minutes} ] [ \hlink{Seconds} ]\\
\x $\mid$ \hlink{Minutes} [ \hlink{Seconds} ]\\
\x $\mid$ \hlink{Seconds}\\

\htgt{Hours} ::=\\
\x \hlink{IntegerWithoutPrecision} {\bf HRS}\\

\htgt{Minutes} ::=\\
\x \hlink{IntegerWithoutPrecision} {\bf MIN}\\

\htgt{Seconds} ::=\\
\x \{ \hlink{IntegerWithoutPrecision} 
	$\mid$ \hlink{FloatingPointNumberWithoutPrecision} \} {\bf SEC}

\input{ConstantExpression.bnf}

\input{ConstantFIXEDExpression.bnf} 
        
\input{Term.bnf}

\input{Factor.bnf}

\subsection{Problem Part}   % A 3.4

\input{ProblemPart.bnf}

\subsubsection{Declaration}    % A 3.4.1

\htgt{Declaration} ::=\\
\x \hlink{LengthDefinition} \\
\x $\mid$ \hlink{TypeDefinition}\\
\x $\mid$ \hlink{VariableDeclaration}\\
\x $\mid$ \hlink{FormatDeclaration}\\
\x $\mid$ \hlink{ProcedureDeclaration}\\
\x $\mid$ \hlink{TaskDeclaration}\\
%%%%%\begin{removed}
%%%%%\x $\mid$ OperatorDefinition\\
%%%%%\x $\mid$ PredenceDefinition\\
%%%%%\end{removed}


\input{LengthDefinition.bnf} 

\input{TypeDefinition.bnf}
\input{Type.bnf}
\input{TypeFixed.bnf}
\input{TypeFloat.bnf}
\input{TypeBitString.bnf}
\input{TypeCharacterString.bnf}
\input{TypeClock.bnf}
\input{TypeDuration.bnf}

\input{LengthOfString.bnf}

\input{ScalarVariableDeclaration.bnf}
%%%\input{VariableDenotation.bnf}
\input{ScalarDeclareDataSentence.bnf}
\input{TypeAttribute.bnf}

\begin{discuss}
sind DeclareSentence und DationAttribute noch notwendig?

Es gibt ScalarVariableDeclaration, Bezug zu VariableDeclaration?

\htgt{VariableDeclaration} ::=\\
\x \{ {\bf DECLARE $\mid$ DCL} \} \hlink{DeclareSentence} [ , \hlink{DeclareSentence} ] $^{...}$ ;\\
% wie ist "Satz" hier gemeint ???

\htgt{DeclareSentence} ::=\\
\x \hlink{OneIdentifierOrList} [ \hlink{DimensionAttribute} ]\\
\x \{ \hlink{ProblemDataAttribute} $\mid$ \hlink{SemaAttribute} $\mid$ \hlink{BoltAttribute} $\mid$ \hlink{DationAttribute} \}\\ 
\end{discuss}


\begin{discuss}
Doppelte Regel mit DationDeclaration

\htgt{DationAttribute} ::=\\
\x \hlink{TypeDation}\\
\x [ \hlink{GlobalAttribute} ]\\
\x {\bf CREATED} ( \hlink{Identifier}$\S $UserNameForSytemDation )\\
\end{discuss}


\input{OneIdentifierOrList.bnf}

\input{DimensionAttribute.bnf}

\input{DimensionBoundaries.bnf}
        
\htgt{ProblemDataAttribute} ::=\\
\x [ {\bf INV} ] \{ \hlink{SimpleType} $\mid$ \hlink{StructuredType} 
	$\mid$ \hlink{TypeReference} \}\\
\x [ \hlink{GlobalAttribute} ] [ \hlink{InitialisationAttribute} ]\\

\input{SimpleType.bnf}

\htgt{StructuredType} ::=\\
\x \hlink{TypeStructure} $\mid$ \hlink{Identifier}$\S $ForNewDefinedType\\

% war schon mal da \input{TypeStructure.bnf}

\input{StructureComponent.bnf}

\input{TypeAttributeInStructureComponent.bnf}

\input{TypeReference.bnf}

\input{VirtualDimensionList.bnf}

%%%\begin{removed}
%%%\input{Type-VOID.bnf}
%%%\end{removed}

\input{ArrayDeclaration.bnf}
\input{ArrayDenotation.bnf}
\input{TypeAttributeForArray.bnf}
% war schon mal da \input{DimensionAttribute.bnf}
% war schon mal da \input{DimensionBoundaries.bnf}

\input{ArraySpecification.bnf}
\input{ArrayDenotationS.bnf}

\input{StructureDeclaration.bnf}
\input{StructureDenotation.bnf}
\input{TypeStructure.bnf}
% ist schon mal da  \input{StructureComponent.bnf}
% ist schon mal da \input{TypeAttributeInStructureComponent.bnf}
\input{StructureSpecification.bnf}
\input{StructureDenotationS.bnf}

\htgt{SemaAttribute} ::=\\
\x {\bf SEMA} [ \hlink{GlobalAttribute} ]\\
\x [ {\bf PRESET} ( \hlink{ConstantFIXEDExpression}
	[ , \hlink{ConstantFIXEDExpression} ] $^{...}$ ) ]\\

\htgt{BoltAttribute} ::=\\
\x {\bf BOLT} [ \hlink{GlobalAttribute} ]\\

\input{GlobalAttribute.bnf}

\input{InitialisationAttribute.bnf}

\input{InitElement.bnf}


\input{DationDeclaration.bnf}

%%%\input{DationAttribute.bnf}

\input{TypeDationSystem.bnf}
\input{TypeDation.bnf}

\input{SourceSinkAttribute.bnf}

%%%%%\begin{removed}
%%%%%ClassAttribute ::= \\
%%%%%\x {\bf ALPHIC $\mid$ BASIC} $\mid$ TypeOfTransmissionData\\
%%%%%\end{removed}

%%%%%\begin{added}
\input{ClassAttribute.bnf}
%%%%%\end{added}

\input{TypeOfTransmissionData.bnf}
\input{IOCompoundType.bnf}
\input{IOStructure.bnf}
\input{IOStructureComponent.bnf}

\input{Typology.bnf}

\input{AccessAttribute.bnf}

\htgt{FormatDeclaration} ::= \\
\x \hlink{Identifier} : {\bf FORMAT} ( \hlink{FormatOrPosition} [ , \hlink{FormatOrPosition} ] $^{...}$ ) ;\\

\input{ProcedureDeclaration.bnf}

\input{ProcedureBody.bnf}

\input{ListOfFormalParameters.bnf}
        
\input{FormalParameter.bnf}

\input{ParameterType.bnf}

\input{TypeRealTimeObject.bnf}

\input{ResultAttribute.bnf}

\input{ResultType.bnf}

\input{TaskDeclaration.bnf}

\input{TaskBody.bnf}
        
\input{PriorityAttribute.bnf}

\begin{removed}
\input{OperatorDefinition.bnf}

\input{OpName.bnf}

\input{OpParameter.bnf}

\input{PrecedenceDefinition.bnf}

\end{removed}

\subsection{Specification and Identification}    % A 3.4.2
\begin{discuss}
Auswirkung auf alte Teile pruefen

\input{ScalarVariableSpecification.bnf}
\input{VariableDenotationS.bnf}
\end{discuss}


\htgt{Specification} ::=\\
\x \{ {\bf SPC $\mid$ SPECIFY} \} \hlink{SpecificationDefinition} [ , \hlink{SpecificationDefinition} ] $^{...}$ ;\\

\htgt{SpecificationDefinition} ::=\\
\x \hlink{OneIdentifierOrList}\\
\x \{ \hlink{SpecificationAttribute} $\mid$ \hlink{ProcedureUsageAttribute} $\mid$ \hlink{TaskUsageAttribute} \}\\

\htgt{SpecificationAttribute} ::=\\
\x [ \hlink{VirtualDimensionList} ] \\
\x \{ [ {\bf INV} ] \{ \hlink{SimpleType} $\mid$ 
	\hlink{StructuredType} $\mid$
	\hlink{TypeReference} \} \\
\x \x $\mid$ {\bf SEMA $\mid$ BOLT $\mid$ 
	INTERRUPT $\mid$ IRPT $\mid$ SIGNAL} 
	$\mid$ \hlink{TypeDation}\\
\x \}\\
\x [ \hlink{GlobalAttribute} ]\\

\htgt{ProcedureUsageAttribute} ::=\\
\x \{ {\bf ENTRY} $\mid$ [ : ] {\bf PROC} \} [ 
	\hlink{ParameterListForSPC} ] [ \hlink{ResultAttribute} ] \\
\x \x \hlink{GlobalAttribute}\\

%%%\begin{removed}
%%%\htgt{ParameterListFor-SPC} ::=\\
%%%\x ( \hlink{FormalParameterFor-SPC} 
%%%	[ , \hlink{FormalParameterFor-SPC} ] $^{...}$ )\\
%%%      
%%%\htgt{FormalParameterFor-SPC} ::=\\
%%%\x [ \hlink{Identifier}$\S $OnlyForDocumentation ] 
%%%	[ \hlink{VirtualDimensionList} ]\\
%%%\x \hlink{ParameterType} [ {\bf IDENTICAL $\mid$ IDENT} ]\\
%%%\end{removed}
\input{ProcedureSpecification.bnf}
\input{ListOfParametersForSPC.bnf}
\input{ParameterSpecification.bnf}
\input{TypeProcedure.bnf}

\htgt{TaskUsageAttribute} ::=\\
\x {\bf TASK} \hlink{GlobalAttribute}\\

\input{Identification.bnf}


\input{IdentificationAttribute.bnf}

\subsection{Expressions}    % A 3.4.3

\input{Expression.bnf}

\input{MonadicOperator.bnf}

\input{DyadicOperator.bnf}

\input{Operand.bnf}

\input{Name.bnf}

\input{Index.bnf}

\input{FunctionCall.bnf}

\input{ListOfActualParameters.bnf}

%%%\begin{discuss}
%%%aus Sprache gestrichen
%%%
%%%\input{ConditionalExpression.bnf}  %%% noch nicht in front genutzt!
%%%\end{discuss}

\input{Dereferentiation.bnf}

\input{StringSelection.bnf}

\subsection{Statements}    % A 3.4.4

\htgt{Statement} ::=\\
\x [ \hlink{Identifier}$\S $Label : ] $^{...}$ \hlink{UnlabelledStatement}\\

\htgt{UnlabelledStatement} ::=\\
\x \hlink{EmptyStatement} $\mid$ \hlink{Assignment} $\mid$ \hlink{Block} $\mid$ \hlink{SequentialControlStatement}\\
\x $\mid$ \hlink{RealTimeStatement} $\mid$ \hlink{ConvertStatement} $\mid$ \hlink{I/O-Statement}\\

\input{EmptyStatement.bnf}

\input{Assignment.bnf}
\input{ScalarAssignment.bnf}
\input{StructureAssignment.bnf}
\input{RefProcAssignment.bnf}

\input{Block.bnf}

\input{SequentialControlStatement.bnf}

\input{IfStatement.bnf}

\input{CaseStatement.bnf}

\input{CaseStatement1.bnf}

\input{CaseStatement2.bnf}

\input{CaseIndex.bnf}

\input{CaseList.bnf}

\input{IndexRange.bnf}

\input{LoopStatement.bnf}
        
\input{ExitStatement.bnf}

\input{ProcedureCall.bnf}

\input{ReturnStatement.bnf}

\input{GoToStatement.bnf}

\htgt{RealTimeStatement} ::=\\
\x \hlink{TaskControlStatement} $\mid$ \hlink{TaskCoordinationStatement}\\
\x $\mid$ \hlink{InterruptStatement} $\mid$ \hlink{SignalStatement}\\

\htgt{TaskControlStatement} ::=\\
\x \hlink{TaskStarting} $\mid$ \hlink{TaskTerminating}\\             
\x $\mid$ \hlink{TaskSuspending} $\mid$ \hlink{TaskContinuing}\\   
\x $\mid$ \hlink{TaskResuming} $\mid$ \hlink{TaskPreventing}\\

\input{TaskStarting.bnf}

\input{SimpleStartCondition.bnf}

\input{PriorityExpression.bnf}

\input{StartCondition.bnf}

\input{Frequency.bnf}

\input{TaskTerminating.bnf}

\input{TaskSuspending.bnf}

\input{TaskContinuing.bnf}

\input{TaskResuming.bnf}

\input{TaskPreventing.bnf}

\input{TaskCoordinationStatement.bnf}
\input{SemaStatement.bnf}
\input{BoltStatement.bnf}

\input{InterruptStatement.bnf}

%%%%%\begin{removed}
%%%%%SchedulingSignalReaction ::=\\
%%%%%\x {\bf ON} Name$\S $Signal \{ [ 
%%%%%	{\bf RST} ( Name$\S $ErrorVariable-FIXED ) ] :\\
%%%%%\x \x SignalReaction $\mid$ 
%%%%%	{\bf RST} ( Name$\S $ErrorVariable-FIXED ) \};\\
%%%%%
%%%%%SignalReaction::= UnlabeledStatement\\
%%%%%
%%%%%InduceStatement ::=\\
%%%%%\x {\bf INDUCE} Name$\S $Signal 
%%%%%	[ {\bf RST} ( Expression$\S $ErrorNumber-FIXED ) ] ;\\
%%%%%\end{removed}

%%%%%\begin{added}
\input{SignalStatement.bnf}
\input{SignalReaction.bnf}
\input{SignalFinalStatement.bnf}
\input{InduceStatement.bnf} 
%%%%%\end{added}

\htgt{ConvertStatement} ::=\\
\x \hlink{ConvertToStatement} $\mid$ \hlink{ConvertFromStatement}\\

\input{ConvertToStatement.bnf}

\input{ConvertFromStatement.bnf}

\input{FormatOrPositionConvert.bnf}

\input{PositionConvert.bnf}

\htgt{I/O-Statement} ::=\\
\x \hlink{OpenStatement} $\mid$ \hlink{CloseStatement}\\
\x $\mid$ \hlink{PutStatement} $\mid$ \hlink{GetStatement}\\
\x $\mid$ \hlink{WriteStatement} $\mid$ \hlink{ReadStatement}\\
\x $\mid$ \hlink{SendStatement} $\mid$ \hlink{TakeStatement}\\
         
\input{OpenStatement.bnf}

\input{OpenParameter.bnf}

\input{CloseStatement.bnf}

\input{CloseParameter.bnf}

\input{PutStatement.bnf}

\input{GetStatement.bnf}


\input{WriteStatement.bnf}
        
\input{ReadStatement.bnf}
        
%%%%%\begin{removed}
%%%%%SendStatement ::=\\
%%%%%\x {\bf SEND} [ \hlink{Expression} ] {\bf TO} \hlink{Name}$\S $Dation\\
%%%%%\x [ {\bf BY} RST-S-CTRL-Format [ , RST-S-CTRL-Format ] $^{...}$ ] ;\\
%%%%%
%%%%%TakeStatement ::=\\
%%%%%\x {\bf TAKE} [ \hlink{Name} ] {\bf FROM} \hlink{Name}$\S $Dation\\
%%%%%\x [ {\bf BY} RST-S-CTRL-Format [ , RST-S-CTRL-Format ] $^{...}$ ] ;\\
%%%%%
%%%%%RST-S-CTRL-Format ::=\\
%%%%%\x {\bf RST} ( \hlink{Name}$\S $ErrorVariable-FIXED )\\
%%%%%\x $\mid$ {\bf S} ( \hlink{Name}$\S $Variable-FIXED )\\
%%%%%\x $\mid$ {\bf CONTROL} ( \hlink{Expression} [ , \hlink{Expression} [ , \hlink{Expression} ] ] )\\
%%%%%\end{removed}

%%%%%\begin{added}
\input{SendStatement.bnf} 

\input{TakeStatement.bnf}

\input{RSTFormat.bnf}

%%%%%\end{added}

\input{Slice.bnf}

\input{FormatOrPosition.bnf}
      
%%%%%\begin{removed}  
%%%%%Factor ::=\\
%%%%%\x ( \hlink{Expression}$\S $IntegerGreaterZero ) $\mid$ \hlink{IntegerWithoutPrecision}$\S $GreaterZero\\
%%%%%
%%%%%.. and sustituted Factor by FormatFactor in Convert and Put/GET statements
%%%%%\end{removed}  
%%%%%\begin{added}  
\input{FormatFactor.bnf}

%%%%%\end{added}  

\input{Format.bnf}
\input{FixedFormat.bnf}
\input{FloatFormat.bnf}
\input{CharacterStringFormat.bnf}
\input{BitFormat.bnf}
\input{TimeFormat.bnf}
\input{DurationFormat.bnf}
\input{ListFormat.bnf}
\input{RFormat.bnf}
\input{FieldWidth.bnf}
\input{DecimalPositions.bnf}
\input{Significance.bnf}

\input{Position.bnf}
\input{AbsolutePosition.bnf}
\input{RelativePosition.bnf}

\twocolumn[\section{List of Keywords with Shortforms}   % A 4
\begin{discuss}
das muss noch auf index, makeindex und printindex 
mit Seitennummer
umgestellt werden.
13.5 und 13.6 fallen dann zusammen.

\end{discuss}
The denotation behind the keyword refers to the paragraph where it
is introduced.\\]

{
{\bf ACTIVATE} 9.2.2\\
{\bf AFTER} 9.2.1\\
{\bf ALL} 9.2.1\\
{\bf ALPHIC} \ref{sec_dation_problem_part}\\
{\bf ALT} 7.2\\
{\bf AT} 9.2.1\\

{\bf BASIC} \ref{sec_dation_problem_part}\\
{\bf BEGIN} 4.4 \\
{\bf BIT} 5.4\\
{\bf BOLT} 9.3.2\\
{\bf BY} 7.3, \ref{sec_open_close}\\

{\bf CALL} 8.2\\
{\bf CASE} 7.2\\
{\bf CHARACTER}, {\bf CHAR} 5.5\\
{\bf CLOCK} 5.7\\
{\bf CLOSE} \ref{sec_open_close}\\
{\bf CONT} 5.9\\
{\bf CONTINUE} 9.2.5\\
{\bf CONTROL} \ref{sec_dation_problem_part}\\
{\bf CONVERT} \ref{sec_dation_problem_part}\\
{\bf CREATED} \ref{sec_dation_problem_part}\\
{\bf CYCLIC} \ref{sec_dation_problem_part}\\
 
{\bf DATION} \ref{sec_dation_problem_part}\\
{\bf DECLARE}, {\bf DCL} 4.1\\
{\bf DIM} \ref{sec_dation_problem_part}\\
{\bf DIRECT} \ref{sec_dation_problem_part}\\
{\bf DISABLE} 9.4.2\\
{\bf DURATION}, {\bf DUR} 5.8\\
{\bf DURING} 9.2.1\\

{\bf ELSE} 7.1\\
{\bf ENABLE} 9.4.2\\
{\bf END} 4.4, 8.1, 9.1\\
{\bf ENTER} 9.3.2\\
{\bf ENTRY} 8.1\\
{\bf EXIT} 7.5\\

{\bf FIN} 7.1\\
{\bf FIXED} 5.2\\
{\bf FLOAT} 5.3\\
{\bf FOR} 7.3\\
{\bf FORBACK} \ref{sec_dation_problem_part}\\
{\bf FORMAT} 10.5.8\\
{\bf FORWARD} \ref{sec_dation_problem_part}\\
{\bf FREE} 9.3.2\\
{\bf FROM} 7.3, \ref{sec_read_write}\\
 
{\bf GET} \ref{sec_get_put}\\
{\bf GLOBAL} 4.1\\
{\bf GOTO} 7.4\\
 
{\bf HRS} 5.8\\
 
{\bf IDENTICAL}, {\bf IDENT} 4.1, 8.1\\
{\bf IF} 7.1\\                           
{\bf IN} \ref{sec_dation_problem_part}\\
{\bf INDUCE} 11\\
{\bf INITIAL}, {\bf INIT} 5.14\\
{\bf INOUT} \ref{sec_dation_problem_part}\\
{\bf INTERRUPT}, {\bf IRPT} 9.4.1\\
{\bf INV} 5.15\\
 
{\bf LEAVE} 9.3.2\\
{\bf LENGTH} 5.6\\
 
{\bf MAIN} 9.1\\
{\bf MAX} \ref{sec_dation_problem_part}\\
{\bf MIN} 5.8\\
{\bf MODEND} 4.1\\
{\bf MODULE} 4.1\\
 
{\bf NIL} 5.9\\
{\bf NOCYCL} \ref{sec_dation_problem_part}\\
{\bf NOSTREAM} \ref{sec_dation_problem_part}\\

{\bf ON} 11\\
{\bf OPEN} 10.3\\
{\bf OPERATOR} 6.2\\
{\bf OUT} 7.2, \ref{sec_dation_problem_part}\\
 
{\bf PRECEDENCE} 7.2\\
{\bf PRESET} 9.3.1\\
{\bf PREVENT} 9.2.7\\
{\bf PRIORITY}, {\bf PRIO} 9.1\\
{\bf PROBLEM} 4.3\\
{\bf PROCEDURE}, {\bf PROC} 8.1\\
{\bf PUT} \ref{sec_get_put}\\

{\bf READ} \ref{sec_dation_read_write}\\
{\bf REF} 5.9\\
{\bf RELEASE} 9.3.1\\
{\bf REPEAT} 7.3\\
{\bf REQUEST} 9.3.1\\
{\bf RESERVE} 9.3.2\\
{\bf RESUME} 9.2.6\\
{\bf RETURN} 8.2\\
{\bf RETURNS} 8.1\\
 
{\bf SEC} 5.8\\
{\bf SEMA} 9.3.1\\
{\bf SEND} \ref{sec_dation_take_send}\\
{\bf SIGNAL} 11\\
{\bf SPECIFY}, {\bf SPC} 4.1\\
{\bf STREAM} \ref{sec_dation_problem_part}\\
{\bf STRUCT} 5.11\\
{\bf SUSPEND} 9.2.4\\
{\bf SYSTEM} 4.2\\
 
{\bf TAKE} \ref{sec_dation_take_send}\\
{\bf TASK} 9.1\\
{\bf TERMINATE} 9.2.3\\
{\bf TFU} \ref{sec_dation_problem_part}\\
{\bf THEN} 7.1\\
{\bf TO} 7.3, \ref{sec_dation_read_write}\\
{\bf TRIGGER} 9.4.2\\
{\bf TYPE} 5.12\\
 
{\bf UNTIL} 9.2.1\\
 
{\bf WHEN} 9.2.1\\
{\bf WHILE} 7.3\\
{\bf WRITE} \ref{sec_dation_read_write}\\
}
\onecolumn

\twocolumn[\section{Other Word Symbols in PEARL}    % A 5

The denotation behind the word symbol refers to the paragraph where it
is introduced.\\]

{
{\bf A} \ref{sec_dation_a_format}\\
{\bf ABS} 6.1.1\\
{\bf ADV} \ref{sec_dation_read_write}\\
{\bf AND} 6.1.2\\
{\bf ANY} \ref{sec_dation_open_close}\\
{\bf ATAN} 6.1.1\\

{\bf B} \ref{sec_dation_b_format}\\
{\bf B1} \ref{sec_dation_b_format}\\
{\bf B2} \ref{sec_dation_b_format}\\
{\bf B3} \ref{sec_dation_b_format}\\
{\bf B4} \ref{sec_dation_b_format}\\

{\bf CAN} \ref{sec_dation_open_close}\\
{\bf CAT} 6.1.2\\
{\bf COL} \ref{sed_dation_read_write}\\
{\bf CONT} 5.9\\
{\bf COS} 6.1.1\\
{\bf CSHIFT} 6.1.2\\
 
{\bf D} \ref{sec_dation_d_format}\\
{\bf DATE} A2.2\\
 
{\bf E} \ref{sec_dation_e_format}\\
{\bf ENTIER} 6.1.1\\
{\bf EQ} 6.1.2\\
{\bf EXOR} 6.1.2\\
{\bf EXP} 6.1.1\\
 
{\bf F} \ref{sec_dation_f_format}\\
{\bf FIT} 6.1.2\\
 
{\bf GE} 6.1.2\\
{\bf GT} 6.1.2\\
 
{\bf IDF} \ref{sec_dation_open_close}\\
{\bf IS} 5.9\\
{\bf ISNT} 5.9\\

 
{\bf LE} 6.1.2\\
{\bf LINE} \ref{sec_dation_read_write}\\
{\bf LIST} \ref{sec_dation_listformat}\\
{\bf LN} 6.1.1\\
{\bf LT} 6.1.2\\
{\bf LWB} 6.1.1, 6.1.2\\
 
{\bf NE} 6.1.2\\
{\bf NEW} \ref{sec_dation_open_close}\\
{\bf NOT} 6.1.1\\
{\bf NOW} A2.1\\
 
{\bf OLD} 10.3\\
{\bf OR} 6.1.2\\
 
{\bf PAGE} \ref{sec_dation_read_write}\\
{\bf POS} \ref{sec_dation_read_write}\\
{\bf PRM} \ref{sec_dation_open_close}\\
 
{\bf R} \ref{sec_dation_r_format}\\
{\bf REM} 6.1.2\\
{\bf ROUND} 6.1.1\\
{\bf RST} \ref{sec_dation_open_close}, \ref{sec_dation_rst} \\
 
{\bf S} \ref{sec_dation_a_format}\\
{\bf SHIFT} 6.1\\
{\bf SIGN} 6.1.1\\
{\bf SIN} 6.1.1\\
{\bf SIZEOF} 6.1\\
{\bf SKIP} \ref{sec_dation_read_write} \\
{\bf SOP}  \ref{sec_dation_read_write}\\
{\bf SQRT} 6.1.1\\
 
{\bf T} \ref{sec_dation_t_format}\\
{\bf TAN} 6.1.1\\
{\bf TANH} 6.1.1\\
{\bf TOBIT} 6.1.1\\
{\bf TOCHAR} 6.1.1\\
{\bf TOFIXED} 6.1.1\\
{\bf TOFLOAT} 6.1.1\\
{\bf TRY} 9.3.1\\
 
{\bf UPB} 6.1.1, 6.1.2\\

{\bf X} \ref{sec_dation_read_write}
}

%\twocolumn[\section{Index}    % A 5
%
%???? The denotation behind the word symbol refers to the paragraph where it
%is introduced.\\]
%


\printindex
