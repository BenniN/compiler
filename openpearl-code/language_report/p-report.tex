\documentclass{book}
\usepackage{geometry}  % paper size
\geometry{a4paper}     % -"-
\usepackage[color]{changebar} % use side change bars
%\usepackage{color}  % use color marks
% note: tcolorbox is required in the version from 2014
%       the usual tex packages from the linux distributions
%       are known to be too old (r. mueller: 28-dec-2014)
\usepackage[skins,breakable]{tcolorbox}  
\tcbuselibrary{breakable}

\usepackage[pdftex]{hyperref}

%\textwidth16cm
%\textheight24cm
%\oddsidemargin1cm
%\evensidemargin1cm
% \topmargin1cm
% DIN-A4: 29,6 cm Hoehe: 29,6-24=5,6 = 2,8 oeb + 2,8 unten
% => topmargin = 2,54 + 0,26
%\topmargin 0,26cm 
\parindent0pt
\parskip\baselineskip

\newcommand{\x}{\hspace*{1cm}}

\def\tekst #1,#2;{\parbox{#1}{\tiny\
                  \begin{center}#2\end{center}}}
\begingroup\makeatletter\ifx\SetFigFont\undefined
% extract first six characters in \fmtname
\def\x#1#2#3#4#5#6#7\relax{\def\x{#1#2#3#4#5#6}}%
\expandafter\x\fmtname xxxxxx\relax \def\y{splain}%
\ifx\x\y   % LaTeX or SliTeX?
\gdef\SetFigFont#1#2#3{%
  \ifnum #1<17\tiny\else \ifnum #1<20\small\else
  \ifnum #1<24\normalsize\else \ifnum #1<29\large\else
  \ifnum #1<34\Large\else \ifnum #1<41\LARGE\else
     \huge\fi\fi\fi\fi\fi\fi
  \csname #3\endcsname}%
\else
\gdef\SetFigFont#1#2#3{\begingroup
  \count@#1\relax \ifnum 25<\count@\count@25\fi
  \def\x{\endgroup\@setsize\SetFigFont{#2pt}}%
  \expandafter\x
    \csname \romannumeral\the\count@ pt\expandafter\endcsname
    \csname @\romannumeral\the\count@ pt\endcsname
  \csname #3\endcsname}%
\fi
\fi\endgroup

%%%%%%%%%%%%%%%%%%%%%%%%%%%%%%%%%%%%%%%%%%%%%%%%%
%\usepackage[normalem]{ulem}
\usepackage{soul}
\def\OpenPEARL{{\it OpenPEARL}}
\colorlet{color-added}{yellow!55!} %{RGB}{0,127,127}
\definecolor{color-removed}{RGB}{128,128,128}
\colorlet{color-tobedone}{blue!5!}  %{RGB}{0,0, 15}
\definecolor{color-modified}{RGB}{128,128,0}

\begin{document}

%%%%%%%%%%%%%%%%%%%%%%%%%%%%%%%%%%%%%%%%%%%%%%%%%
\newenvironment{tobedone}%
{\begin{tcolorbox}[breakable,colback=blue!5!white,colframe=blue!75!black,title=TO BE DONE]
\parskip\baselineskip
}
{\end{tcolorbox}}

\newenvironment{removed}%
{\begin{tcolorbox}[colback=black!50!white,colframe=black!75!black,title=REMOVED]
\parskip\baselineskip
}
{\end{tcolorbox}}

\newenvironment{added}%
{\begin{tcolorbox}[colback=yellow!55!white,colframe=yellow!75!black,title=ADDED]
\parskip\baselineskip
}
{\end{tcolorbox}}

\newenvironment{modified}%
{\begin{tcolorbox}[colback=orange!15!white,colframe=orange!75!black,title=MODIFIED]
\parskip\baselineskip
}
{\end{tcolorbox}}



\newcommand{\tobedoneblock}[1]{\cbcolor{color-tobedone}
\cbstart 
#1
\cbend
}
\newcommand{\removedblock}[1]{
\cbstart 
\cbcolor{color-removed}
\color{color-removed}
#1
\cbend
\cbcolor{black}
\color{black}
}
\newcommand{\addedblock}[1]{
\cbstart 
\cbcolor{color-added}
#1
\cbend
\cbcolor{black}
\color{black}
}

\newcommand{\tobedonetext}[1]{
\cbstart
\cbcolor{color-tobedone}
\color{black}
\sethlcolor{color-tobedone}
\hl{\mbox{#1}}
\cbend
\cbcolor{black}
\color{black}
}

\newcommand{\removedtext}[1]{
\cbstart
\cbcolor{color-removed}
\color{black}
\sethlcolor{color-removed}
\hl{\mbox{#1}}
\cbend
\cbcolor{black}
\color{black}
}

\newcommand{\addedtext}[1]{
\cbstart
\cbcolor{color-added}
%\color{color-added}
\color{black}
\sethlcolor{color-added}
\hl{\mbox{#1}}
\cbend
\cbcolor{black}
\color{black}
}


%%%%%%%%%%%%%%%%%%%%%%%%%%%%%%%%%%%%%%%%%%%%%%%%%
\begin{titlepage}
\sethlcolor{yellow}

\begin{center}
{\Huge \OpenPEARL} \\
\vspace{1cm}
Implementation Description \\
\vspace{5cm}
%2014-Oct-23\\
\today{}\\
\vspace{10cm}
\removedtext{GI FG 4.4.2}
\begin{added}
Arbeitskreis ''Compiler''\\
GI Fachgruppe ''Echtzeitsysteme''
\end{added}

\end{center}
\end{titlepage}

%%%%%%%%%%%%%%%%%%%%%%%%%%%%%%%%%%%%%%%%%%%%%%%%%

\tableofcontents

%%%%%%%%%%%%%%%%%%%%%%%%%%%%%%%%%%%%%%%%%%%%%%%%%
\input 01-introduction   	% 1
\input 02-fundamentals.tex 	% 2
\input 03-language_forms.tex	% 3
\input 04-program_structure.tex	% 4
\input 05-variables_and_constants.tex 	% 5
\input 06-expressions.tex		% 6
\input 07-sequential.tex		% 7
\input 08-procedures.tex		% 8
\input 09-parallel_activities.tex	% 9
\input 10-input_and_output.tex		% 10
\input 11-signals.tex			% 11
\input 90-appendix.tex			% A1

\end{document}


