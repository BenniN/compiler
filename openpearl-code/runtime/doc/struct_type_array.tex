\chapter{Struct, Type and Arrays}

\section{Struct}

PEARL structs may be mapped to C++ structs.
The C++ operators like \verb|=| (assignment), \verb|.| (dot) and 
\verb|->| (follow) will work.

One problem arises when structs are multiply defined
in PEARL. All struct with the same internal structures as
taken to be identical. C and C++ structs as not compatible
if multiply defined.

The struct statement should be mapped to a named C++ struct and the identifier
of the struct is derived from the struct components.
The C++ standard states that a C++ compiler should support at least 1024
characters for identifiers. The g++ has no limit for the length of identifiers.

Each possible data type for a struct component is mapped to a letter.
The length has a maximum length of 64 for FIXED, FLOAT and BIT. CHAR
variables may have a length of up to 32767 characters.
 C++ provides only 63 different
 characters for the use in identifiers. So we use 1-5 digits
for the length.

If the component is an array, the number of dimensions and the
 limits are
passed as \verb|_| (underscore) separated list of decimal integers
to the type.

Mapping of data types to letters:

\begin{tabular}{|l|c|c|}
\hline
Datatype & letter & REF \\
\hline
FIXED & A &a\\
FLOAT & B &b\\
BIT  & C &c\\
CHARACTER & D &d \\
CLOCK  & E &e \\
DURATION & F&f \\
TASK & & g \\
PROC  & & h \\
SEMA & I & i \\
BOLT & J & j \\
\hline
STRUCT & S &s\\
\hline
\end{tabular}

\paragraph{Note:} Semaphores and Bols variables are not allowed in structs,
but the encoding mapping is also used for the inter module checker.


Structs are mapped to the key 'S' followed by an integer
indicating the length of the struct components.
E.g. 

Example:
\begin{PEARLCode}
... STRUCT [ 
    x FIXED(15);
    y FLOAT(53);
    ]
...
\end{PEARLCode}

The component x ist translated into A15 and y into B53.
The struct consists of these two elements with a length of
the type string of 3 digits each. Thus the content
of the struct is described by the string "'A15B53"'.
This is prefixed with "'S"' and the length of the component
descriptor leading to S6A15B53.

\begin{CppCode}
struct S6A15B53 {
   pearlrt::Fixed<15> _x;
   pearlrt::Float<53> _y;
};
\end{CppCode}

In case of combination of arrays in structs and array of structs we get:

\begin{PEARLCode}
DCL x STRUCT [
   a FIXED(7);
   b(1:3,3:10) FLOAT(24);
   c(2) STRUCT [
     c1(19) FIXED(15);
     c2 BIT(5);
     ];
  ];
\end{PEARLCode}

The name of the struct results into: 

$S37\underbrace{\underbrace{A7}_{FIXED(7)}}_{a}\underbrace{\underbrace{B24}_{Float(24)}\underbrace{\_2\_1\_3\_3\_10}_{(1:3,3:10)}}_{b(1:3,3:10)}\underbrace{\underbrace{S12\underbrace{A15\_1\_1\_19}_{c1(19) FIXED(15)}\underbrace{C5}_{c2 BIT(5)}}_{STRUCT ..}\underbrace{\_1\_1\_2}_{(2)}}_{c(2)}$


\begin{CppCode}
struct S37A7B24_2_1_3_3_10S12A15_1_1_19C5_1_1_2 {
   pearlrt::Fixed<7> _a;
   pearlrt::Float<24> data_a[32];
   struct S13A15_1_1_19C5 {
     pearlrt::Fixed<15> data_c1[19];
     pearlrt::BitString<5>  _c2;
   } data_c[2];
};

// for array descriptors see corresponding section
DCLARRAY(ad_2_1_3_3_10,2,LIMITS{{1,3,6},{3,10,1}};
DCLARRAY(ad_1_1_19,1,LIMITS{{1,19,1}}};
DCLARRAY(ad_1_1_2,1,LIMITS{{1,2,1}}};
\end{CppCode}



\section{TYPE}
The type statement must we resolved by the compiler. 
When used in STRUCTs, it is recommended that they are replaced by the
native elements.
%should be mapped on \verb|typedef| with a suitable prefix 
%(here \verb|type_|). For details about struct nameing see next section.

Example:
\begin{PEARLCode}
TYPE complex STRUCT [
     real FLOAT(53);
     imaginary FLOAT(53);
];
\end{PEARLCode}

\begin{CppCode}
struct _SB53B53 {
   pearlrt::Float<53> _real;
   pearlrt::Float<53> _imaginary;
};

\end{CppCode}
%

\section{Array}

PEARL supports multidemensional arrays with individual index
boundaries. Different arrays may be passed to procedures if the
number of dimensions are identical.
Arrays may be part of data structures.

\subsection{Array Data}
The data storage for the array elements may be defined as a plain
linear data array. The access of array elements is done via the 
{\em array descriptor}. It is recommended to name the array
descriptor like the according the user element and make
a decoration for the data storage. 
The calculation of the required number of data elements must be
done in the PEARL$\rightarrow$C++ compiler.

\subsection{Array Descriptor}
The operations of arrays are done with the {\em array descriptor}
and the pointer to the first element of the data storage.
The descriptor is of the type \verb|Array *|.

\begin{classSummary}
 Class & \verb|Array| \\
 Specification & Array.h \\
 Namespace & pearlrt \\
 Implementation & Array.cc \\
 \verb|offset(...)|      & returns the linear index 
       			to the specified array element \\
 \verb|upb(Fixed<31> idx)| & return upper bound of the
			 specified index; index starts counting at 1\\
 \verb|lwb(Fixed<31> idx)| & return lower bound of the
			 specified index; index starts counting at 1 \\
\end{classSummary}

The definition of the array descriptor is done via a C-macro.
The methods \verb|offset|, \verb|upb| and \verb|lwb| 
will throw \verb|ArrayIndexOutOfBoundsSignal|
if the index is out of the bounds.

\paragraph{Note:} The DCLARRAY macro may be compressed into a single statement.
The separate statements are for clearance.
The elements should become static to avoid conflicts, if more than 
one module is used. A simple way of inter module reuse is not known up to now.

\begin{verbatim}
#define DCLARRAY(name,dimensions,limits) \
   pearlrt::ArrayDescriptor<dimensions> a_##name = { dimensions, limits }; \
   pearlrt::Array b_##name((pearlrt::ArrayDescriptor<0>*)&a_##name); \
   pearlrt::Array * name = &(b_##name);
\end{verbatim}

The parameters are:
\begin{description}
\item[name] a compiler generated identifier, which will become
    the name  of the array descriptor. The descriptors should be 
    named according the array structure. One descriptor for all
    arrays with the same structure is sufficient, independent of
    the contained data type.
\item [dimensions] is a C integer constant with the number of 
   array dimensions. This must be larger than 0.
\item[limits] is a initializer list for the limits data structure.
   The elements of the limits structure are also C integers
   \begin{enumerate}
   \item start index as integer
   \item end index as integer
   \item number of elements in next sub array. This value is 1 for the 
     last index. It is the product of all index ranges of all subarrays.
     E.g. a(3,4,5) $\rightarrow$ LIMITS(3,\{1,3,4*5*1\},\{1,4,5*1\},\{1,5,1\}\}
   \end{enumerate}
\end{description}

The access to the array elements is done via the method \verb|offset|, 
with gets a variadic parameter list with all array indices.

Used prefix decoration scheme:

\begin{tabular}{|l|l|}
\hline
data & data storage as linear array \\
\hline
a & array descriptor object \\
\hline
b & array object which stores the pointer of the array descriptor \\
\hline
\end{tabular}

\subsection{Example}

\begin{PEARLCode}
DCL x(10:20) FIXED(31); 
DCL y(1,10,10:20) FLOAT(24); 
TYPE MyStruct STRUCT [
           dummy FIXED(31);
           array1 (0:9,9:9) FIXED(31);
           dummy2 FIXED(31);
           ];
DCL s(0:19) MyStruct;
...
x(10) := x(15);
y(10,10) := 0.0;
testValue := s(2).array1(0,1);
\end{PEARLCode}


\begin{CppCode}
// DCL x(10:20) FIXED(31); 
static Fixed<31> data_x[11]; // 11 data elements
                             // identifier with decoration
DCLARRAY(ad_1_10_20, 1, LIMITS{{10,20,1}});
// - - - - - - 
// DCL y(1,10,10:20) FLOAT(24); 
 // 10*11 data elements
 // identifier with decoration
static pearlrt::Float<24> data_y[110];
DCLARRAY(ad_2_1_10_10_20, 2, LIMITS{{1,10,11},{10,20,1}});

// - - - - - - 
   struct S17A31A31_1_1_100A31 {
      pearlrt::Fixed<31> _dummy1;
      pearlrt::Fixed<31> _array1[100];
         // arrayData Fixed(0:9,0:9);
      pearlrt::Fixed<31> _dummy2;
   } ;
   S17A31A31A31_1_1_100A31 _s[20]; 

                        // indices 0..19;
                        //multiply with 1
                        // since no subarray present
   DCLARRAY(ad_1_0_19, 1, LIMITS({{0,19,1}}));
   DCLARRAY(ad_2_0_9_0_9, 2, LIMITS({{0,9,10},{0,9,1}}));

...
   *(_x+ad_1_10_20->offset(pearlrt::Fixed<31>(10))) =
             *(x_data+ad_1_10_20->offset(pearlrt::Fixed<31>(15)));
   *(_y+ad_2_1_10_10_20->offset(pearlrt::Fixed<31>(10),
            pearlrt::Fixed<31>(10)) =
                          pearlrt::Float<24>(0.0);
...
   pearlrt::Fixed<31> testValue;

   testValue = 
        *((*(_s+ad_1_1_2->offset(pearlrt::Fixed<31>(2))) )._array1)+
             ad_2_0_9_0_9->offset(pearlrt::Fixed<31>(0),pearlrt::Fixed<31>(1)
        );

\end{CppCode}

\subsection{Usage as Paramater}
The array descriptor is a pointer, which must be passed as additional parameter
to procedures. 
Passing arrays of structs containing some array may lead to the requirement
to pass more than one array descriptor in addition to the pointer to the 
reals data object.
By convention the array descriptors should be passed after the pointer to the
data storage in the order of declaration of the array descriptors (DCLARRAY).

\begin{PEARLCode}
DCL x(10:20) FIXED(31); 
TYPE MyStruct STRUCT [
           dummy FIXED(31);
           array1 (0:9,9:9) FIXED(31);
           dummy2 FIXED(31);
           ];
DCL s(0:19) MyStruct;
...
CALL anyFunc(x);
CALL otherAnyFunc(s);
...
anyFunc(a() FIXED(31) IDENT) 
   a(LWB a) := 0;  // clear first element
END;
\end{PEARLCode}

\begin{CppCode}
// DCL x(10:20) FIXED(31); 
 // 11 data elements
 // identifier with decoration
static pearlrt::Fixed<31> _x[11];
DCLARRAY(ad_1_10_20, 1, LIMITS{{10,20,1}});

// - - - - - - 
   struct S19A31A31_2_0_9_0_9A31 {
      pearlrt::Fixed<31> _dummy1;
                       // arrayData Fixed(0:9,0:9);
      pearlrt::Fixed<31> _array1[100]; 
      pearlrt::Fixed<31> _dummy2;
   };

   struct S19A31A31_2_0_9_0_9A31  _s[20]; 
   DCLARRAY(ad_1_1_20, 1, LIMITS({{0,19,1}}));
   DCLARRAY(ad_2_0_9_0_9, 2, LIMITS({{0,9,10},{0,9,1}}));

...
   anyFunc(_x, ad_1_10_20);
   otherAnyFunc(_s, as_2_0_9_0_9);
...
void anyFunc(pearlrt::Fixed<31> * _a,
             pearlrt::Array * ad_a) {
  *(_a+ad_a->offet(ad_a->lwb(1))) =
          (pearlrt::Fixed<31>)(0)
}
\end{CppCode}

If a complete array should be written to a DATION with e.g. WRITE,
the address of the first array element must be used.


