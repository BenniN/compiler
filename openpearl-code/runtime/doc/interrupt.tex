\chapter{Interrupt}
An interrupt is an asynchrous event from the outside world.
It is {\bf NOT} the hardware interrupt of the processor.

\section{Specification and Declaration}
An interrupt is declared in the system part.

Example: Declare the identifier \verb|ctrlc| as the plattform specific
interrupt source \verb|^C|

\begin{verbatim}
ctrlc: UnixSignal(2);
\end{verbatim}

The specification takes place in the problem part:

\begin{verbatim}
SPC ctrlc INTERRUPT;
\end{verbatim}

A specific plattform may support several different interrupt sources.
All concrete interrupt sources must be derived from the parent class
\verb|pearlrt::Interrupt|.


\section{C++ Mapping between System and Problem Part}
The user defined identifier in the system part denotes a specifice interrupt
source. The identifier in the problem part denotes a generalized interrupt.
The compiler should generate a pointer to the generalized interrupt object like
shown in the example below:

\begin{PEARLCode}
SYSTEM;
  ctrlc: UnixSignal(2);
PROBLEM;
   SPC ctrlc INTERRUPT;
\end{PEARLCode}

The user supplied identifier (\verb|ctrlc|) is used as base of derived
identifiers.
\begin{description}
\item[sys\_] prefix denotes the real defined interrupt object 
\item[\_] prefix (as usual for all user supplied idetifiers) denotes the
    pointer to the generalized interrupt object.
\end{description}

\begin{CppCode}
// SYSTEM;
static pearlrt::UnixSignal sys_ctrlc(2);
       pearlrt::Interrupt * _ctrlc = (pearlrt::Interrupt*)&sys_ctrlc;
// PROBLEM;
extern pearlrt::Interrupt * _ctrlc;
\end{CppCode}

\section{Method Mapping}
The interrupt class provides methods for the PEARL language statements
working directly on interrupts.

The translation from PEARL to C++ is generic. 
The interrupt identifier is a pointer to the generalized interrupt object.

\begin{methodMapping}
\verb|ENABLE|  & \verb|enable()| & \\
\verb|DISABLE|  & \verb|disable()|&  \\
\verb|TRIGGER|  & \verb|trigger()| & \\
\end{methodMapping}

Example:

\begin{PEARLCode}
! ctrlc is specified as INTERRUPT;
...
ENABLE ctrlc;
DISABLE ctrlc;
TRIGGER ctrlc;
\end{PEARLCode}

Should translate into:
\begin{CppCode}
_ctrlc->enable();
_ctrlc->disable();
_ctrlc->trigger();
\end{CppCode}

\section{Usage in Task Scheduling}
See chapter \ref{task}

\section{Software Interrupts}
For all plattforms the \verb|SoftInt| interrupt source is provided.
There is no external connection to this interrupt type.

Synopsis: \verb|SoftInt(nbr)|

The parameter \verb|nbr| denotes the software interrupt number. 
The values from 1 to 32 are allowed. It is forbidden to define more
than one user identifier for the same number.
In case of attempting more than one user defined interrupt for 
a UNIX signal the InternalDataTypeSignal is induced 
during system initializing phase.

\section{How to Create an Interrupt Source}
Interrupt sources are always plattform specific.
In order to create such an interrupt source follow the instructions below:
\begin{enumerate}
\item choose a good name for the interrupt source
\item define the parameters if required
\item create a class with the same name as the your interrupt 
  \begin{enumerate}
  \item  derive your class from Interrupt  
  \item supply a constructor which fits to the required parameters
  \item make shure that there is may only one object become created
        by the user
  \item do all required operations, which are required to gather 
        the interrupt afterwards (register in the system, create 
        worker thread, ...)
  \end{enumerate}
\item make shure to call the method \verb|trigger()| from the parent
      class each time the interrupt in detected.
\item add the documentation of your interrupt in the plattform specific part
      of the OpenPEARL documentation
\end{enumerate}

