\chapter{Procedures}

\section{Declaration ans Specification}
PEARL procedures are mapped to C++ functions.
They have a hidden first parameter with the pointer to the executing task.
The namespace rules apply as on data values.


\section{REF PROC}
References to procedures are mapped to the template class Ref, which allows
the assignment and derefence operations.

\section{Formal Parameter Types}
Bit and Char-slices need special attention.

\subsection{Call By Value}
In this case an intermediate copy of the actual parameter 
must become created. This may be done by the C++ compiler via the 
normal call by value mechanisme.

\subsection{INV Call By Value}
This must be treated by the compiler. No assignment is allowed to the formal 
parameter, nether passing by reference without INV to other
procedures is allowed.

\subsection{Call By Reference (IDENT)}
All data types except \textbf{BIT} and \textbf{CHAR} types may be 
passed as C++ pointers.

For \textbf{BIT} and \textbf{CHAR} types the actual parameter may be a slice.
Any changes on the formal parameter must affect the current parameter.
To enshure the operation on the same primary data, \textbf{CHAR} and 
\textbf{BIT} values 
must be passed as slice by value. 
Even overlaping slices in the parameter list may be possible.

Slices \textbf{MUST NOT} passed as pointers, since this may cause
side effects when passing the same slice as different parameters.

Example:
\begin{PEARLCode}
DCL bits BIT(16) INIT('ABCD'B4);
DCL hello CHAR(10) INIT('Hello');
...
p: PROC(b1 BIT(4) IDENT, b2 BIT(16) IDENT, c CHAR(2) IDENT);
   b1 := '5'B4;      !! this affects bits.BIT(2:5) !!
   IF b2.BIT(3) THEN !! this must check the modified value !!
      ...
   FIN;

   c := 'x';         !! set the first two characters of the
                     !! actual parameter to 'x '
END;

t: TASK;
   CALL p(bits.BIT(2:5), bits, hello.CHAR(1:2));
END;
\end{PEARLCode}

\begin{CppCode}
...
pearlrt::BitString<16> _bits(0xabcd);
pearlrt::Character<10> _hello("Hello");

void _p(pearlrt::Task * me,
        pearlrt::BitSlice _b1,
        pearlrt::BitSlice _b2,
        pearlrt::CharSlice _c) {

    // modify the bits in the actual parameter (IDENT)
    _b1.setSlice(CONST_BIT4_five);

    // select the bit inside the slice and obtain the bool value
    if ((_b1.
            getSlice(CONST_FIXED_3)->
            mkBitString(pearlrt::BitString<1>*typeInfo)
        ).getBoolean() ) {
       ...
    }
    ...
   _c.getSlice(CONST_FIXED_1,CONST_FIXED_2)->setSlice(CONST_CHAR_1_x);
}

DCLTASK(_t, ...) {
   ...
   _p(me,*pearlrt::BitSlice(_bits).getSlice(2,5),
         pearlrt::BitSlice(_bits),
         pearlrt::CharSlice(_c).getSlice(1)); 
\end{CppCode}

\paragraph{Notes:} 
\begin{enumerate}
\item Bit and char slices are passed by value. The slices point 
internally to the base data value.
\item passing a slice to another procedure by reference (\textbf{IDENT}) 
is possible as complete or subslice.
\item Constants and \textbf{INV} variables are not allowed as actual parameters
\end{enumerate}

\subsection{INV Call By Reference (IDENT)}
The compiler must check that there is no change of the actual parameter
by assignment or parameter passing.

The invocation with constants is possible. They must become converted 
to a bit- or char slice by the compiler.

\subsection{REF Types}
\textbf{REF} types may be passed by value as well as by reference 
( \textbf{IDENT}).
The mapping to C++ pointers applies.
This applies also for type \textbf{REF CHAR(}x\textbf{)}
 (e.g. \textbf{REF CHAR(4)});.

\subsubsection{REF CHAR()}
A type \textbf{REF CHAR()} allows a kind of variable string length.
This type realized in the RefChar-class. 

Depending on the actual parameter, diffent actions must be performed 
by the compiler. If the actual parameter is of type:
\begin{description}
\item[CHAR(x):] A temporary RefChar data must be created and passed.
\item[REF CHAR():] The RefChar data object may by passed directly 
   by value or as pointer depending on the presence of IDENT
\end{description}

\subsubsection{REF INV CHAR()}
Even constants may be passed. In this case, the compiler must create
a temporary RefChar object for any CHAR-values and expressions.

\subsection{Arrays}
Arrays must be passed in two parameters. The array descriptor and the 
array data.


 

\subsection{STRUCT}
The passing of STRUCTs works like simple variables.

\section{Return Value}
The language report forbids \textbf{REF CHAR()} as result type.

\section{Diskussion}
\begin{enumerate}
\item Der letzte Abschnitt in 5.9 ist mir nicht klar.
RETURNS(REF CHAR()) ist wohl verboten.
Aber als IDENT Parameter zul\"assig?

\begin{PEARLCode}
DCL (s,t) CHAR(10);
...
P: PROC(x CHAR(10), y REF CHAR() IDENT, z CHAR(10) IDENT);
   x := y; ! ok, evtl CharTooLongSignal
   y := x; ! sollte nicht gehen
   y := z; ! sollte nicht gehen, da die Lebensdauer nicht kontrollierbar ist
   y := s; ! sollte gehen, da Lebensdauer unkritisch ist
END;
\end{PEARLCode}

\item Bei der Einf\"uhrung der Slices in Kap 6.1 steht, dass eine .CHAR(a:b)
 bei evtl. 
erst zur Laufzeit bestimmt wird. Ein solcher Ausdruck darf nur 
bei CAT nicht benutzt werden --- also als Parameter zul\"assig?
\item Expressions sind eigentlich nicht per IDENT \"ubergebbar. Allerdings sind
Bit und Char Slices --- im Sprachreport Segment genannt --- auch auf der 
linken Seite einer Zuweisung erlaubt (lvalue).
\item Sollen auch Array-Slices als Prozedurparameter m\"oglich sein?
Im Sprachreport habe ich nichts gefunden. M\"oglich w"are es:\\ 
If the actual parameter is a slice, an array descriptor must be created
 according the  current size of the slice. As data array  the adress
of the first element of the slice must be given.

\end{enumerate}
