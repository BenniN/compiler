\chapter{Structs and Types}

PEARL structs may be mapped to C++ structs.
The C++ operators like \verb|=| (assignment), \verb|.| (dot) and 
\verb|->| (follow) will work.

One problem may arise when structs are multiply defined
in PEARL. All struct with the same internal structures as
taken to be identical. C and C++ structs as not compatible
if multiply defined.

\section{TYPE}
The type statement should be mapped on \verb|typedef| with a suitable prefix 
(here \verb|type_|). For details about struct nameing see next section.

Example:
\begin{PEARLCode}
TYPE complex STRUCT [
     real FLOAT(53);
     imaginary FLOAT(53);
];
\end{PEARLCode}

\begin{CppCode}
struct _SB53B53 {
   pearlrt::Float<53> _real;
   pearlrt::Float<53> _imaginary;
};

typedef struct _SB53B53 type_complex; 
\end{CppCode}

\section{STRUCT}
The struct statement should be mapped to a named C++ struct and the identifier
of the struct is derived from the struct components.
The C++ standard states that a C++ compiler should support at least 1024
characters for identifiers. The g++ has no limit for the length of identifiers.

Each possible data type for a struct component is mapped to a letter.
The length has a maximum length of 64. C++ provides only 63 different caharacters for the use in identifiers, so we use 1-2 digits for the length.
If the component is an array, the number of dimensions and the limits are
passed as \verb|_| (underscore) separated list of decimal integers.

Mapping of data types to letters:

\begin{tabular}{|l|c|c|}
\hline
Datatype & letter & REF \\
\hline
FIXED & A &a\\
FLOAT & B &b\\
BIT  & C &c\\
CHARACTER & D &d \\
CLOCK  & E &e \\
DURATION & F&f \\
TASK & & g \\
PROC  & & h \\
\hline
own type & Z &z\\
\hline
\end{tabular}

Own data types are mapped to a pattern starting with Z and the identifier
 length followed with the identifier.

Example:
\begin{PEARLCode}
TYPE complex STRUCT [
     real FLOAT(53);
     imaginary FLOAT(53);
];

... STRUCT [ 
    x FIXED(15);
    y FLOAT(53);
    b(1:4) BIT(64);
    c(10,11) complex;
    ] ...

\end{PEARLCode}

\begin{CppCode}
struct _SB53B53 {
   pearlrt::Float<53> _real;
   pearlrt::Float<53> _imaginary;
};

typedef struct _SB53B53 type_complex; 

struct _SA15B53C641_1_4Z12type_complex2_1_10_1_11
{
   pearlrt::Fixed<15> _x;
   pearlrt::Float<53> _y;
   pearlrt::BitString<64> _b[4];
   type_complex _complex[110];
};


\end{CppCode}

