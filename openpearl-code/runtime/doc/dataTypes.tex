\chapter{Simple Data Types}
PEARL defines the data types FIXED, FLOAT, BIT, CHARACTER, REF CHARACTER,
DURATION and CLOCK.

Specific for PEARL is the signal mechanism. It is expected that
each operation induces a signal (in C++ exception) if the operator
does no finish successfully. C++ does not provide this feature.

For each data type a separate class provides the required operations.
FIXED, FLOAT, BIT and CHARACTER may be defined in PEARL with a length
attribute. This is mapped to C++-templates.

For details about ctor parameters and member functions please check the
doxygen documentation, which can be created for a specific plattform
by \verb|make doc| in the corresponding directory.


\section{Fixed}
The class {\em Fixed63} provides save operations on fixed values 
with 63 bits plus sign. This development of this class was influenced by
the header SafeInt.hxx, which is provided with the MSPL license.
This class is used internally for the implementation of fixed values
with individual lengths.

The application specific versions of  the type fixed are realized with the 
C++ template mechanism in {\em Fixed.h},
 requiring the number of mantissa bits as template parameter. 
All lengths from 1 to 63 are suported.
The operations are mapped to the usual C++-operators.  
For PEARL-specific operations member functions are supplied.

All operations, which produce numerical {\em overflow}, {\em underflow}
 or {\em divide by zero}  throw an exception.

\begin{classSummary}
 Class & \verb|Fixed<int S>| \\
 Specification & Fixed.h \\
 Implementation & - \\
 \verb|size(Fixed<7>)| &   1 byte \\
 \verb|size(Fixed<15>)| & 2 bytes \\
 \verb|size(Fixed<31>)| & 4 bytes \\
 \verb|size(Fixed<63>)| & 8 byte  \\
\end{classSummary}


\begin{methodMapping}
  \verb|+| (monadic)      &        & to be treated by the compiler \\
  \verb|-| (monadic)    & \verb|operator-()| &  \\
 \verb|ABS|            & \verb|absVal()|  &  \\
 \verb|SIGN|           & \verb|sign()|  & \\
 \verb|TOBIT|          & \verb|BitString| &
                    realized as Ctor in BitString 
                     with the component '.x' of the FIXED variable \\
 \verb|TOCHAR|         & \verb|toChar()|  & realized in Character.h \\
  \verb|+| (dyadic)     & \verb|operator+(rhs)| & \\
  \verb|-| (dyadic)     & \verb|operator-(rhs)| & \\
  \verb|*| (dyadic)     & \verb|operator*(rhs)| & \\
  \verb|//| (dyadic)   & \verb|operator/(rhs)| & \\
  \verb|REM| (dyadic)   & \verb|operator%(rhs)| & \\
  \verb|**| (dyadic)    & \verb|pow(rhs)|    & rhs is the exponent \\
  \verb|<|          & \verb|operator<(rhs)|  & \\
  \verb|>|           & \verb|operator>(rhs)|  & \\
  \verb|<=|         & \verb|operator<=(rhs)| & \\      
  \verb|>=|         & \verb|operator>=(rhs)| & \\   
  \verb|==|         & \verb|operator==(rhs)| &   \\
  \verb|/=|        & \verb|operator!=(rhs)|  &   \\
  FIT               & \verb|fit(rhs)|    & rhs defines target size \\
\end{methodMapping}

\paragraph{Remarks:}
\begin{itemize}
\item If no preset value is specified, the variable is initialized with 0.
\item There is no default length in the runtime system.
\item C++ literals correspond usually to 32 bit integers. 
If we need longer constants the postfix \verb|LL| must be added by
the compiler to enforce 64-bit integers.
\end{itemize}

\begin{PEARLCode}
DCL x FIXED(31) INIT(15); 
DCL y FIXED(33) INIT(16); 
DCL z FIXED INIT(17); 
DCL z1 FIXED;
DCL longFixed Fixed(53) INIT(123456789012); 
z := x + y;
x := z FIT x;
x := z ** y;
\end{PEARLCode}


\begin{CppCode}
Fixed<31> x(15);
Fixed<33> y(16);
Fixed<31> z(17);
Fixed<31> z1;
Fixed<53> longFixed(123456789012LL); 
z = x + y;
x = z.fit(x);
x = z.pow(y);
\end{CppCode}


\section{Float}

The type FLOAT is provided by templates with the lengths 24 and 53 bit
mantissa precission, both with the hidden one presentation.
These lengths are defined by the IEEE754 
standard types single and double.

New float variables are initialized with {\em NaN}, which is defined
as \verb|NAN| in the {\em math.h} GNU-library if IEEE754 is
available on the current system. This preset avoids problems with 
uninitialized float values in calculations.

Each operation (except assignment and comparison) contain checks
for unexpected results. A result of {\em NaN} or {\em INF} causes the signals
{\em FloatIsNaNSignal} or {\em FloatIsINFSignal} to be emitted.

The operations are mapped to the usual C++-operators.  
For PEARL-specific operations member functions are supplied.

\begin{classSummary}
 Class & \verb|Float<24>| and \verb|Float<53>|\\
 Specification & Float.h \\
 Implementation & - \\
 \verb|size(Float<24>)| &   4 byte \\
 \verb|size(Float<53>)| & 8 bytes \\
\end{classSummary}


\begin{methodMapping}
  \verb|+| (monadic)      &        & to be treated by the compiler \\
  \verb|-| (monadic)    & \verb|operator-()| &  \\
 \verb|ABS|            & \verb|abs()|  &   \\
 \verb|SIGN|           & \verb|sign()|   &  \\
 \verb|ENTIER|          & \verb|entier()| & \\
 \verb|ROUND|          & \verb|round()| & \\
  \verb|+| (dyadic)     & \verb|operator+(rhs)| & rhs may be Fixed or Float\\
  \verb|-| (dyadic)     & \verb|operator-(rhs)| & rhs may be Fixed or Float\\
  \verb|*| (dyadic)     & \verb|operator*(rhs)| & rhs may be Fixed or Float\\
  \verb|/| (dyadic)   & \verb|operator/(rhs)| & rhs may be Fixed or Float\\
  \verb|**| (dyadic)    & \verb|pow(rhs)|    &
                rhs is the exponent with type \verb|Fixed<S>| \\
  \verb|<|          & \verb|operator<(rhs)|  & rhs may be Fixed or Float\\
  \verb|>|           & \verb|operator>(rhs)|  & rhs may be Fixed or Float \\
  \verb|<=|         & \verb|operator<=(rhs)| & rhs may be Fixed or Float \\     
  \verb|>=|         & \verb|operator>=(rhs)| & rhs may be Fixed or Float \\   
  \verb|==|         & \verb|operator==(rhs)| & rhs may be Fixed or Float \\
  \verb|/=|        & \verb|operator!=(rhs)|  & rhs may be Fixed or Float   \\
  FIT               & \verb|fit(rhs)|    & rhs defines target size \\
SIN			& \verb|sin()|	& rhs must be Float \\ 
COS			& \verb|cos()|	& rhs must be Float \\ 
TAN			& \verb|tan()|	& rhs must be Float \\ 
ATAN			& \verb|atan()|	& rhs must be Float \\ 
TANH			& \verb|tanh()|	& rhs must be Float \\ 
EXP			& \verb|exp()|	& rhs must be Float \\ 
LN			& \verb|ln()|	& rhs must be Float \\ 
SQRT			& \verb|sqrt()|	& rhs must be Float \\ 
\end{methodMapping}

\paragraph{Remarks:}
\begin{itemize}
\item There is no default length in the runtime system.
\item C++ literals are passed as {\em double} to ctor.
\item mixed operations with Fixed and Float are treated b non member methods
\item sin, cos, ... with fixed is not supported yet; this must be 
   treated by the compiler with the semantic analysis by automatic
   type conversion towards Float.
\item entier and round behavior must be compared with existing implementations; 
    this implementation behaves like:
    
    \begin{tabular}{c||c|c|c|c|c|c}
     x &         10.5 & 10.4 & 10.0 & -10.0 & -10.4 & -10.5 \\
     \hline
     entier(x) & 10 & 10& 10& -10& -11 & -11 \\
     \hline
     round(x) & 11& 10& 10& -10& -10& -11 \\
   \end{tabular} 
\end{itemize}

\begin{PEARLCode}
DCL x FLOAT(24) INIT(1.5); 
DCL y FLOAT(53) INIT(1.6); 
DCL z FLOAT INIT(1.7); 
DCL z1 FIXED(31);
y := x + y;
x := z FIT x;
x := z ** y;
z := SIN(y);
z := SIN(x+y);
\end{PEARLCode}


\begin{CppCode}
Float<24> x(1.5);
Float<53> y(1.6);
Float<53> z(1.7);
Float<31> z1;
y = x + y;
x = z.fit(x);
x = z.pow(y);
z = y.sin();
z = (x+y).sin();
\end{CppCode}


\section{Bit}
The type BIT is supported by the templated class {\em BitString}.
The name was choosen to make clear that there may be  more than one bit
in variable of this type.
The internal data type is used according the length of the BitString. 
The smallest possible native data size is used (1,2,4 or 8 bytes).

The supported sizes are 1 to 64.

The alignment of the bits inside the BitString is left adjusted. 
This means that the most significant bits are used. This was choosen due to 
the definition of operations of BitStrings with differnt lengths.

Member functions are supplied for the PEARL operations. 

Bit constants define their length intrinsic. All given digits are 
considered to be part of the bit constant. E.g.

\begin{tabular}{|r|c|}
\hline
BIT constant & Type \\
\hline
'1'B & BIT(1) \\
'1'B2 & BIT(2) \\
'1'B3 & BIT(3) \\
'1'B4 & BIT(4) \\
'01'B4 & BIT(8) \\
\hline
\end{tabular}

\begin{classSummary}
 Class & \verb|BitString<int len>| \\
 Specification & BitString.h \\
 Implementation & - \\
 \verb|size(BitString<8>|) & 1 byte \\
 \verb|size(BitString<16>|) & 2 bytes \\
 \verb|size(BitString<32>)| & 4 bytes \\
 \verb|size(BitString<64>)| & 8 byte  \\
\end{classSummary}

\begin{methodMapping}
  \verb|NOT|       & \verb|bitNot()| &  \\
  \verb|TOFIXED|   & \verb|toFixed()| &  \\
  \verb|==|        & \verb|operator==(rhs)|   &  \\
  \verb|/=|        & \verb|operator!=(rhs)|   & \\
  \verb|AND|       & \verb|bitAnd(rhs)|   & \\
  \verb|OR|       & \verb|bitOr(rhs)|   & \\
  \verb|EXOR|     & \verb|bitXor(rhs)|    & \\
  \verb|><| (CAT) & \verb|bitCat(rhs)|    & \\
  \verb|<>| (CSHIFT) & \verb|bitCshift(rhs)| & \\
  \verb|SHIFT|       & \verb|bitShift(rhs)|  & \\
  \verb|.BIT(x)|     & \verb|getBit(...)|  & 
				{\em on right hand side of the expression}\\
  \verb|.BIT(x)|    & \verb|setBit(...)|  & 
				{\em on left hand side of the expression}\\
  \verb|.BIT(x,y)|  & \verb|getSlice(...)| & 
				{\em on right hand side of the expression}\\
  \verb|.BIT(x,y)|   & \verb|setSlice(...)| & 
				{\em on left hand side of the expression}\\
\end{methodMapping}

The slice operations \verb|.BIT(...)| may be used in PEARL 
on the left hand side and the right hand side of the expression.
 

\paragraph{Remarks:}
\begin{itemize}
\item The preset value is passed to the ctor as integer type. The
   radix evaluation must be done by the compiler.
\item The passed integer is shifted internally by the number of not used 
   bits in the used native data type to the left side. This results
   in the proper interal alignment.
\item If no preset value is specified, the variable is initialized with 0.
\item There is no default length in the runtime system.
\item the method names like {\em bitAnd} were choosen since {\em and} 
is a reserved keyword in C for computer systems without \& character
\item C++ literals correspond usually to 32 bit integers. 
If we need longer constants the postfix \verb|LL| must be added by
the compiler to enforce 64-bit integers.
\item The initialising constant is of type int and right adjusted!
\item {\em getSlice} and {\em setSlice} take the start index and a 
    {\em BitString} defining the length and/or value as parameter.
\end{itemize}

Sample Code:
\begin{PEARLCode}
DCL x BIT(4) INIT ('1'B4);
DCL y BIT(4);
DCL z BIT(3);
DCL c BIT(8);
DCL f FIXED(4) INIT(2);
y := '01'B1;  // implicit conversion BIT(2) -> BIT(4)
y := x CSHIFT 1;
z := y AND x;
c := x CAT y;
y := TOBIT f;
f := TOFIXED y;
z.BIT(2,2)  := c.BIT(1,2);
\end{PEARLCode}

\begin{CppCode}
BitString<4> x4(1);
BitString<4> y;
Fixed<4> f(2);
y = (pearlrt::BitString<2>)(1);
y = x.bitCshift(1);
z = y.bitAnd(x);
c = x.bitCat(y);
y = BitString<4>(f.x);
f = y.toFixed();
z.setSlice(2,c.getSlice(1,BitString<2>dummy));
\end{CppCode}

\section{CHARACTER}
The data type CHARACTER in PEARL has a definite and fixed length.
The templated class {\em Character} provides the storage allocation
and necessary methods for this data type.
All operations check for range overflows and throw exceptions in case of
range violation.

The language report states at one location that the length is of type FIXED(15).
Therefore the supported lengths are 1 to 32787.

The preset of a character variable may be set as C-string constant.

\begin{classSummary}
 Class & \verb|Character<size_t len>| \\
 Specification & Character.h  \\
 Implementation  &  Character.cc \\
 size   &  1 byte per Character \\
 max length & 32767  \\
\end{classSummary}

\begin{methodMapping}
  \verb|UPB|      & \verb|upb()|    &  {\em returns size\_t}\\
  \verb|TOFIXED| & \verb|toFixed()| &     \\ 
  \verb|<|       & \verb|operator<()| &  \\
  \verb|>|       & \verb|operator>()| &  \\
  \verb|<=|      & \verb|operator<=()| & \\
  \verb|>=|      & \verb|operator>=()| & \\
  \verb|==|      & \verb|operator==()| & \\
  \verb|/=|      & \verb|operator!=()| & \\
  \verb|><| (CAT) & \verb|catChar()| & {\em see note below} \\
  \verb|.CHAR(x)| & \verb|setCharAt(..)|&  
				on left hand side in expression
				{\em see note below}\\
  \verb|.CHAR(x)| & \verb|getCharAt(..)| & 
				on right hand side in expression
				{\em see note below}\\
  \verb|.CHAR(x,y)| & \verb|setCharSlice()| & 
				on left hand side in expression
				{\em see note below}\\
  \verb|.CHAR(x,y)| & \verb|getCharSlice()| &
				on right hand side in expression
				{\em see note below}\\
\end{methodMapping}
The slice operations \verb|.CHAR(...)| may be used in PEARL 
on the left hand side and the right hand side of the expression.

Remarks:
\begin{itemize}
\item The slice operations are defined with a different interface than 
   BitString. This was choosen to avoid character copies.
\item the setSliceChar and getSliceChar methods take as $4^{th}$ element
a Character value defining the retrieved/new data
 from/for the container Char-value
\end{itemize}

\begin{PEARLCode}
DCL x CHARACTER(5) INIT ('PEARL');
DCL y CHARACTER(4);
y := x.CHAR(2,5);
x.CHAR(2,5) := y;
\end{PEARLCode}
\begin{CppCode}
Character<5> x("PEARL");
Character<4> y;
getSliceChar(x, 2, 5, y);
setSliceChar(x, 2, 5, y);
\end{CppCode}


\section{REF CHARACTER}
The type REF CHARACTER allows character string with variable length.
A REF CHARACTER needs a normal CHARACTER variable as working storage.
It contains an additional value to maintain the current used length.

The class {\em RefCharacter} provides the necessary operations.

\begin{classSummary}
Class & RefCharacter  \\
 Specification & RefChar.h  \\
 Implementation & RefChar.cc \\
 size   &  12 byte \\
\end{classSummary}

\begin{methodMapping}
  \verb|:=|     & \verb|setWork()|    & REF CHAR := CHAR(x) \\
  \verb|CONT|   & \verb|\verb|store()|    & CHAR := CONT REF CHAR(x) \\
\end{methodMapping}

The content based operations like comparison are done upon the
CHARACTER-type.


\begin{PEARLCode}
DCL wrk CHARACTER(100);
DCL result CHARACTER(50);
DCL rc REF CHAR();
rc := wrk;
CONT rc := 'Hallo';
CONT rc := CONT rc CAT ' Welt.';
result := CONT rc;
\end{PEARLCode}

\begin{CppCode}
Character<100> wrk;
Character<50> result;
RefCharacter rc;
rc.setWork(wrk);
rc.clear();
Character<5> hello((char*)"Hallo");
rc.add(hello);
Character<6> welt((char*)" Welt.");
rc.add(welt);
rc.store(result);
\end{CppCode}

\section{DURATION}
The values of type DURATION are managed by the class {\em Duration}.
The values are stored in a 64 bit integer ({\em Fixed63}).
The base unit is $1 \mu s$.
The range of 64 bit allows durations up to $\approx 100 days$.

All calculations on durations are done within the full possible range.
Thus values larger than 24 hours may occur.
Even negativ values are possible.

The mapping of duration values to the C++ types is done via the type
\verb|double| giving the duration in seconds.

Operations are defined as required by the language to mix DURATION, CLOCK, 
   FIXED and FLOAT.

\paragraph{Notes:}
\begin{itemize}
\item The division Duration / Duration delivers a \verb|Float<24>|, 
   which may be assigned to larger Float type
\item The division Duration/Float is realized as template for 
   all Float-lengths 
\end{itemize}

\begin{classSummary}
 Class & Duration \\
 Specification & Duration.h \\
 Implementation & Duration.cc \\
 size   &  8 bytes \\
\end{classSummary}


\begin{methodMapping}
  \verb|+| (monadic)  &        & to be treated by the compiler \\
  \verb|-| (monadic)  & \verb|operator-()|  & \\
  \verb|ABS|          & \verb|abs()|  & \\
  \verb|SIGN|         & \verb|sign()|  & \\
  \verb|+| (dyadic)   & \verb|operator+(rhs)| & duration+duration \\
  \verb|-| (dyadic)   & \verb|operator-(rhs)| & duration-duration \\
  \verb|*| (dyadic)   & \verb|operator*(rhs)| & duration*\verb|Float<S>| \\
  \verb|/| (dyadic)   & \verb|operator/(rhs)| & duration/duration returns
        \verb|Float<24>| \\
  \verb|/| (dyadic)   & \verb|operator/(rhs)| & duration/\verb|Float<S>| \\
  \verb|<|            & \verb|operator<(rhs)|& \\
  \verb|>|            & \verb|operator>(rhs)| &  \\
  \verb|<=|           & \verb|operator<=(rhs)|& \\
  \verb|>=|           & \verb|operator>=(rhs)| & \\
  \verb|==|           & \verb|operator==(rhs)| & \\
  \verb|/=|           & \verb|operator!=(rhs)| & \\
\end{methodMapping}

\begin{PEARLCode}
DCL minute DURATION INIT(1 MIN);
DCL d1 DURATION INIT(1 HRS 3 MIN 3.1415 SEC);
DCL d2 DURATION INIT(0.001415 SEC);
d2 := d1 + minute;
d2 := d1 + minute / 60.0;
\end{PEARLCode}
\begin{CppCode}
Duration minute(60.0);
Duration d1(3783.1415);
Duration d2(0.001415);
d2 = d1 + minute;
d2 = d1 + minute / (Float<24>)(60);
\end{CppCode}

\section{CLOCK}
The values of type CLOCK are managed by the class {\em Clock}.
The values are stored in a 64 bit integer ({\em Fixed63}).
The base unit is $1 \mu s$.

All calculations on clock values are done within the
period of 1 day.

The mapping of duration values to the C++ types is done via the type
\verb|double| giving the duration in seconds.

Operations are defined as required by the language to mix DURATION and CLOCK. 

The specification requires a procedure \verb|NOW| which delivers the current
time as \verb|CLOCK|-value. This routine is plattform specific.


\begin{classSummary}
 Class & Clock  \\
 Specification & Clock.h  \\
 Implementation &  Clock.cc \\
 size   &  8 byte \\
\end{classSummary}


\begin{methodMapping}
  \verb|+| (dyadic) & \verb|operator+(rhs)| & Clock + Duration \\
  \verb|-| (dyadic) & \verb|operator-(rhs)| & Clock-Duration and Clock-Clock \\
  \verb|<|          & \verb|operator<(rhs)| &
                    {\em returns bool } \\
  \verb|>|          & \verb|operator>(rhs)| & 
                    {\em returns bool } \\
  \verb|<=|         & \verb|operator<=(rhs)| &
                    {\em returns bool } \\
  \verb|>=|         & \verb|operator>=(rhs)| &
                    {\em returns bool } \\
  \verb|==|         & \verb|operator==(rhs)| & 
                    {\em returns bool } \\
  \verb|/=|         & \verb|operator!=(rhs)| & 
                    {\em returns bool } \\
  \verb|NOW|        & \verb|Clock::now()| & this method is plattform specific\\
\end{methodMapping}

Remarks:
\begin{itemize}
\item the arithmetic operations are defined for also for
   mixed Duration and Clock parameters
\end{itemize}


\begin{PEARLCode}
DCL start CLOCK(11:30);
DCL current CLOCK;
DCL diff DURATION;
current := NOW;
diff := current - start;
\end{PEARLCode}

\begin{CppCode}
Clock start(41400.0);
Clock current;
Duration diff;
current = now();
diff = current - start;
\end{CppCode}

