\chapter{Introduction}

This document describes the overview of the runtime system
of OpenPEARL for all platforms.
The compiler API is described.
Internal methods are documented as far as need for comprehension.
Details on methods and parameters are 
described in the source code based documentation with doxygen.

\section{Structure of Files}
The project targets different platform. 
This is reflected in the directory structure inside the runtime system.

The structure is kept flat. The top level of runtime directories
just contains folders for the diffent platforms and the platform
independent part. Files, which are special to a specific platform
are stored in parallel folders which are named according the content.
\begin{description}
\item[common] contains the platform independent parts.
   The file \verb|Files.common| contains lists which are used in the
  platform specific  \verb|Makefile|
\item[linux] contains the linux specific parts, as well as the common
   linux device drivers. 
\item[FreeRTOS] contains the FreeRTOS specific parts
  \begin{description}
  \item[FreeRTOS] contains FreeRTOS V8.1.2
  \item [PEARL] contains the FreeRTOS specific classes of the runtime system
  \item[addOns] contains modules, which extend ether the c-library, or
                FreeRTOS
  \end{description}
\item[cortex-M] contains files for the target system on base of theARM-Cortex-M
   processor
   \begin{description}
   \item[CMSIS] contains the Cortex-M-System-Interface routines
   \item[LPCOpen] contains the access libriaries for NXP-processors and
              gcc compiler
   \end{description}
\item[lpc1768] constains the processor and board specific
   files for the FreeRTOS port for the Landtiger target board.
\item[includeComposer] contains helper an application to generate 
   one include file for the complete runtime system
\item[codeStyle] contains source code formating check routines
\end{description}

Some modules have a platform independent part and a platform
depended part. The naming is usually done in that way that the 
platform independent part is named as e.q. {\em TaskCommon} and the 
platform depended part is names eq. {\em Task}. This provides a simple
interface towards the compiler, since the compiler just instanciates
an object of type {\em Task}.

\section{Build System}
\label{sec:buildsystem}
The build system for the complete OpenPEARL system is based on the 
configuarion tools of the Linux system and make.
Only \texttt{make menuconfig} is tested.

On top level, there are two commands required
\begin{enumerate}
\item \texttt{make menuconfig}
\item \texttt{make install}
\end{enumerate}

The configuration part provides for the different parts of the system
some configuration files in the folder \texttt{./configuration/fm/}.
The configuration tool generates include files for \texttt{make} and
\texttt{C/C++}.

The top level makefile delegates the job to the main parts
 (compiler, runtime sytem).

The runtime system delegates the build process to specific makefiles 
for each target system. A result, there are
\begin{itemize}
\item a include files (\texttt{PearlIncludes.h} for all target systems
\item platform specific libraries and linker files for each target system
\item one compile/linkrun command (\texttt{prl})
\end{itemize}
The install directory may be specified in the configuration tool.

The target specific makefile work similar to the linux kernel make file.
The contributing directories are included bei makefile extensions and
path substitutions.

In order to distinguish the different targets in the common header include
file, each target system is assigned with an unique number.

\begin{table}[bpht]
\begin{tabular}{|r|c|l|}
\hline
number & short form & target system \\
\hline
1 & linux &  Linux, even Rasperry Pi\\
2 & lpc1768 & LPC1768 Landtiger board \\
\hline
\end{tabular}
\caption{Mapping of Target System on an  Id}
\label{tab:targetId}
\end{table}

\subsection{Minimal Porting Guide for a new Linux derivate}
Linux systems are very similar on the sourcecode level. Differences are
at the presence of devices and especially for maker scene systems at 
the i/o-capabilities.

It should be possible to manage these differences with the 
configuration menu (see chapter \ref{sec:buildsystem})

Use the Raspberry Pi as reference to  add a new derivate. 
Define the required menu items and use them in the Makefile in the 
linux part of the runtime build.

\subsection{Minimal Porting Guide for a new FreRTOS based Microcontroller}
A different microcontroller adds completely new devices and may be different
compilers to the build system. That the reason why there is mor to do.

\begin{enumerate}
\item add a new target system id to table \ref{tab:targetId} and 
      invent a shortform for this target
\item create a suitable directory under \verb|runtime/| - use the same name
   as the short form description
\item add this target into \verb|runtime/Makefile| 
\item provide in the platform specific directory the files for
      build (\verb|cc_bin.inc|) and download/run (\verb|run_bin.inc|)
\item use the files from the lpc1768 platform as reference for your
      platform. The minimal support must provide 
      \begin{itemize}
      \item main.cc
      \item console support via e.g. an UART. This part is divided
            into a low level part (see \verb|Lpc17xxUartInternal|), 
            the dation support (see \verb|Lpc17xxUart|)
            and print-support (\verb|SystemConsole.cc|).
      \item adopt Makefile for your required files
      \item adopt makefile.conf for compiler and linker settings
      \item adopt PearlIncludes.h. It must list the common and specific 
            header files for this new platform
      \item adopt the startup routine (see \verb|startup_lpc1768.S|)
      \item adopt the linker script (see \verb|OpenPEARLlpc1768.ld|)
      \end{itemize}
\item check the FreeRTOSConfig.h setting for your device
\item in case you need some files unchanged, please move them into a 
      directory which is used by all targets. If the source depends on
      FreeRTOS, place the folder under FreeRTOS, if it depends on 
      Cortex-M, move it under cortexM. In other cases find a good place for
      it.
\end{enumerate}

\section{Companion Tools}
\subsection{Include Composer}
There are a lot of header files in the runtime system. 
The headers may include other headers again.
To allow an easy deployment the tool {\em include composer} 
was build using lex and yacc. It reads one header file {\em PearlIncludes.h}
and creates one header which contains all definitions from the headers
included bei PearlInclude.h. The system includes are treated in that way
that they appear only once in the resulting header file.

\begin{itemize}
\item The tool treats the preprocessor statements 
ifdef, ifndef, else, endif, define undef.
\item The system includes (path separator \verb|<..>|) are collected
and dumped in order to speed up the compilation process of the 
pearl system.
\item ifdef/ifndef statemente are evaluated. Only really required code
   is passed to standard ouput stream.
\item Limits may be changed in the source code: 
   \begin{itemize}
   \item length of defined identifier: 59 characters
   \item number of defines identifier: 200
   \item include depth: 10
   \item IF depth: 20
   \item include file name length: 29
   \end{itemize}
\item The user includes (path separator \verb|"..."|) are treated
\end{itemize}
 

\subsection{Delete White Space Lines}
This tool removes multiple lines, which contain only white space characters
and reduces them to one line.

\section{Code Generation}
There are some point to be considered at code generation.
\begin{itemize}
\item to avoid name conflicts, the identifiers from the PEARL code
   are prepended with a prefix  (underscore)
\item the runtime system is placed in the namespace \verb|pearlrt|.
\item the compiler output code is in no namespace
\end{itemize}

\textbf{
In this document the namespace qualifiers  and name decorators 
are omitted in many places for better readability.
}

Signal handling and device definitions need the definition of 
different components, which are related to one user define pearl 
element.
The corresponding chapters define a decoration scheme. If suitable
for the compiler othe schemes are possible. 


\section{Internal Helper Classes}
\begin{description}
\item[CSemaCommon]  provides the interface definition for counting semphores
\item[MutexCommon] provides the interface definition for mutexes
\item[IfThenElseTemplate] provides conditional compilation
\item[NumberOfBytes] provides templates for size of data computations
\end{description}
The documentation of runtime internal classes is not subject of this document.
Please refer to the doxygen output.

\section{Code Portability}
The code of the runtime system is according C++ standard.

The following extensions provided by gcc are used:
\begin{description}
\item[labels as values] is used in the signal handling. 
   If this featureis not available, the compiler must produce a switch 
   statement instead of the {\em computed goto} to resolve
   the current goto target.
\item[\_\_attribute\_\_(format)] is used in the log facility to check
   proper argument matching according the printf-semantic.
   This option may be removed without problems, as long as the logging
   calls are correct.
\item[\_\_attribute\_\_(used)] is used in the FreeRTOS port to bypass linkage
   problems which appear with FreeRTOS and LTO (Link Time Optimization)
\item[\_\_attribute\_\_((section("...")))] is used to control the
   memory usage for the microcontroller.
\end{description}
   
