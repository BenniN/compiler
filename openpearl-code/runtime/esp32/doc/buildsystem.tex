\section{Buildsystem}

The ESP32 is provided and supported by Espressif (www.espressif.com).
An IoT debelopment Framework (IDF) and a c/c++ buildsystem are provided. 

The IDF is structured in socalled {\em components}. An IDF-Project contains
the project specific files which are combined by the framework with the 
framework components. The build of an ESP-IDF application is done 
{\em out of place} --- in the build folder of the application folder.

The OpenPEARL buildsystem requires the invocation of the suitable 
g++ compiler/linker upon the c++ files, which result from the PRL to C++ 
translation of the OpenPEARL compiler.
For the usage of the ESP-IDF by OpenPEARL, the files of the IDF should NOT 
become compiled for each PEARL application build.

\subsection{Integration of the ESP-IDF with the OpenPEARL-Repostory}
The ESP-IDF is extended in high rate. Thus we do not copy a specific
version of the ESP-IDF in our repository.

The OpenPEARL build system is based on \verb|make|. The build is done 
currently {\em in place}.
The OpenPEARL specific parts of the runtime system is added to the ESP-IDF
by setting some links into the OpenPEARL working directory.
E.g. \verb|/user/xxx/Openpearl/openpearl-code/runtime/esp32/openpearlspecific ~/esp/esp-idf/components/openpearl|
After this, the ESP-IDF recognize a new component named 'openpearl' and
 treat this like any other ESP-IDF component.

To use the ESP-IDF, the steps for installation must be done as described in 
getting started guide from Espressif.
To check the installation an presence of all prerequisites you should build
 one the provided examples of the ESP-IDF as described there.

The \verb|Makefile|  for the esp32 port of OpenPEARL creates automatically all
required links.

\subsection{cc\_bin and run\_bin}
The overall script 'prl' for compiling and linking of an OpenPEARL application
needs platform specific commands to compile c++ files and link them with the 
platform specific libraries.

For each platform two files must be provided named 'cc\_bin.inc'
and 'run\_bin.inc'. They are include during the installation of the OpenPEARL
distribution in the overall 'prl' script.

\section{Kconfig parameters}
There are some parameters for the OpenPEARL configuration.
\begin{description}
\item[ESP32\_IDF\_PATH] must point to the installation location of yout ESP-IDF.
\item[ESP32\_FLASH\_INTERFACE] must identify the device name
   of your interface to the ESP32
\end{description}
More item may be available in future.

\section{FreeRTOS}
The ESP32-IDF provides a modified version of FreeRTOS which supports a kind of 
symetric multiprocessing.
This ESP-IDF component needs a \verb|FreeRTOSConfig.h| file.
OpenPEARL needs some setting in the \verb|FreeRTOSConfig.h|. 
This configuration is part of the OpenPEARL repository. The ESP-IDF provided
configuration file is replaced by the OpenPEARL specific version.

\section{Notes about updates in ESP-IDF}
If a new release of ESP-IDF should be used, several checks must be done:
\begin{itemize}
\item update the \verb|sdkconfig| in the esp-idf-project in the repository.
\item check changes in the ESP-IDF \verb|FreeRTOSConfig.h|
\item update the xtensa compiler to the new version
\item the FreeRTOS implementation of tasks.c and tasks.h was extended by one
  funktion. This must be added by include in taskAddOns.c.inc and taskAddOns.h
\end{itemize}

\section{Open Issues}
\subsection{Core Usage}
Currently only one core is used.
In future, the second core should be used also.
The automatic restart of an application task assumes that only one processor core ist 
available.
There are two possibilities to overcome this obstancle:
\begin{enumerate}
\item enshure that all application tasks are on the same core and use the othr core for i/o-tasks
\item add a patch to FreeRTOS to restart an overrun task automatically at its termination.
\end{enumerate}

\subsection{Esp32Uart Device}
The ESP-IDF is subject of rapid changes. Therefore it is not 
useful to modify any provides code for usage of the internal devices.

Especially the uart is very complicated. Thus a separate FreeRTOS thread 
is used for the write operation. The read operation uses the 
provided mechanism with an event queue.

Suspend and terminate during i/o operations are treated by a simulation
of an receive event or semaphore unlocking for send done.


\subsection{missing Device Drivers}
\begin{enumerate}
\item Console
\item I2C
\item Ethernet
\item ...
\end{enumerate}


