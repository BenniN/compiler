\section{System Overview}

\subsection{System Architecture}
Several components are required for the runtime system.
All directories are given relative to \verb|openpearl/runtime|

\begin{description}
\item[./FreeRTOS\_lpc1768] contains board specific items like
Makefile, linker script, peripheral dependend stuff like real time clock,..
\item[./FreeRTOS] contains items, which only depend on FreeRTOS like
 \verb|FreeRTOSConfig.h| task and semaphore mapping.
\item[./FreeRTOS\_V8.1.2] contains the files from FreeRTOS itself.
 The files must copied into this directory from the
  location of your FreeRTOS download. The files from 
  {\em portable/GCC/ARM\_CM3}  must be copied also into this directory.
  The directory \verb|portable/MemMang/| must alo be copied. 
  In the momenet \verb|heap_3.c| is used. 
\item[./common] contains all items form the runtime library, which are common
  to all plattforms.
\end{description}
All objects are packed in the PEARL-library \verb|plib.a|.\footnote{Due
 to LTO usage
the gcc-ar frontend must be used for archiving the objects.
}

Each directory contains a file named like \verb|Files.common|, which is
included from the plattform specific Makefile in order to add the 
specific files to the make process.

\subsection{System Initialization}
The system initialization  is controlled by the following files:
\begin{description}
\item[gcc\_lpc1768.ld] is the linker script. This files is derived from
   the samples from the Launchpad cross-tool-chain file gcc.ld. 
   Only the ram and rom sizes were modified to fit the LPC1768 
\item[startup\_ARMCM3.S] contains the interrupt vector table and the 
   reset handling routine. The reset handler 
   \begin{enumerate}
   \item copies the preset values from flash memory to RAM
   \item clears the BSS section 
   \item  calls the SystemInit() function
   \item  branches to the initialization of glibc (\_start); glibc
      initializes inself and the static objects and branches into \verb|main()|
   \end{enumerate}
\item[systeminit.c] contains the actions, which must be executed
   before the clib is started. Up to now the system core clock is 
   set.
\item [retarget.c] contains some functions, which redirects console 
   input and output to the UART0.  This allows the use of printf(), 
   scanf(), cin and cout in the application. The UART initialization is done 
   upon first usage of any i/o-operation of this module.
\item[main.c] contains the application start code. The major steps are:
  \begin{enumerate}
  \item send a list of defined tasks to the Log::debug-interface
  \item activate all \verb|MAIN| tasks according their priority
  \item initialize the real time clock (RTC)
  \item initialize the software time from the RTC
  \item start the FreeRTOS scheduler 
  \end{enumerate}
  The termination is done in the class \verb|TaskMonitor|. 
\end{description}

\subsection{Linkage}
The gcc linker option LTO is used to remove unused functions and data. 
The inidividual test and application programs are linked together with 
the static library \verb|plib.a|.

\subsection{Unit Tests}
The linux trunk of the project, which is the main trunk, uses the
goole test framework to run unit tests on nearly all internal classes.
The complete gtest framework is too large to be run on the microcontroller.

A simple test framework, which supports the same syntax as gtest is provided
to run the unit tests on the target system.
The implementation resides in the files
 \verb|FreeRTOS_lpc1768/tests/simpleTests.h| and 
 \verb|FreeRTOS_lpc1768/tests/simpleTests.cc| 

The following elements of gtest are provided:
\begin{description}
\item[TEST] defines a unit test
\item[EXPECT\_EQ] tests, whether the two given parameters are equal, 
   assuming an operator== exists.
\item[EXPECT\_STREQ] tests, whether the two given parameters are equal c string 
\item[EXPECT\_TRUE] tests, whether the  given parameter is non zero. 
\item[EXPECT\_FALSE] tests, whether the  given parameter is zero. 
\item[EXPECT\_THROW] tests, whether the first given expression throws
   an exception of the type as given as  second parameter.
\item[ASSERT\_...] is defined for all listed EXPECT-versions. 
\end{description}

\section{Task Mapping}

The task mapping is described in the thesis of Florian Mahlecke in detail.

In short:
\begin{itemize}
\item GenericTask.h defines C-macros for task declararion and specification
\item Each PEARL task is derived from the class {\em Task} (\verb|Task.cc|).
      The tasking methods are implemented in \verb|Task.cc|.
\item Each PEARL task implements the method \verb|task(TaskCommon*me)| 
    with the C++ code generated by the compiler.
\item {
  \begin{description}
  \item[class Task] provides the required plattform specific implementation
     of the tasking methods
  \item[class TaskTimer] provides the time related stuff for the tasking
     methods. This class is currently working on base of the 
     FreeRTOS tick.
  \item[class PrioMapper] provides a 1:1 mapping of PEARL to FreeRTOS
      priorities
  \end{description}
}
\end{itemize}

\subsection{class TaskTimer}
The class TaskTimer provides a very simple timer facility.
It uses a oneshot timer from freeRTOS. The callback function retriggers
the timer if required. There is currently no mean provided to correct 
time shifts due to the execution duration of the callback function.
The minimum delay is 1 tick. In case of a required delay of less than 1 tick,
the delay is set to 1 tick. There is no rounding mechanism in the 
transformation od delay time to ticks.


\subsection{To be done}
  \begin{itemize}
  \item allocate static TCB+Stack as patch for FreeRTOS to avoid 
    memory fragementation due to ACTIVATE/TERMINATE cycles.
  \item provide device drivers for the standard devices (UART, SD, Ethernet)

  \end{itemize}


