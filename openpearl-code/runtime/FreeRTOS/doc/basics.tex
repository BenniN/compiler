\section{System Overview}
FreeRTOS is used as scheduler. This system was selected, because
\begin{itemize}
\item it provides preemtive priority scheduling
\item it is avaliable on nearly every micro controller system
\item it is available in source code
\end{itemize}

PEARL tasks are mapped on FreeRTOS tasks. The required macros
are provided by \texttt{GenericTask.h}.

The FreeRTOS system has the following disadvantages:
\begin{enumerate}
\item task control blocks and task stack are allocated on the system
   heap on task creation. This may lead to memory fragmentation
\item ... and more ?? ... 
\end{enumerate}

Item 1 is treated by a modification of FreeRTOS with static TCB and stack
and an user supplied heap manager, with implements a pool allocation pattern.

\section{System Architecture}
There are some problems with the linker and \texttt{weak} symbols.
The effect is that \texttt{weak} symbols are not overwritten, if
a corresponding strong symbol resides in a library.
\texttt{weak} symbols are usually used in the interrupt vector table.
This leads to
\begin{enumerate}
\item an object file which contains the interrupt service handlers like
   reset-handler, timer and other device service routines
\item a runtime library with the remaining object files
\end{enumerate}

\subsection{System Initialization}
The system initialization is described in the target specific part.

\subsection{Linkage}
The gcc linker option LTO is used to remove unused functions and data. 
The individual test and application programs are linked together with 
the static library.

\section{Unit Tests}
The linux trunk of the project, which is the main trunk, uses the
goole test framework to run unit tests on nearly all internal classes.
The complete gtest framework is too large to be run on the microcontroller.

A simple test framework, which supports the same syntax as gtest is provided
to run the unit tests on the target system.
The implementation resides in the files\footnote{ugtest stands for $\mu$gtest}

 \verb|ugtests/simpleTests.h| and 
 \verb|ugtests/simpleTests.cc| 

The following elements of gtest are provided:
\begin{description}
\item[TEST] defines a unit test
\item[EXPECT\_EQ] tests, whether the two given parameters are equal, 
   assuming an operator== exists.
\item[EXPECT\_STREQ] tests, whether the two given parameters are equal c string 
\item[EXPECT\_TRUE] tests, whether the  given parameter is non zero. 
\item[EXPECT\_FALSE] tests, whether the  given parameter is zero. 
\item[EXPECT\_THROW] tests, whether the first given expression throws
   an exception of the type as given as  second parameter.
\item[ASSERT\_...] is defined for all listed EXPECT-versions. 
\end{description}

\section{Service Task}
There is a high priority task, which perform service jobs from 
other tasks.
The priority of the service task is above all PEARL application
tasks.
This is similar to jobs in an conventionel operating
system. 

The service task receives the jobs via a FreeRTOS message queue. 
The queue entries consist of a struct like:
\begin{verbatim}
typedef struct { 
  void (*job)(void*param);
  void * param;
} ServiceJob;
\end{verbatim}
The first entry is a pointer to a function, which receives the second entry
as parameter.

This service task is used by the
\begin{description}
\item[timer facility], when a time period has expired
\item[OpenPEARL tasks], if they need a direct restart at their
     termination in cases of interrupt overrun.
\item[interrupt devices] to trigger the internal treatment
\end{description}

\section{System Console}
Each plattform needs a system console for emergency error logging.
The retarget mechanismn of the glibc needs access to stdin and stdout
to realize printf and scanf. These functions are nice to use during 
system debugging.
An uart is the first choice for this purpose. It is desired to multiplex
the system console with a normal dation. 

The  architecture consists a GenericUart class, which defines the low 
level interface 
of a concrete uart (e.q. Lpc17xxUartInternal). This class is
realized as singleton to make shure that only 1 instance for a 
concrete uart is instantiated.
The GenericUart class performs the multiplexing of retargeting,
user dation and actions for the xon/xoff-protocol.

The GenericUartDation class defines the additional interface of a
SystemDationNB call, which realizes a serial system dation. 

The interface for the retarget job is realized in the Retarget class.
This class implements the basic C functions  \verb|_read| and \verb|_write|,
which delegates to a plattform specific class SystemConsole, which
is a singleton. 
The implementation of \verb|_read| and \verb|_write| starts in polling mode
and is switch over to interrupt mode, when FreeRTOS is started.

The class SystemConsole contains just the singleton and instantiates 
the system consol uart with plattform specific parameters.

The concrete SystemDationNB class for the uart with a system console
must use the low level implementation  of the uart. It must implement
the interfaces from SystemDationNB and GenericUartDation. 

\paragraph{Summary:}\ \\
GenericUart, GenericUartDation and Retarget are common to all micro
 controllers.

The plattform specific classes are very small:

\begin{itemize}
\item SystemDation
\item the low level uart interface
\item the complete uart interface
\end{itemize}

By this architecture the port to other controllers should be very simple.

\section{Time Base}

The presence of a real time clock (RTC) is not guaranteed on a
microprocessor system.
This is no big problem for most PEARL applications, since they works 
in many cases with relative times.

FreeRTOS provides a tick based system time. The resolution is specified
in the file \texttt{FreeRTOSConfig.h} --- 
set in OpenPEARL to $1 ms$.

Some applications many require absolute times and more precise timing.
That's the reason why the FreeRTOS provided timers are not used. 
There is an implemtation of the \texttt{itimer} system. This 
implementation allows the specification of a time source.
This system was introduced  by Jonas Meyer in his thesis.

\subsection{Timer-Treatment}
The principle of operation of the implementation of the timer facility is
that there are two lists of timer entries. A timer entry contains 
the information about an initial delay and a repetition rate of the timer.
This is similar to the itimer of POSIX.
The operation is cyclic:
\begin{enumerate}
\item the definition of a new timer inserts the correspondig timer entry
   in the sorted list of active timers
\item The timer entry which elapses first defines the setup of a hardware timer.
\item As soon as the hardware timer elapses, the interrupt service  routine
triggers the timer task to continue.
\item The timer task checks which timers have elapsed and performs the 
callback of the defined timer. This callback is implemented in the
class TaskTimerCommon.
\item if an elapsed timer should operate repetitive, the timer is rescheduled
   at the time $currentSchedule + period$. \
	If the elapsed timer is a oneshot timer, it is added in the free-list.
\item the next timer entry in the active list defines the new delay for the
  next hardware timer delay. This delay is limited to one second. If the 
  desired period is longer than one second, several iterations are required.
\end{enumerate}

The implementation of the hardware specific part
(e.g. interrupt service routine) must be implemented in the concrete
timer facilities. The FreeRTOSClock provides this bevavior on base
of the 1ms system tick.

Each timer informs the service task about time exprirations and 
sends its own callback function pointer for further treatment.

\subsection{class FreeRTOSClock}
The class TaskTimer provides the FreeRTOS tick as
base for itimers.
The absolute time is set to 1.1.2016 0:0:0 at system startup.

Other time sources are plattform specific and described there.

\section{Modification in FreeRTOS}
FreeRTOS was modified in the following points:
\begin{description}
\item[static tcb and stack:]  Normally FreeRTOS places task control blocks (tcb)
   and task stack in the heap storage. This may produce memory fragmentation,
   when application tasks are deleted and restarted in a random sequence.

   A new configuration item in \texttt{FreeRTOSConfig.h}  changes this
   behavior. If \texttt{configUSE\_STATIC\_STACK\_AND\_TCB} is set to 1
   the function \texttt{xTaskCreate} assumes a pointer to a struct with
   pointers to the previous applications parameters and two new pointers
   to the tasks control block (with type \texttt{TCB\_t}) and tasks stack
   (type \texttt{StackEntry\_t *}). The application is expected to be
   responsible to pass valid pointers.
\item[getTaskCurrentFreeStack] is a new function, which returns the number
   of free stack element (of size 32 bit) on the current active tasks stack.
   This feature is enabled by the option
    \texttt{INCLUDE\_xTaskGetCurrentFreeStack}
\end{description}

\section{Stack Checking}
The gcc tool chain provides the possibility of code instrumentation
for tracing. The option \texttt{-finstrument-functions} additional
function call at each function entry and exit of the compilation unit.
The corresponding functions must be supplied by the application program.

The signature of the functions is:
\begin{verbatim}
extern "C" {
void __cyg_profile_func_enter(void*this_fn, void* call_site)
     __attribute__((no_instrument_function));

void __cyg_profile_func_exit(void*this_fn, void* call_site)
     __attribute__((no_instrument_function));
};
\end{verbatim}

The configuration option \texttt{CONFIG\_LPC1768\_CHECK\_STACK\_OVERFLOW}
controls the usage of the compiler option and inclusion of the
module \texttt{FreeRTOS/addOns/checkStackOverflow.cc}.


In case of the remaining stack of less than 250 elements (1000 bytes), or a
remaining unused size of less than 250 elements will cause the raising
of the signal StackOverflowSignal. This signal is intended to be non catchable
by the application program. 
The compiler arm-non-eabi-g++ v4.9.3 proved to need approximatelly 190 stack elements to perform the stack unrolling, the limit should not be set with caution.

The monitoring of functions of the template classes for Fixed and Bit 
is switched off for runtime efficiency.

\section{Heap usage}
FreeRTOS uses for several resources like tcp, stacks, queues, semaphores 
on the heap.
For OpenPEARL, the heap is configured as static data array of a fixed size.
A new heap allocator (heap\_1\_2.c) provides a {\em pool allocation pattern}
for heap elements to avoid memory fragmentation, if any heap element
of freed and allocated cyclic.

Usually FreeRTOS tasks locate stack and tcb on the heap. 
These elements are allocates and released at task start and termination.
This would produce heap fragmentation, when task stacks have different lengths.
To avoid the fragmentation, stacks and tcbs are allocated static in the
application code.

A OpenPEARL task needs aside from the applications data:

\begin{tabular}{lcc}
    & .bss & FreeRTOS Heap \\
tcb & 300 byte & 88 byte \\
stack & 30 - 2400 byte & - \\
\end{tabular} 

Note: Stack checking requires a unused buffer of 800 bytes!

Semphores and queues are not deleted during the lifetime of an 
OpenPEARL-application. They are still allocted on the heap of the application.
Note: All OpenPEARL semaphores uses one (1!) FreeRTOS semaphore together.

The heap of the c-library is made available by the linker script. 
Maybe some C++-objects are allocated in this heap.
This should only be used, if these objects are never deleted.


\section{Stack Size and Allocation Control}
{\em missing part; 
proposal: pseudo devices allow the specification of stack size.

this will be solved together with the cross module linker
}

\section{Added Features to FreeRTOS and glib-c}
There are some additional functions in \texttt{FreeRTOS/addOns},
 which supply the toolchain
with UNIX-like time functions like gettimeofday.
\begin{description}
\item[clock.c]provides gettimeofday and some more
\item[timer.c] provides the itimers
\item[testStackOverflow.cc] provides the implementation of the 
  stack checking profiling function
\item[service.c] provides the service task
\end{description}

Implementation of FreeRTOS Hooks:
\begin{description}
\item[assert.c] provides the implementation of the function
  \texttt{assert()}, which is used in FreeRTOS to verify
  proper operation
\item[FreeRTOSHooks.c] provides some error hooks for stack and head overflow.
\end{description}

\section{Task Mapping}

The task mapping is described in the thesis of
 Florian Mahlecke in detail.

In short:
\begin{itemize}
\item GenericTask.h defines C-macros for task declararion and specification
\item Each PEARL task is derived from the class {\em Task} (\verb|Task.cc|).
      The tasking methods are implemented in \verb|Task.cc|.
\item Each PEARL task implements the method \verb|task(TaskCommon*me)| 
    with the C++ code generated by the compiler.
\item {
  \begin{description}
  \item[class Task] provides the required plattform specific implementation
     of the tasking methods. The automatic restart of a task in case of
     \verb|WHEN ... ACTIVATE| overruns is treated by the service task.
  \item[class TaskTimer] provides the time related stuff for the tasking
     methods. This class is working on base of the itimer extension 
     in \texttt{FreeRTOS/addOns}
  \item[class PrioMapper] provides a 1:1 mapping of PEARL to FreeRTOS
      priorities. Furthermore is provides the list of other used
      priorities in the system for background threads like the timer task.
      For details about the list of background task, please 
      check the doxgen documenation of the class PrioMapper.
  \end{description}
}
\end{itemize}

\section{Logging}
The logging is done slightly different to the linux version.
In order to safe stack in the application tasks, the formatting is passed
to a special log-task which runs with high priority.
The logging class checks for each logging request if the FreeRTOS scheduler
is running. If the scheduler is not running - the formatting is done in the 
context of the calling thread. If the scheduler is running, the parameters
are passed to the log-task.
This mechanism reduces the stack usage of the logging down to 17 words. This
is important for the case of logging the stack overflow situation.

\section{Device Driver}
The device drivers should be implemented with zero-copy principle.
If a background task is suitable for the device operation, the 
tasks priority must be registered in the class PrioMapper. The usage
of the the same priority for several devices is ok.

Usually, the device driver is called in the context of the calling thread.
The operation of the driver job is passed to an interrupt service routine and
the calling thread is blocked by a FreeRTOS mechanism e.q. binary semaphore.
At completion of the driver job in the isr the calling thread is continued.

A device driver is ether derived from \verb|UserdationNB|,
 \verb|SystemdationB| or \verb|Interrupt|.
It must implement the virtual methods according their
expected behavior. For details please check the doxygen documentation
of these methods.

UserdateionNB works internally only with the newline character to 
perform a skip. The translation into carriage return and linefeed must
be done in the driver if this is desired.

The usual device driver does not support reentrant operations from
different  tasks. 
The mechanism of the \verb|USERDATION| enshures that only one dation
operation is avtive at on time on the same userdation. In cases, where
a system device may be used in more than one userdation, a buffering
algorithmn must be implemented. 

The usual kind of operation is:
\begin{enumerate}
\item provide an interrupt service routine with a name fitting to the
   setting in \verb|startup_xxx.S|
\item choose an interrupt priority, eg. 10
\item check the device parameters in the ctor. The ctor is the last chance to
   allocate any memory on the FreeRTOS-heap.
\item provide the virtual methods from the base class. Mutual exclusion
   between the API methods and the interrupt service routine may
   be achived by disabling the corresponding interrupt in the NVIC.
\item create test applications for 
  \begin{enumerate}
  \item google tests for e.g. tests in the ctor
  \item sample PEARL application, maybe  written in C++ to verify
        the device operation
  \end{enumerate}
\item add device description is the documentation, as
  \begin{enumerate}
  \item section in the plattform specific part
  \item as entry in the table of devices
  \end{enumerate}
\end{enumerate}


\section{Cortex-M Interrupt Priorities}

\subsection{Cortex-M Basics}

As descibed in \cite{yiu}, the Cortex-M processor supports
 in principle up to 256 interrupt priorities.
Lower numbers denote better priorities.
The execution of interrupt service routines is carried out in order of
the priority of the interrupt.
There is also a grouping of interrupt priorities which defines, that 
interrupts of the same group are executied sequentially in order of the
priority.

The cmsis library in LPCOpen defines \verb|#define NVIC_PRIO_BITS 5|

This means, that only the upper 5 bits of the priority level is regarded.

\begin{table}
\begin{tabular}{|c|c|c|c|c|c|c|c|}
\hline
Bit 7 & Bit 6 & Bit 5 & Bit 4 & Bit 3 & Bit 2 & Bit 1 & Bit 0 \\
\hline
\multicolumn{5}{|c|}{Implemented}&\multicolumn{3}{c|}{Not implemented} \\
\multicolumn{5}{|c|}{}           &\multicolumn{3}{c|}{read as 0} \\
\hline
\multicolumn{7}{|l|}{Preemption Priorities (Group=0)}& x \\
\hline
\multicolumn{4}{|l|}{Preemption Priorities (Group=3)}& sub & x & x & x \\
\hline
\end{tabular}
\caption{Priority Level register with 5-bit implemented. With 
priority grouping of 0, there are no subgroups due to 5 implemented
interrupt priority bits.\cite{yiu}}
\end{table}

The cmsis-function  \verb|NVIC_SetPriority| shifts the given priority value 
according the number of implemented priority bits. This leads to
valid parameters from 0 to 31 --- reflecting the processor interrupts
0,8,16, ...248.

To set the interrupt grouping, the cmsis function 
\verb|NVIC_SetPriorityGrouping| is supplied.
The priority group (which is a plain number) defines the upmost bit of the
interrupt priority as the most significant bit
 of the socalled subpriority field.
In reset state, the priority group is 0 resulting in preemptive interrupts
 with numbers in bit[7:1] and a sub priority field bit[0].
Since only the implementation of the LPC1768 supports only 5 priority bits, we
get a sub priority field, which is always 0 and a 32 preemptive 
interrupt levels. Interrupt priority grouping become effectiv with a group 3.

\subsection{FreeRTOS Settings}
FreeRTOS states
\begin{itemize}
\item  that there may be some grouping on interrupt priorities 
by third party software. 
\item the priority value in FreeRTOS are already left adjusted according the
number in priority bits. 
\end{itemize}
FreeRTOS defines the priority of the kernel to be the highest number, which
means the lowest interrupt priority. In our case it is 31, which results into
248. 
FreeRTOS reserves some priorities, which are dedicated to interrupts, which do not interact with FreeRTOS. They are never blocked by FreeRTOS.
This level is defined as 5, resulting in priority 40.

Interrupts from device drivers may use the interrupt levels from 5 to 30.
All of them are preemptive. 
The preemption of interrupts may leed to a stack problem, it is a part of
system optimization to use different interrupt levels in the future.
As long as interrupt preemption is not needed, all device drivers should use 
a common interrupt priority of 10, which leaves space in both directions.
Since there is no subpriority avaliable, the 
interrupts are executed according the interrupt number --- 
the smaller exception number is executed first.




\subsection{To be done}
  \begin{itemize}
  \item provide device drivers for the standard devices (USB, SD, Ethernet)
  \item device wait queue for tasks in begin/end-sequence?
  \end{itemize}


