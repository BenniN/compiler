\section{Devices}

\subsection{Lpc17xxDigitalIn}
The device Lpc17xxDigitalIn provides digital input bits on all
GPIO bits of the LPC1768. It is {\em NOT} checked, whether the 
required bits are in use by other devices.

The LPC1768 provides 5 GPIO ports, numbered from 0 to 4.
Each port has a maximum of 32 bits.
It is stated in the user manual, that a lot of bits are not
available.

Synopsis: \verb|Lpc17xxDigitalIn(port, start, width)|

\begin{description}
\item [port] specifies the port number. Valid numbers are 0,1,2,3,4
\item[start] specifies the number of leftmost bit of the group. Valid numbers
     are 31..0
\item [width] specifies the number of bits which are grouped together.
     Valid numbers are 1..32.
     The group must fit into one port.
\end{description}

The access to the dation is done by reading a BIT(width)-value from the
opened system dation.
Writing other data than BIT-types  may cause a signal, if the size of the
data does not match the required size according the width.

\subsection{Lpc17xxDigitalOut}
The device Lpc17xxDigitalOut provides digital output bits on all
GPIO bits of the LPC1768. It is {\em NOT} checked, whether the 
required bits are in use by other devices.

The LPC1768 provides 5 GPIO ports, numbered from 0 to 4.
Each port has a maximum of 32 bits.
It is stated in the user manual, that a lot of bits are not
available.

Synopsis: \verb|Lpc17xxDigitalOut(port, start, width)|

\begin{description}
\item [port] specifies the port number. Valid numbers are 0,1,2,3,4
\item[start] specifies the number of leftmost bit of the group. Valid numbers
     are 31..0
\item [width] specifies the number of bits which are grouped together.
     Valid numbers are 1..32.
     The group must fit into one port.
\end{description}

The access to the dation is done by writing a BIT(width)-value to the
opened system dation.
Writing other data than BIT-types  may cause a signal, if the size of the
data does not match the required size according the width.

\subsection{Lpc17xxInterrupt}
The device Lpc17xxInterrupt provides interrupt sources from GPIO inputs.
The LPC1768 can produce interrupts on both edges of the GPIO ports 0 and 2.
The Lpc17xxInterrupt device produces interrupts {\em only on the falling edge}. 

Synopsis: \verb|Lpc17xxInterrupt(port, bit)|

\begin{description}
\item [port] specifies the port number. Valid numbers are 0,2
\item[bit] specifies the number of the  bit to monitor.
 Valid numbers are 31..0. 
\end{description}

Note: The LPC1768 does monitor only bits
0-11 and 15-30 on port 0 and 0-13 on port 2.

The landtiger board has Key1 on P2.11 and SW2 an P2.10.
For usage of SW2 the jumper JP5 must be closed (which is the
usual position if working with lpc21isp download).


\subsection{Lpc17xxUart}
The device Lpc17xxUart provides access to the serial ports of the LPC1768.
Since the LandTiger board provide access only to uart0 and uart2, which
both lack of hardware control lines, no hardware handshake is provided 
in the driver.
The device provides the possibility of line editing as well as raw data
transfer.
The used terminal must support backspace for line edit functions!

Synopsis: \verb|Lpc17xxUart(port, baud, bits, stopbits, parity, mode)|

\begin{description}
\item [port] specifies the port number. Valid numbers are 0,2
\item[baud] specifies the baud rate. The usual baud rates from 300 up to 115200 are supported.
\item[bits] the number of bits per character (5-8 are ok)
\item[stopbits] the number of stopbits
\item[parity] parity 'O', 'E', or 'N'
\item[mode] provides additional switches
   \begin{description}
   \item[Bit 0-15] Input buffer size --- required for line edit function
   \item [Bit 16] 1=line editing enabled; needs an input buffer size > 0
   \item [Bit 17] 1=xon/xoff; 0=no protocol
   \end{description}
   Examples:
   \begin{itemize}
   \item  0x3 xon/xoff protocol and line editing
   \item  0x0 raw data transfer
   \end{itemize}
\end{description}
 
Note: If local edit is selected, the device may be used only in 
half duplex mode. The input buffer is needed.
All received characters are stored in the input buffer, which is 
realized as a non blocking ring buffer. If the buffer is full, a  '?'
(question mark) is displayed and the character is discarded. Backspace 
must be used to remove the last character.
On reception of carriage return (0x0d), the input buffer is returned to 
the application.

Note: The function \verb|printf| ends up in the function \verb|_write()|, which
is implemented in e.q. \verb|lpc17_uart_retarget.c|. The implementation
in \verb|lpc17_uart_retarget.c| works with simple polling on the line status
register. This behavior does not cooperate von \verb|Lpc17xxUart(0,...)|


\subsection{Hy32}
The LCD graphics display supports a resolution of 320x240 Pixel with
a color resolution of 5 bit (red and blue) 6 bit (green).

The display may be used in all 4 orientations. The device supports
text positioning and output with two fonts. The positioning is done with
a subset of vt100 esc-sequences.
The device supports no scrolling, since only vertical scroll is supported 
by hardware.

The following codes are supported. Note that CSI stand for
control sequence initiator, which is an ESC-character abd and opening square 
brace (\verb|"\033["| in C-notation):
\begin{description}
\item[{\em CSI}  {\em li};{\em col} H] 
    Set position. {\em li} and {\em col} are the line
    number and column number starting at 1 each. Both elements may contain
    1 or 2 digits (without leading space). Example \verb|\033[01;01H|
 is cursor home
\item[{\em CSI}  {\em n} J]  clear screen\\
n=0: fill screen from current cursor location to the end of the 
    screen with the current background color
\\
n=1: fill screen from current cursor location to the beginning of the 
    screen with the current background color
\\
n=2: fill entire screen with the current background color
\item[{\em CSI}  {\em n} m] {\em set Grafic Rendition}
    \begin{description}
    \item[n=10] select normal font (here: small font)
    \item[n=11] select alternate font (here: big font)
    \item[n=30-37] set foreground color according table below (n=30+index)
    \item[n=40-47] set background color according table below (n=40+index)
    \end{description}
\end{description}

Note that the calculation of the text position is done according the 
selected font(-size). If the font size is changed the new font geometry 
is used to calculate the pixel adress. The vertical spacing between two lines
is set to 2.

Note that clearing the display does no affect the cursor location, 
font selection and color setting.

\begin{tabular}{|l|l|c|}
\hline
index & color  & rgb \\
\hline
0 & black  & 0,0,0\\
1 & red  & 170,0,0\\
2 & green & 0, 170, 0 \\
3 & yellow & 255,255,0 ???\\
4 & blue & 0,0,170\\
5 & magenta & 170,0,170\\
6 & cyan & 0,170,170\\
7 &white & 255,255,255\\
\hline
\end{tabular}

Screen capacity:

\begin{tabular}{|l|l|l|l|}
\hline
Orientation & Font & Lines & Columns \\
\hline
 0 & small & 22 & 30  \\
   & big   & 14 & 15 \\
\hline
1 & small & 17 & 40 \\
  & big &   13 & 20 \\
\hline
 2 & small & 22 & 30  \\
   & big   & 14 & 15 \\
\hline
3 & small & 17 & 40 \\
  & big &   13 & 20 \\
\hline
\end{tabular}

\subsection{Device Summary}
Table \ref{lpc17_device_list} contains all avaliable device drivers 
with additional information for the system overview.

\begin{table}
\begin{tabular}{|l|r|r|}
\hline
Name             & Int\_Prio & Excepion Number(s) \\
\hline
Lpc17xxRTC       &  5 & 33 \\ 
Lpc17xxTimer0    &  5 &  17 \\
Lpc15xxInterrupt & 10 &  37 \\
Lpc15xxUart      & 10 &  21(UART0), 23(UART2) \\ 
\hline
\end{tabular}
\caption{Device Drivers with interrupt usage.}
\label{lpc17_device_list}
\end{table}

The interrupt execution is preemptive, with lower priority first and 
sequential within same priority starting with lower exception number first.

