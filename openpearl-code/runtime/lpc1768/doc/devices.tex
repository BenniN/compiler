\section{Devices}

\subsection{Lpc17xxDigitalIn}
The device Lpc17xxDigitalIn provides digital input bits on all
GPIO bits of the LPC1768. It is {\em NOT} checked, whether the 
required bits are in use by other devices.

The LPC1768 provides 5 GPIO ports, numbered from 0 to 4.
Each port has a maximum of 32 bits.
It is stated in the user manual, that a lot of bits are not
available.

Synopsis: \verb|Lpc17xxDigitalIn(port, start, width)|

\begin{description}
\item [port] specifies the port number. Valid numbers are 0,1,2,3,4
\item[start] specifies the number of leftmost bit of the group. Valid numbers
     are 31..0
\item [width] specifies the number of bits which are grouped together.
     Valid numbers are 1..32.
     The group must fit into one port.
\end{description}

The access to the dation is done by reading a BIT(width)-value from the
opened system dation.
Writing other data than BIT-types  may cause a signal, if the size of the
data does not match the required size according the width.

\subsection{Lpc17xxDigitalOut}
The device Lpc17xxDigitalOut provides digital output bits on all
GPIO bits of the LPC1768. It is {\em NOT} checked, whether the 
required bits are in use by other devices.

The LPC1768 provides 5 GPIO ports, numbered from 0 to 4.
Each port has a maximum of 32 bits.
It is stated in the user manual, that a lot of bits are not
available.

Synopsis: \verb|Lpc17xxDigitalOut(port, start, width)|

\begin{description}
\item [port] specifies the port number. Valid numbers are 0,1,2,3,4
\item[start] specifies the number of leftmost bit of the group. Valid numbers
     are 31..0
\item [width] specifies the number of bits which are grouped together.
     Valid numbers are 1..32.
     The group must fit into one port.
\end{description}

The access to the dation is done by writing a BIT(width)-value to the
opened system dation.
Writing other data than BIT-types  may cause a signal, if the size of the
data does not match the required size according the width.

\subsection{Lpc17xxInterrupt}
The devive Lpc17xxInterrupt provides interrupt sources from GPIO inputs.
The LPC1768 can produce interrupts on both edges of the GPIO ports 0 and 2.
The Lpc17xxInterrupt device produces interrupts only on the falling edge. 
Synopsis: \verb|Lpc17xxInterrupt(port, bit)|

\begin{description}
\item [port] specifies the port number. Valid numbers are 0,2
\item[bit] specifies the number of the  bit to monitor.
 Valid numbers are 31..0. 
\end{description}

Note: The LPC1768 does monitor only bits
0-11 and 15-30 on port 0 and 0-13 on port 2.

The landtiger board has Key1 on P2.11 and SW2 an P2.10.
For usage of SW2 the jumper JP5 must be closes (which is the
usual position if working with lpc21isp download).


