\section{Add a new SIGNAL}
The signals of PEARL are mapped to C++ exceptions in OpenPEARL.
For each PEARL signal a special class is required.
These classes are very similar. The creation of theses classes
is automated via an OpenOffice-spreadsheet and a perl-skript.

As long as no new signal group is required, the new signal
must be added to the spreadsheet \texttt{Signals.ods}.
For each signal the following elements must be specified, as 
data in the corresponding column in the spreadsheet:
\begin{description} 
\item[GroupIndex] is usually the next free index in this group.
   The value must be an integer $>= 0$
\item [ExceptionClass] is the name of the class to be generated.
   The name must be a valid C++ identifier. The text 'Signal' should
   be used as postfix.
\item[ParentClass] is used for signal grouping in the signal handlers
\item[System] use GENERIC, if the signal appears in all target
   platform of OpenPEARL, or select the concerned target platform.
\item[Message] is the message which is printed to the error log.
   This message should not exceed 40 characters.
\item[Description] is a TeX formatted text for the documentation
\end{description}

During the build process, the spreadsheet is translated into a C++ source
and  header file as well as LaTeX input for the documentation.
