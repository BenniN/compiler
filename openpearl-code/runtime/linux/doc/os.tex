\section{System Overview}

\subsection{Tasks}
The tasks are realized as static objects of type Task.
The tasks ctor adds itself into a list which contains all tasks.
This list is used for lookup reasons. 

\subsection{System Initialisation}
The PEARL application is started from the main, which is located in 
\verb|os.cc| ({\em Operation System}).

The system startup contains several steps:
\begin{description}
\item[read runtime configuration file] from the current directory or ---
   if not found in the user home directory.
   The configuration file is named \texttt{.pearlrc}.
   It is possible to modify the log level, maximum cpu time and number of 
   used cores. 
   For details see the description in the file (\ref{pearlrc}).
\item[setup CPU affinity] to the first CPU, regardless of the
   total number of available CPUs. This is done to get a more simple
   system behavior to verify the scheduling behavior.
\item[setup the task scheduler] to \verb|SCHED_FIFO| which provides
   the same behavior as PEARL requires. Tasks with the same priority
   are run with round-robin, tasks with differnt priority are run with
   preemptive priority scheduling.
   This operation needs root-priviledges. If the PEARL application is 
   started with normal user priviledges, this is detected and the 
   Linux {\em completely fair scheduler} is used.
\item[setup of run time limit] to 180 seconds. This is a fail safe
   operation to end the application automatically in case to run away
   high priority task. Pending on the Linux keernel configuration,
   it would be possible that no shell input like CTRL-C is will
   be done, when the complete CPU-power is consumed by a free running thread.
\item[start of a thread] to do all timer operations. This thread 
   receives the real-time signals for the scheduled tasking operations.

   The signal data
   structure deliver the task, which shall get the requested operation.
   \begin{description}
    \item[SIGRTMIN] for scheduled suspend (obsolete)
    \item[SIGRTMIN+1] for scheduled activations
    \item[SIGRTMIN+3] for scheduled continue
    \end{description}
\item[setup no more tasks handler] to end the PEARL application.
   In desktop environments, the application should return to the 
   command prompt if no more activity is possible.
   The {\em TaskMonitor} maintains a counter of pending and active 
   tasks. If this counter reaches $0$, the signal SIGRTMIN+4 is emitted.
   The corresponding signal handler performs a shutdown of the application.
\item[start of main-tasks] will activate all tasks with the attribute
     \verb|MAIN| in sequence according their priority.
\end{description}

\subsection{Runtime Configuration File (.pearlrc)}
\label{pearlrc}
\begin{lstlisting}
! -------------------------------------
! PEARL runtime configuration file (.pearlrc)
! this file is searched in the current directory (first)
! and the current users home directory (second)
!
! The following case sensitive parameters may be set:
!   LogLevel  (hex)
!     the parameter is a hexadecimal number which is bitwise encoded.
!     0x01: with debug-messages
!     0x02: with info-messages
!     0x04: with warning-messages
!     0x08: with error-messages
!     default: 0x0c (warn+error)
!     NOTE: enabling debug messages will involve the runtime behavior
!           dramatically
!
!  MaxCpuTime (dec)
!     define the maximum cpu time allowed for the application.
!     The time is given in seconds
!     The value (-1) defines eternal execution
!     Default value is 300
!
! Cores (dec)
!     defines the number of used cores for the application.
!     If less cores are available than specified in this command
!     the maximum of possible cores is used.
!     Default value: 1
!  
! -------------------------------------
!LogLevel    0x0c   (1=debug, 2=info, 4=warn, 8=error)
!MaxCpuTime 300
!Cores      1
\end{lstlisting}

\subsection{Command Line Interface}
After the system start the main-threads acts as system console.
The syntax is inspired by the Siemens ORG-300 operating system.
This works currently only if no task uses the StdStream(0) (stdin) device.


The following commands are realized:
\begin{description}
\item[/PRLI] lists all tasks and their detailed state.
\end{description}
