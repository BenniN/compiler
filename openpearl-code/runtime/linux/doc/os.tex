\section{System Overview}

PEARL tasks of a PEARL application are mapped the pthreads in a single
process.  The thread scheduling is done by the system scheduler of the 
linux system. If the apllication runs with normal user priviledges, the
completly fair scheduler (CFS) is used and PEARL priorities are not
regarded at task scheduling. If run with root priviledges, the 
\texttt{SCHED\_FIFO} scheduler is used, which provides the 
PEARL-specific preemptive priority scheduling with the lack of some 
priority levels. This can also be achieved by setting the allowed
real-time priorities in \verb|/etc/security/limits.conf|.

The timing functions are taken from the native itimer system.
The clock resolution is defined by the settings in the linux system.

\subsection{Tasks}
The tasks are realized as static objects of type Task.
The tasks ctor adds itself into a list which contains all tasks.
This list is used for lookup reasons. 
The macros \texttt{DCLTASK} and \texttt{SPCTASK} are provided in 
\texttt{Task.h}. They map the PEARL-tasks to the class \texttt{Task}.

\subsection{System Initialisation}
The PEARL application is started from the main, which is located in 
\verb|os.cc| ({\em Operation System}).

The system startup contains several steps:
\begin{description}
\item[read runtime configuration file] from the current directory or ---
   if not found in the user home directory.
   The configuration file is named \texttt{.pearlrc}.
   It is possible to modify the log level, maximum cpu time and number of 
   used cores. 
   For details see the description in the file (\ref{pearlrc}).
\item[setup CPU affinity] to the first CPU, regardless of the
   total number of available CPUs. This is done to get a more simple
   system behavior to verify the scheduling behavior.
\item[setup the task scheduler] to \verb|SCHED_FIFO| which provides
   the same behavior as PEARL requires. Tasks with the same priority
   are run with round-robin, tasks with differnt priority are run with
   preemptive priority scheduling.
   This operation needs root-priviledges. If \verb|SCHED_FIFO|
   is not available (e.q. run as normal user), this is detected and the 
   Linux {\em completely fair scheduler} is used.
\item[setup of run time limit] to 180 seconds. This is a fail safe
   operation to end the application automatically in case to run away
   high priority task. Pending on the Linux kernel configuration,
   it would be possible that no shell input like CTRL-C is will
   be done, when the complete CPU-power is consumed by a free running thread.
\item[start of a thread] to do all timer operations. This thread 
   receives the real-time signals for the scheduled tasking operations.

   The signal data
   structure deliver the task, which shall get the requested operation.
   \begin{description}
    \item[SIGRTMIN] for scheduled suspend (obsolete)
    \item[SIGRTMIN+1] for scheduled activations
    \item[SIGRTMIN+3] for scheduled continue
    \end{description}
\item[setup no more tasks handler] to end the PEARL application.
   In desktop environments, the application should return to the 
   command prompt if no more activity is possible.
   The {\em TaskMonitor} maintains a counter of pending and active 
   tasks. If this counter reaches $0$, the signal SIGRTMIN+4 is emitted.
   The corresponding signal handler performs a shutdown of the application.
\item[start of main-tasks] will activate all tasks with the attribute
     \verb|MAIN| in sequence according their priority.
\end{description}


