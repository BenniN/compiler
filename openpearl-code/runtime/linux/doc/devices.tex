\section{System devices}
\label{x8632devices}

System devices are defined by entries in the system part.
In case of problems during creation of the system device entry, the
usual exception mechanism does not work, since these objects are created
statically. Please check the log file for details.

\subsection{Disc}
The system device \verb|Disc| allows the access of files in the system.

Synopsis: \verb|Disc(path, nbr)|

\begin{description}
\item [path] specifies the system path of the directory which should contain
    the files defined by the {\em IDF} parameter in the {\em OPEN} statement
\item[nbr] specifies the number of concurrent opened files with this dation
     specified by {\em CREATED(...)} in the {\em OPEN} statement
\end{description}

Supported attributes: FORWARD, DIRECT, IN, OUT, INOUT

Multiple dations may by defined on the same path.

\subsection{StdStream}
The system device \verb|StdPath| allows the access of UNIX standard streams
like stdin, stdout and stderr.

Synopsis: \verb|StdStream(path)|

\begin{description}
\item [path] specifies the system stream with the usual values. 
   \begin{description}
    \item [0] denotes stdin
    \item [1] denotes stdout
    \item [2] denotes stderr
   \end{description}
\end{description}

Supported attributes: FORWARD, IN, OUT

Only one system device shall be defined per standard stream to avoid
intermixing of input and output.

\subsection{Pipe}
The system dation {\em Pipe} is useful for testing and interaction with
other applications. The dation provides access to named pipes.
The linux pipe blocks write attempts, if no reading file descriptor
is opend on the pipe. 
A dummy open with read may be achieved with the option OPEN1.

Synopsis: \verb|Pipe(path, nbr, params)|

\begin{description}
\item [path] is a C-string, which specifies the system path of the fifo file
    of the named pipe
\item[nbr] specifies the number (as int) of concurrent opened files with this dation
     specified by {\em CREATED(...)} in the {\em OPEN} statement
\item [params] is a string value for detailed control of the behavior
     at creation and termination. The parameters are searched in the string
     case sensitive. The sequence of the parameters is artificial.
   \begin{description}
     \item[OLD] the entry in the file system must exist before starting
     \item[NEW] the entry in the file system must not exist at starting
     \item[ANY] the entry may exist or will be created else; an existing
                file syste entry must by of type FIFO (default).
     \item[OPEN1] specifies that one file descriptor should be opened 
                automatically.
     \item[CAN] the file system entry shall be deleted at the end of the 
                application
     \item[PRM] the file system entry shall remain at the end of the
                application (default)
   \end{description} 
   The permissions are set to \verb|ugo=rwx| for new file system entries.
\end{description}

Supported attributes: FORWARD, IN, OUT

The declaration of multiple pipe devices upon the same system file shall
be avoided.
 
\subsection{OctopusDigitalOut}
The device OctopusDigitalOut provides an access to digital output bits
on the so called OCTOPUS-Board of Embedded Projects.
The Octopus board provides 6 ports with 8 bits each. Each bit may be
used as input or output.

Synopsis: \verb|OctopusDigitalOut(port, start, width)|

\begin{description}
\item [port] specifies the port letter. Valid characters are 'A', 'B', .. 'F'
\item[start] specifies the number of leftmost bit of the group. Valid numbers
     are 7,6,... 0
\item [width] specifies the number of bits which are grouped together.
     Valid numbers are 1..8.
     The group must fit into one port.
\end{description}

The access to the dation is done by writing a BIT(width)-value to the
opened system dation. 
Writing other data than BIT-types  may cause a signal, if the size of the
data is larger than 1 byte. If the size is 1 byte, unpredictable results may
occur.

It is checked that a i/o-bit is not used by different declarations within
one application.
  
\subsection{OctopusDigitalIn}
The device OctopusDigitalIn provides an access to digital input bits
on the so called OCTOPUS-Board of Embedded Projects.
The Octopus board provides 6 ports with 8 bits each. Each bit may be
used as input or output.

Synopsis: \verb|OctopusDigitalIn(port, start, width)|

\begin{description}
\item [port] specifies the port letter. Valid characters are 'A', 'B', .. 'F'
\item[start] specifies the number of left most bit of the group. Valid numbers
     are 7,6,... 0
\item [width] specifies the number of bits which are grouped together.
     Valid numbers are 1..8.
     The group must fit into one port.
\end{description}

The access to the dation is done by reading a BIT(width)-value to the
opened system dation. 
Reading other data than BIT-types  may cause a signal, if the size of the
data is larger than 1 byte. If the size is 1 byte, unpredictable results may
occur.

It is checked that a i/o-bit is not used by different declarations within
one application.
  
  
\subsection{Creating a new Device for User Dations}
The creation of a new device is very simple:
\begin{enumerate}
\item find a suitable device name
\item pass the parameters needed for construction of the device as
      parameters
\item add a target rule for the device in the Makefile
\item create a C++ class with the name of the device and add this class
      in the Makefile
\item provide {\em capabilities} in the ctor in that way that the allowed
      capabilities are listed
\item if more than on dation may be opened in a concurrent way on the new
      device, you should provide a pool mechanism like in Disc.
      If the new device allows only one opened dation at one time, you 
      should omit the pool like in Serio.
\item provide the dationOpen, dationClose, dationRead, dationWrite 
      and dationSeek methods. Check the actual given parameters on
      validity and create Log-entries and throw exceptions in case
      of bad parameters.
      Note that dationSeek is not required, if only FORWARD dations
      may be created.
\item create unit tests for google tests framework in tests/ and add the
      tests in tests/Makefile
\item run tests/DationTests and check for proper operation of the new device
\end{enumerate}

