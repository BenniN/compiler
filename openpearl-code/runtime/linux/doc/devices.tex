\section{System devices}
\label{x8632devices}

System devices are defined by entries in the system part.
In case of problems during creation of the system device entry, the
usual exception mechanism does not work, since these objects are created
statically. Please check the log file for details.

\subsection{Disc}
The system device \verb|Disc| allows the access of files in the system.

Synopsis: \verb|Disc(path, nbr)|

\begin{description}
\item [path] specifies the system path of the directory which should contain
    the files defined by the {\em IDF} parameter in the {\em OPEN} statement
\item[nbr] specifies the number of concurrent opened files with this dation
     specified by {\em CREATED(...)} in the {\em OPEN} statement
\end{description}

Supported attributes: FORWARD, DIRECT, IN, OUT, INOUT

Multiple dations may by defined on the same path.

\subsection{Standard Streams: StdIn, StdOut, StdError}
The system devices \verb|StdIn|,
\verb|StdOut| an \verb|StdError| allows the access of UNIX standard streams
like stdin, stdout and stderr.

\begin{tabular}{ll}
Synopsis: & \verb|StdIn| \\
          & \verb|StdOut| \\
          & \verb|StdError| \\
\end{tabular}

Supported attributes: FORWARD, IN, OUT --- IN and OUT depend on the
particular stream.

Only one system device shall be defined per standard stream to avoid
intermixing of input and output.

\subsection{Pipe}
The system dation {\em Pipe} is useful for testing and interaction with
other applications. The dation provides access to named pipes.
The linux pipe blocks write attempts, if no reading file descriptor
is opend on the pipe. 
A dummy open with read may be achieved with the option OPEN1.

Synopsis: \verb|Pipe(path, nbr, params)|

\begin{description}
\item [path] is a C-string, which specifies the system path of the fifo file
    of the named pipe
\item[nbr] specifies the number (as int) of concurrent opened files with this dation
     specified by {\em CREATED(...)} in the {\em OPEN} statement
\item [params] is a string value for detailed control of the behavior
     at creation and termination. The parameters are searched in the string
     case sensitive. The sequence of the parameters is artificial.
   \begin{description}
     \item[OLD] the entry in the file system must exist before starting
     \item[NEW] the entry in the file system must not exist at starting
     \item[ANY] the entry may exist or will be created else; an existing
                file syste entry must by of type FIFO (default).
     \item[OPEN1] specifies that one file descriptor should be opened 
                automatically.
     \item[CAN] the file system entry shall be deleted at the end of the 
                application
     \item[PRM] the file system entry shall remain at the end of the
                application (default)
   \end{description} 
   The permissions are set to \verb|ugo=rwx| for new file system entries.
\end{description}

Supported attributes: FORWARD, IN, OUT

The declaration of multiple pipe devices upon the same system file shall
be avoided.
 
\subsection{OctopusDigitalOut, OctopusDigitalIn}
The device OctopusDigitalOut provides an access to digital input and 
output bits
on the so called OCTOPUS-Board of Embedded Projects.
The Octopus board provides 6 ports with 8 bits each. Each bit may be
used as input or output.

\begin{tabular}{ll}
Synopsis:& \verb|OctopusDigitalOut(port, start, width)|\\
        & \verb|OctopusDigitalIn(port, start, width)|\\
\end{tabular}

\begin{description}
\item [port] specifies the port letter. Valid characters are 'A', 'B', .. 'F'
\item[start] specifies the number of leftmost bit of the group. Valid numbers
     are 7,6,... 0
\item [width] specifies the number of bits which are grouped together.
     Valid numbers are 1..8.
     The group must fit into one port.
\end{description}

The access to the dation is done by reading or writing a BIT(width)-value to the
opened system dation. 
Writing other data than BIT-types  may cause a signal, if the size of the
data is larger than 1 byte. If the size is 1 byte, unpredictable results may
occur.

It is checked that a i/o-bit is not used by different declarations within
one application.
  
\subsection{RPiDigitalOut, RPiDigitalIn}
The device RPiDigitalOut provides an access to digital input and output bits
on the GPIO-bits of the raspberry pi.

\begin{tabular}{ll}
Synopsis: & \verb|RPiDigitalOut(start, width)| \\
          & \verb|RPiDigitalIn(start, width, pud)| \\
\end{tabular}

\begin{description}
\item[start] specifies the number of leftmost bit of the group. Valid numbers
     depend on the model of the raspberry pi. The starting number if
     the leftmost gpio bit number (large value)
\item [width] specifies the number of bits which are grouped together.
     Valid numbers depend on the model of the raspberry pi.
\item [pud] Pull-up / pull-down configuration.
    Valid values are 'u' for pull up, 'd' for pull down and 
    '' for no pull-up/pull-down resistor. This option is only
    available for digital inputs.
\end{description}

The access to the dation is done by writing a BIT(width)-value to the
opened system dation. 
Writing other data than BIT-types  may cause a signal, if the size of the
data does not match the width parameter. 

It is checked that a i/o-bit is not used by different declarations within
one application.

Supported bits:

\begin{tabular}{|l|l|}
\hline
Model & Bits \\
\hline
1B & 2,3,4,7,8,9,10,11,17,18,22,23,24,25,27,28,29,30,31 \\
\hline
2B,3B & 2-27 (except 14 and 15) \\
\hline
\end{tabular}  

\paragraph{Note:} 
\begin{itemize}
\item the Broadcom gpio block supports the access to the bits only 
via set and clear registers. Thus the writing of a BIT(x) value with x $>$ 1
need two sequential operations to clear and set the output bits, which lead
to a short period of an intermediate value. E.g. '10' shall be changed to '01'. Bit 1 will be cleared leading to the output value '00' and bit 2 
is set afterwards leading to '01'.

\item Bits 14 and 15 are used by the kernel for the 
  system console during system start.
\item Other bits become not available, if i2c- or spi-dations are used
  in the PEARL application.  If the i2c and/or spi is enabled in the
  raspberry configuration but not used in the PEARL application,
  the usage of the corresponding bits 
  as gpios may cause unpredictable results.
\item the current user needs priviledges to access the system device 
  \verb|/dev/gpiomem|.
\end{itemize}
 
\subsection{I2CBus}
Linux supports normal access to an i2c bus via a device node,
 if i2c is available. 
The corresponding device file are named like \texttt{/dev/i2c-0}.
The i2c subsystem of linux provides test features for the availability
to the so called {\em repeated start} feature (see I2C\_FUNCS\_I2C)

\begin{tabular}{ll}
Synopsis: & \verb|I2CBus(deviceName)|\\ 
\end{tabular}

\begin{description}
\item[deviceName] specifies the name to the device file (e.g. "/dev/i2c-0")
\end{description}

\paragraph{Note:}
The I2C bus speed must be set at system level depending on the 
i2c device module.


\subsection{PEAK CAN Bus Interface}
The CAN bus interface from the company PEAK provides the access via a character 
device node (e.q. \texttt{/dev/pcan32} and via SocketCAN.
The socket CAN infrastructure allows no change of the communicatipon setting
like bitrate for normal users --- the character device from PEAK provides
such features.

After installation of the PCAN device driver on the system as character 
device, the corresponding device nodes and libraries are installed 
in the system (e.g. \verb|/dev/pcan32|).


\begin{tabular}{ll}
Synopsis: & \verb|PCan(deviceName, speed)|\\ 
\end{tabular}

\begin{description}
\item[deviceName] specifies the name to the device file (e.g. "/dev/pcan32")
\item[bus speed] specifies the bus speed. 
Only the values 125k, 250k, 500k and 1M bps are supported by 
the OpenPEARL driver.
\end{description}

The transmission data must have the following structure:

\begin{verbatim}
   class Can2AMessage {

   public:
     Fixed<11> identifier;
     BitString<1> rtrRequest;
     Fixed<4> dataLength;
     BitString<8> data[8];

   };
\end{verbatim}




\subsection{Creating a new Device}
The creation of a new device is very simple:
\begin{enumerate}
\item find a suitable device name
\item pass the parameters needed for construction of the device as
      parameters
\item add a target rule for the device in the Makefile
\item create a C++ class with the name of the device and add this class
      in the Makefile
\item derive the C++ class from SystemDationB or SystemDationNB depending
      on the dation type. SystemDationB is the parent for process dations.
      SystemDationNB is parent for dations for formatted and unformatted
      i/o with PUT,GET,READ and WRITE
\item provide {\em capabilities} in the ctor in that way that the allowed
      capabilities are listed
\item if more than on dation may be opened in a concurrent way on the new
      device, you should provide a pool mechanisme like in Disc.
      If the new device allows only one opened dation at one time, you 
      should omit the pool.
\item provide the dationOpen, dationClose, dationRead, dationWrite 
      and dationSeek methods. Check the actual given parameters on
      validity and create Log-entries and throw exceptions in case
      of bad parameters.
      Note that dationSeek is not required, if only FORWARD dations
      may be created.
\item create a suitable XML description file.
\item insert a configuration item  
      in the system configuration menu
      and set the default to {\em not available} ('n').
\item modify the corresponding Makefile toi add the device depending on
      the configuration item. If additional libraries are needed,
      add them to \texttt{LDPATH}. 
\item create unit tests for google tests framework in tests/ and add the
      tests in tests/Makefile
\item run the tests and check for proper operation of the new device
\end{enumerate}

