\section{Checklist: How to create a new System Device}
In linux the system devices are represented by a device driver of
the linux kernel.
It is possible to access the devices via the normal interface
with {\em open}, {\em close}, {\em read}, {\em write} and {\em ioctl}.
The device node name in the folder \verb|/dev| must have access
priviledges for {\em users}.

These functions should be used in the methods of the C++ class 
for the device driver.


The creation of a new device is very simple:
\begin{enumerate}
\item find a suitable device name
\item pass the parameters needed for construction of the device as
      parameters
\item add a target rule for the device in the Makefile
\item create a C++ class with the name of the device and add this class
      in the Makefile
\item derive the C++ class from SystemDationB or SystemDationNB depending
      on the dation type. SystemDationB is the parent for process dations.
      SystemDationNB is parent for dations for formatted and unformatted
      i/o with PUT,GET,READ and WRITE
\item provide {\em capabilities} in the ctor in that way that the allowed
      capabilities are listed
\item if more than on dation may be opened in a concurrent way on the new
      device, you should provide a pool mechanisme like in Disc.
      If the new device allows only one opened dation at one time, you 
      should omit the pool.
\item provide the dationOpen, dationClose, dationRead, dationWrite 
      and dationSeek methods. Check the actual given parameters on
      validity and create Log-entries and throw exceptions in case
      of bad parameters.
      Note that dationSeek is not required, if only FORWARD dations
      may be created.
\item create a suitable XML description file, which describes the
      device characteristics.
\item If the device depends on special libraries or 
      system setup, insert a configuration item  
      in the system configuration menu
      and set the default to {\em not available} ('n').
\item modify the corresponding Makefile and add the device depending on
      the configuration item. If additional libraries are needed,
      add them to \texttt{LDPATH} variable. 
\item create unit tests for google tests framework in tests/ and add the
      tests in tests/Makefile
\item run the tests and check for proper operation of the new device
\item add the documentation of the new device to this document
\end{enumerate}

