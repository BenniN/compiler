\documentclass[10pt]{scrbook}
\usepackage{geometry}
\usepackage{graphicx}
\usepackage{import}
\usepackage{xcolor}
\usepackage{listings}
\usepackage{caption}
\usepackage{hyperref}
\usepackage{tikz}
\usetikzlibrary{calc,positioning,shapes.geometric}

\geometry{a4paper}
\setlength{\parindent}{0pt}

%\DeclareCaptionFont{white}{\color{white}}
%DeclareCaptionFormat{listing}{\colorbox[cmyk]{0.43, 0.35, 0.35,0.01}
\DeclareCaptionFormat{listing}{\colorbox{black!15}
  {\parbox{5cm}{#1#2#3}}}
\captionsetup[lstlisting]{format=listing,
%labelfont=white,textfont=white, 
singlelinecheck=false, margin=0pt, font={bf,footnotesize}}

\begin{document}
\title{OpenPEARL - Platform Manual for Linux Systems}
\author{R. M\"uller}

\lstnewenvironment{CppCode} {
    \lstset{numbers=left,
            title={C++},
            frame=tlrb,
	    breaklines = true,
	    belowcaptionskip=4pt
    }
}%
{}%

\lstnewenvironment{PEARLCode} {
    \lstset{numbers=left,
            title={PEARL},
            frame=tlrb,
	    breaklines = true,
	    belowcaptionskip=4pt
    }
}%
{}%

\pagestyle{myheadings}
\markboth {OpenPEARL Platform Manual Linux}
          {OpenPEARL Platform Manual Linux}

\maketitle

\tableofcontents

%\chapter{Log}
The logging facility itself is platform specific since the
default log interface depends on the target system.. The interface is 
defined platform independently in {\em Log.h}.

The platform independed formatting is located in \verb|common/Log.cc|
and tghe platform, specific ctor ist located in the platform specific
directory as \verb|Log.cc|\footnote{The use of the same file name 
should be resolved in the future!}. 

The following log levels are defined:
\begin{description}
\item[INFO] general information about the program execution
\item[DEBUG] messages used for debugging the runtime system. Note that
   many messages may affect the application execution.
\item[WARN] messages aabout situations, where a backup solution is used.
E.g. start of the application without root priviledes in Linux will 
forbid the usage of the priority scheduler. The normal scheduler will be 
used instead.
\item[ERROR] are diagnostic information is case of signal raising.
\end{description}

The log interface defines the methods
\verb|Log::info()|, \verb|Log::debug()|, Log::warn()| and \verb|Log::error()|.
Each method takes at least on string argument like printf.
Plain text is passed to the corresponding output.
The following format elements are in use: \verb|%d|, \verb|%u|, \verb|%s|,
\verb|%f| and \verb|%c|. No additional parameters are allowed with these 
formatting options.

The output device is platform specific.

%\chapter{Interrupt}
An interrupt is an asynchrous event from the outside world.
It is {\bf NOT} the hardware interrupt of the processor.

\section{Specification and Declaration}
An interrupt is declared in the system part.

Example: Declare the identifier \verb|ctrlc| as the plattform specific
interrupt source \verb|^C|

\begin{verbatim}
ctrlc: UnixSignal(2);
\end{verbatim}

The specification takes place in the problem part:

\begin{verbatim}
SPC ctrlc INTERRUPT;
\end{verbatim}

A specific plattform may support several different interrupt sources.
All concrete interrupt sources must be derived from the parent class
\verb|pearlrt::Interrupt|.


\section{C++ Mapping between System and Problem Part}
The user defined identifier in the system part denotes a specifice interrupt
source. The identifier in the problem part denotes a generalized interrupt.
The compiler should generate a pointer to the generalized interrupt object like
shown in the example below:

\begin{PEARLCode}
SYSTEM;
  ctrlc: UnixSignal(2);
PROBLEM;
   SPC ctrlc INTERRUPT;
\end{PEARLCode}

The user supplied identifier (\verb|ctrlc|) is used as base of derived
identifiers.
\begin{description}
\item[sys\_] prefix denotes the real defined interrupt object 
\item[\_] prefix (as usual for all user supplied idetifiers) denotes the
    pointer to the generalized interrupt object.
\end{description}

\begin{CppCode}
// SYSTEM;
static pearlrt::UnixSignal sys_ctrlc(2);
       pearlrt::Interrupt * _ctrlc = (pearlrt::Interrupt*)&sys_ctrlc;
// PROBLEM;
extern pearlrt::Interrupt * _ctrlc;
\end{CppCode}

\section{Method Mapping}
The interrupt class provides methods for the PEARL language statements
working directly on interrupts.

The translation from PEARL to C++ is generic. 
The interrupt identifier is a pointer to the generalized interrupt object.

\begin{methodMapping}
\verb|ENABLE|  & \verb|enable()| & \\
\verb|DISABLE|  & \verb|disable()|&  \\
\verb|TRIGGER|  & \verb|trigger()| & \\
\end{methodMapping}

Example:

\begin{PEARLCode}
! ctrlc is specified as INTERRUPT;
...
ENABLE ctrlc;
DISABLE ctrlc;
TRIGGER ctrlc;
\end{PEARLCode}

Should translate into:
\begin{CppCode}
_ctrlc->enable();
_ctrlc->disable();
_ctrlc->trigger();
\end{CppCode}

\section{Usage in Task Scheduling}
See chapter \ref{task}

\section{Software Interrupts}
For all plattforms the \verb|SoftInt| interrupt source is provided.
There is no external connection to this interrupt type.

Synopsis: \verb|SoftInt(nbr)|

The parameter \verb|nbr| denotes the software interrupt number. 
The values from 1 to 32 are allowed. It is forbidden to define more
than one user identifier for the same number.
In case of attempting more than one user defined interrupt for 
a UNIX signal the InternalDataTypeSignal is induced 
during system initializing phase.

\section{How to Create an Interrupt Source}
Interrupt sources are always plattform specific.
In order to create such an interrupt source follow the instructions below:
\begin{enumerate}
\item choose a good name for the interrupt source
\item define the parameters if required
\item create a class with the same name as the your interrupt 
  \begin{enumerate}
  \item  derive your class from Interrupt  
  \item supply a constructor which fits to the required parameters
  \item make shure that there is may only one object become created
        by the user
  \item do all required operations, which are required to gather 
        the interrupt afterwards (register in the system, create 
        worker thread, ...)
  \end{enumerate}
\item make shure to call the method \verb|trigger()| from the parent
      class each time the interrupt in detected.
\item add the documentation of your interrupt in the plattform specific part
      of the OpenPEARL documentation
\end{enumerate}


%\chapter{Devices and I/O}

PEARL distinguishes between system dations and user dations.
All examples in the language report shown that the statements TAKE and SEND
operate directly on system dations like a digital i/o.
The operations READ, WRITE, GET and PUT work only on user dations.
User dations are created upon a system dation.

System part and problem part may be compiled separatelly. Therefore 
no information from the system part may be used to compile the problem part.

To avoid matching problems, the compiler produces an XML-file for each 
module with the import and export interface of the module.
The tool IMC (inter module checker) verifies the matching and creates the
C++ code for the system part.


\section{Not Supported Language Elements}
\begin{description}
%\item[TFU] should be ignored by the compiler
%\item[TFU(MAX)] should be ignored by the compiler
\item[FORBACK] is not supported yet, since tape drives are
   difficult to find
\item[R()] remote format must be treated by the compiler
\end{description}

\section{Declarations in System Part}
The compiler checks the elements for proper grammar and passaes the
parsed elements into an XML-file with the same name as the source file.

To provide complete code examples the linux target is used here.
For details of the used system devices please check chapter \ref{x8632devices}.

\begin{PEARLCode}
MODULE (m1);
SYSTEM;
   output: StdStream(1);
PROBLEM;
MODEND;
\end{PEARLCode}

The IMC will respond with a code like:

\begin{CppCode}
static pearlrt::StdStream s_output(1);
       pearlrt::Device * d_output = &s_output;
\end{CppCode};

All devices are derived from {\em Device}.
Devices for basic dations are derived from {\em SystemDationB},
all other devices are derived from {\em SystemDationNB}.
The upcast of the pointer
to the generic {\em Device*} type enables the compiler to generated 
suitable code for the problem part.

The concrete device object is set to {\em static} to avoid namespace polution.
Only the pointer is needed.

\section{Decoration Scheme for Devices}
We need different obejcts, which concern the same user object.
The user objects are prependend with an underscore (\verb|_|).

\begin{description}
\item[d\_...] denotes the upcasted to {\em Device} 
\item[s\_...] denotes the (static defined) real system device object
\item[\_...] without other prefix)denotes the downcasted object in 
     the C++ code of the problem part
\item[dim\_...] denotes the dimension object, which is relate to the 
   user dation
\end{description}

\section{Bus Device Associations}
The update of the language definition introduced {\em BusDeviceAssociations},
which allows to interconnect system part elements.

E.g.
\begin{PEARLCode}
MODULE (m1);
SYSTEM;
   i2c: I2CBus('/dev/i2c-0',100000);  ! access to i2-c
                                      ! interface with 
                                      ! 100kHz bus speed
   temp: LM75('48'B4) --- i2c;        ! one temperatur
                                      ! sensor at
                                      ! the bus i2c
PROBLEM;
   SPC temp DATION SYSTEM BASIC FIXED(15);
MODEND;
\end{PEARLCode}

\begin{CppCode}
// system part only
static pearlrt::I2CBus s_i2c("/dev/i2c-0",100000);

static pearlrt::LM75 s_temp(&s_i2c,0x48);
       pearlrt::Device * d_temp = &s_temp;
\end{CppCode}

Depending to the characteristics of the bus there may be only one client,
or multiple clients. This is specified in the XML-description of the 
device.

The implementation provides two roles:
\begin{itemize}
\item ConnectionProvider, which represents the bus device
\item ConnectionClient, which represents the component on the bus.
\end{itemize}

The classes for clients are derived from \texttt{SystemDationB}.
The classes for providers are be derived from a class which
defines the interface for the connection. 

E.g.:
\begin{description} 
\item[ \texttt{I2CProvider} ] is common to all plattforms
   and defines the communication interface between provider and client
\item[ \texttt{I2CBus}] implements the access to the i2c-bus on a 
   specific plattform
\item[ \texttt{LM75} ] defines a system dation, which is attached to an i2c-bus.
    This element is plattform independent.
\end{description}

\paragraph{Note:} The client has access to the provider by default, since
the provider is passed as first argument in the ctor of the client.
If the provider needs access to the client, the ctor of the client
should register itself at the provider.


\paragraph{Note:} The second parameter in I2CBus is used in microcontroller
   environmants. The linux systems define the transmission speed at
   another location.

\section{Specifications in Problem Part}
The system device is specified in the problem part.
Depending on the class specifier in the {\em SPC}-statement the concete 
dation type is deduced (see tab. \ref{dationTypes}).

\begin{table}[bpht]
\begin{tabular}{l|l|l|l}
  & BASIC & ALPHIC & ALL / type \\
\hline
SYSTEM & SystemDationB & SystemDationNB & SystemDationNB \\
 & DationTS & DationPG & DationRW \\
\hline
\end{tabular}
\caption{Deduced type of the dation object}
\label{dationTypes}
\end{table}


\begin{PEARLCode}
SPC output DATION OUT SYSTEM ALPHIC GLOBAL;
SPC disc DATION OUT SYSTEM ALL GLOBAL;
SPC console DATION OUT ALPHIC GLOBAL;
SPC logbook DATION OUT Fixed(15) GLOBAL;
SPC mot DATION OUT SYSTEM BASIC BIT(4) GLOBAL;
\end{PEARLCode}

\begin{CppCode}
extern pearlrt::Device * d_output;
static pearlrt::SystemDationNB _output* = 
             static_cast<pearlrt::SystemDationNB*>(d_output);

extern pearlrt::Device * d_disc;
static pearlrt::SystemDationNB _disc* = 
             static_cast<pearlrt::SystemDationNB*>(d_disc);

extern pearlrt::DationPG _console;

extern pearlrt::DationRW _logbook;

extern pearlrt::Device * d_mot;
static pearlrt::SystemDationB * _mot = 
             static_cast<pearlrt::SystemDationB*>(d_mot);
\end{CppCode}

\section{Declaration of a User Dation}
The declaration of user dation may appear inside and outside of procedures and 
tasks.
The declaration leads to an instantiation of an object. 
The class attribute of the user dation decide about the type of the object
(see tab. \ref{dationTypes}).

The different types of possible dimension specifications are mapped to
a class hierarchy for {\em DationDim}s. For details see the doxygen part 
of the runtime documentation.

\begin{PEARLCode}
DCL console DATION OUT ALPHIC DIM(*,80) FORWARD STREAM NOCYCL CREATED(output);
DCL file DATION OUT FIXED(15) DIM(*,80) FORWARD STREAM NOCYCL CREATED(output);
DCL file1 DATION OUT ALL DIM(*) FORWARD STREAM NOCYCL CREATED(output);
DCL motor DATION OUT BASIC BIT(4) CREATED(mot);
\end{PEARLCode}

\begin{CppCode} 
// 2-dimensions, first dimension is unbound
static pearlrt::DationDim2 dim_console(80);
static pearlrt::DationPG _console(_output, 
                pearlrt::Dation::OUT      |
                pearlrt::Dation::FORWARD  | 
                pearlrt::Dation::STREAM   |
                pearlrt::Dation::NOCYCL,
                &dim_console);

static pearlrt::DationDim2 dim_file(80);
static pearlrt::DationPG _file(_output, 
                pearlrt::Dation::OUT      |
                pearlrt::Dation::FORWARD  | 
                pearlrt::Dation::STREAM   |
                pearlrt::Dation::NOCYCL,
                &dim_file, sizeof(pearlrt::Fixed<15>));

static pearlrt::DationDim1 dim_file1();
static pearlrt::DationPG _file1(_output, 
                pearlrt::Dation::OUT      |
                pearlrt::Dation::FORWARD  | 
                pearlrt::Dation::STREAM   |
                pearlrt::Dation::NOCYCL,
                &dim_file, 1);

static pearlrt::DationTS _motor(_mot,
                pearlrt::Dation::OUT);
\end{CppCode}

Remarks:
\begin{itemize}
\item the dation parameters are \verb|or|ed together
\item the dimenion object is defined statically.
\item the {\em CREATED()} parameter is passsed as first parameter
\item if the dation is defined as {\em GLOBAL} the \verb|static| must
  be omitted
\item a user dation with transfer type ALL must be 1-dimensional --- may
be bounded --- and the size of a data element must be specified as 1
\item the userdations of type BASIC do not allow Topology- and 
AccessAttribut-specifications 
\end{itemize}

\subsection{DIRECT Dation}
DIRECT dations are implemented on e.g. disc files
\begin{itemize}
\item    as artificial array like structure
\item    the DIM-specifies the structure
\item    absolute positioning is allowed
\item    relative positioning is mapped on the absolute positioning
\item    STREAM allows the positioning across dimension limits
\item    STREAM allows silent reading and writing across dimension limits
 \item   CYCLIC realizes a cyclic positioning modulo the complete
        dimension in both directions
\item    CYCLIC realizes a cyclic writing in that way that a
          repositioning to the beginning of the dation is inserted
          at the end of the dation, if the system device does not support
          CYCLIC operation with the same dimension
%\item    unwritten locations in the dations contain undefined values
\item    CYCLIC on unbounded dations is ridiculous
\end{itemize}

\subsection{FORWARD Dation}
FORWARD and CYCLIC is ridiculous if the system dation is not CYCLIC,
 since FORWARD can not rewind.

FORWARD dations are typically used on devices like stdin, 
stdout or TCP/IP-sockets.

\begin{description}
\item[PUT/GET] dations are created on e.g. stdout. 
   \begin{itemize}
   \item PUT ... BY X, SKIP and PAGE write space, new line or formfeeds
         as required by format
   \item  GET ... BY X, SKIP and PAGE discard input characters until the
         required number of characters, newlines or formfeeds were detected
   \item NOSTREAM causes 
        the specified dimension is used for error detection. 
         GET/PUT across boundary cause an error condition
   \item STREAM:
         number of dimensions control the possibility for X,SKIP and PAGE
         no boundary enforced
   \end{itemize}

\item[READ/WRITE] positioning:
   \begin{itemize}
     \item WRITE ... BY  X,SKIP and PAGE fills zero-records as required 
           by the current location
     \item READ ... BY  X,SKIP and PAGE discards input until 
           the required location is reached.
   \end{itemize}
\item[READ/WRITE + NOSTREAM] dations are implemented on e.g.
        TCP/IP socket or pipes
   \begin{itemize}
    \item The specified dimension is used for error detection
    \end{itemize}
\item[ READ/WRITE + STREAM] will  
        normalize the current location to be within the specified dimensions
        by calculating any out of bounds position by simulating line
        and page wraps. Eg:
\begin{verbatim}
DIM( 10,80)
! current location: (5,79)
WRITE a,b,c TO ...; --> new logical location (5, 81)
WRITE d TO .. BY SKIP;
\end{verbatim}
    \begin{itemize}
     \item calculate virtual normalized location: (6,1)
     \item SKIP fills 79 records with 0; new location (7,1)
     \item d will be written on location (7,1); new location is  (7,2)
     \end{itemize}
\end{description}

\section{Concurrency during I/O}
\label{ioconcurrency}

\subsection{Concurrency on the same user dation}
It is never stated but clearly appreciated that i/o-statements from
different tasks should not intermix within their i/o-operations.
This behavior is achieved by the {\em beginSequence} and 
{\em endSequence} clause. 
This clause locks mutex of the dation object to block concurrent
i/o-statments upon the same dation.
The task's object reference (me-pointer) is passed with
the {\em beginSequence} call.

In case of abnormal termination due to signals in the try-block
the {\em endSequence} in the catch-clause releases the used mutex.
If a RST-value is specified in the statement the given variable is
updated, else the exception is rethrown.

Task suspending and termination within the try-block would cause an
eternal blocking of the dation. 
This behavior is not desired.
Thus the suspend and terminate will not be done only at the end of a record
({\em SKIP}) or after the completion of the i/o-statement.
This is realized internally in the runtime system.

\begin{PEARLCode}
DCL (x,y) Fixed(15);
DCL rst Fixed(31);
...
PUT x,y TO console BY F(4), RST(rst), F(5), SKIP;
\end{PEARLCode}

\begin{CppCode}
// code scheme. With the introcuction of the IOJobs,
// the blocking was moved into the userdate operations put(), get(), ...
//
Fixed<31> rst;
Fixed<15> x
Fixed<15> y;

try {
   _console.beginSequence(me);
   _console.toF(x,(pearlrt::Fixed<31>)4);
   _console.rst(rst);
   _console.toF(y,(pearlrt::Fixed<31>)5);
   _console.toSkip((pearlrt::Fixed<31>)1);
   _console.endSequence();
} catch (Signal& s) {
   if (!_console.updateRst(&s)) {
      _console.endSequence();
      throw;
   }
   _console.endSequence();
}
\end{CppCode}

The try-catch-clause captures all PEARL-signals from the code in the 
type block. If a RST-value is specified in the format list, the method
{\em updateRst} will assign the actual signal code to the RST-variable.
If no RST-vale is specfied for this statement, the signal is propagated.
If the error is produced in the formatting of x, a signal is thrown.
If the error is produced in the formatting of y, no signal is thrown but rst
is set.

If the system dation supports multipe io-operations in the same time,
the locking mechanism of the user dation can be switched off, by implementing
a method \verb|bool allowMultipleIOOperations()| in the system dation.
In this case, ths system dation is responsible for proper operation.

\subsection{Concurrency on the same system dation}
It is possible to create several user dations upon the same system dation.
If the operation on the sstem dation is not reentrant, the developer must
enshure by additional mutex variables, that those operations do not operlap.

\section{Open and Close}
The super class UserDation knows the OPEN and CLOSE operations.
Both have optional parameters. This is solved by a \verb|or|ed
parameter.

\begin{PEARLCode}
DCL f FIXED(31);
...
OPEN console;
...
OPEN logbook BY RST(f), IDF('file.001') ANY;
...
CLOSE console;
...
CLOSE logbook BY CAN;
\end{PEARLCode}

\begin{CppCode}
Fixed<15> f;
...
_console.dationOpen();
...
{
   static Character<8> fileName("file.001");
   _logbook.dationOpen(Dation::IDF|Dation::ANY| Dation::RST, &fileName ,& f);
}
...
_console.dationClose();
...
_logbook.dationClose(Dation::CAN);
\end{CppCode}

Remarks:
\begin{itemize}
\item The presence of a rst-value is signaled by the \verb|or|ed parameter
   in dationOpen and dationClose
\item The filename must be passed by reference. The corresponding
   object must be defined. Eg. as \verb|static| is a new block.
\item The runtime system checks whether the dation parameters from
   user dation match the system dation's parameters. 
   If they do not match a PEARL signal is raised
\end{itemize}

\section{IO-operations}
The IO-operations are treated by methods  of the user dations.
These methods receive a list of data elements and a list of format
elements. These lists will be treated according the semantics of the dation
type. ALPHIC dation are controlled by the data list and will use
the format list cyclic, if more data than formats are in the list.
Non basic dations with a specific data type will treat all positioning
elements first followed by the treatment of the data elements. 
The locking mechanisme \ref{ioconcurrency} is realized in these methods.

For details see the source code documentation of the classes
\begin{description}
\item[IODataEntry] the individual data entries
\item[IODataList] the list of all data entries for one statement
\item[IOFormatEntry] the individual format entries
\item[IOFormatList] the list of all format entries for one statement
\end{description}

All parameters of format elements are expected as type
\verb|pearlrt::Fixed<31>|.

The effective code will be in this case:
\begin{CppCode}
Fixed<31> rst;
Fixed<15> x
Fixed<15> y;

{
  // setup io data and format lists in a new block
  // define variables for expression results in data list and format list
  ...
  IODataEntry dataEntries[] = { {...},{...},...};
  IODataList dataList = {sizeof( dataEntries)/sizeof(dataEntries[0]),
                         dataEntries};
  IOFormatEntry formatEntries[] = { {...},{...},...};
  IOFormatList formatList = {sizeof( formatEntries)/sizeof(formatEntries[0]),
                         formatEntries};
  // evaluate expressions in datalist and format list
  _console.put(me, dataList, formatList);
  // leave block with put statement
}
\end{CppCode}



\chapter{Installation}
The installation must be done via the git repository of the OpenPEARL project.
There are several software packages required on your linux PC. 
The repository contains an install script (\verb|installPackages|),
which identifies your
linux distribution by content of the file \verb|/etc/os-release|.

For details of the additional packages study the installation script.

\paragraph{Prerequisites}
\begin{itemize}
\item you have access to your linux system as user and root
\item you have an internet connection
\item you have a git client installed
\end{itemize}

\paragraph{Perform the following steps:}
\begin{enumerate}
\item login as normal user
\item create a working directory for the installation. \\
   e.g. /home/userx/OpenPEARL\\
    \verb|mkdir /home/userx/OpenPEARL|
\item set you current working directory to this point \\
    \verb|cd /home/userx/OpenPEARL|
\item obtain a read only copy the repository\\
    \verb|git clone git://git.code.sf.net/p/openpearl/code .|\\
    regard the point at the end of the command
\item set the working directory to the top of the real content
     in your working copy 
     (we are sorry about this unnecessary directory level): \\
    \verb|cd openpearl-code|
\item get administration priviledges\\
    \verb|sudo| or \verb|su|
\item run the script:\\
    \verb|./installPackages|\\
    and wait for completion of the installation.
    This takes about 6 minutes at an RaspberryPi 3B
\item still as administrator, execute\\
    \verb|make defconfig|\\
    \verb|make prepare|\\
    \verb|make install|\\
    this takes about 20 minutes at a Rspberry Pi 3B
\end{enumerate}

\paragraph{Verify Installation:}
Work as normal user.


\begin{enumerate}
\item \verb|cd /home/userx/OpenPEARL/openpearl-code/demos|
\item build some of the provided demonstration programs\\
   \verb|prl <demo>.prl|\\
   where \verb|<demo>| is Hello or sched\_demo
\item run the demonstration programm with \\
   \verb|./Hello| or \verb|./sched_demo|.\\
   The programs should pront some messages and terminate themself.
\end{enumerate}


\chapter{Configuration}
\section{Available Options}
\begin{itemize}
\item Target platform: standard linux PC or Raspberry Pi
\item additional device support like PEAK CAN-adapter, I2C-devices,...
\end{itemize}
For details see the {\em HELP} pages in the \verb|make menuconfig| 
entries.

\section{Rebuild of the OpenPEARL Environment}
The tuning the installation must be done as administrator:
\begin{enumerate}
\item \verb|cd /home/userx/OpenPEARL/openpearl-code|
\item change the option upon your demands:\\
      \verb|make menuconfig| 
\item \verb|make install|
\end{enumerate}

\section{Configuration Option for the individual Application}
\import{../../common/platformManual/}{configurationCommon.tex}
\import{./}{log.tex}
\import{./}{pearlrc.tex}

\chapter{Release Information}
\section{Supported Language Elements}
The OpenPEARL project aims to support all language elements  as
described in the language report.

This version does not support:
\begin{itemize}
\item STRUCT
\item TYPE
\item TFU  
\item ...??? fehlt noch was
\item Semantic analysis for good diagnostic error messages.
     In case of problems error messages from the c++ compiler may appear. 
\item semantic analysis for program quality analysis (e.g. unused variables)
\end{itemize}

\section{Restrictions for privileged and unprivileded users}
\begin{itemize}
\item The priority based scheduler of the linux system is not
   accessible for normal users. Thus application will operate for 
   normal users with
   the usual completely fair scheduler. The priority values are 
   not regarded in the scheduling.
\item If the application is run with root priviledges, the socalled 
  round-robin scheduler is used. This provides only 100 priority values.
  For administrative reasons, some priorities are clipped by the OpenPEARL
  system.
  The remaining priorities are: 1-97 and 255
\end{itemize}

\chapter{Supported Devices}
\import{../../common/platformManual/}{devicesCommon.tex}
\import{./}{devices.tex}

\chapter{Supported Interrupt Sources}
\import{../../common/platformManual/}{interruptCommon.tex}
\import{./}{interrupts.tex}

\chapter{Available PEARL Signals}
The OpenPEARL signals are organized in some groups.

\import{../../common/}{signallist.tex}


\chapter{How-To Add new System Elements}
\import{../../common/platformManual/}{addDeviceCommon.tex}
\import{./}{addDevice.tex}
\import{../../common/platformManual/}{addInterruptCommon.tex}
\import{./}{addInterrupt.tex}
\import{../../common/platformManual/}{addSignalCommon.tex}
\end{document}

