\section{Console}
The system device \verb|Console| allows adressing in input tasks
and some system diagnostics.

The device operates upon \verb|StdOut| and \verb|StdIn|. 
It is recommended that theses devices are not used simultaneously.

\begin{tabular}{ll}
Synopsis: & \verb|Console| \\
\end{tabular}

Sample usage:
\begin{PEARLCode}
...
SYSTEM;
  console: Console;

PROBLEM;
   SPC console DATION INOUT SYSTEM ALPHIC;
   DCL con     DATION INOUT ALPHIC DIM(*,80) CREATED(console);

ttt: TASK;
   PUT 'hallo' TO con BY A, SKIP;
...
END;

input1: TASK;
   DCL x FIXED;
   ...
   GET x FROM con BY F(6), SKIP;
END

input2: TASK;
   DCL x CHAR(6);
   ...
   GET x FROM con BY A(6), SKIP;
END

...
\end{PEARLCode}


\paragraph{Note:}\ 
\begin{itemize}
\item Supported attributes: FORWARD, INOUT
\item The first input at all must be prefixed with the task name
 quoted with colons. Eg. \verb|:input1:13|. Otherwise the error message
 \verb|:???: no default task| is emitted.
\item subsequent input are directed automatically to the same task
\item if the adressed task is not waiting for input from the console an
  error message is shown \verb|:???: not waiting|. The next input must be 
  prefixed with a task name.
\item outputs are blocked and queued as long as an input is in progress
\item system control is possible via input lines startinmg with slash (/).
  The available commands may be obtained with the command \verb|/HELP|
\item if adata with leading slash should  be entered to an appilication
   the taskname must be prefixed quoted with colons (e.g. \verb|:input2:/foo|)
\item line edit features are supported with 
   insert on/off, cursor movement left and right, delete, backspace keys
\item UTF-8 characters are processed. The user is responsible that 
   the two byte characters are not separated by too small input buffers.
   Regard, that e.g. the character "a consists of  two characters in the 
   scope of OpenPEARL.
\item try \verb|consoleDemo.prl|  in the OpenPEARL demo folder of your installation
\end{itemize}

\section{Disc}
The system device \verb|Disc| allows the access of files in the system.

Synopsis: \verb|Disc(path, nbr)|

\begin{description}
\item [path] specifies the system path of the directory which should contain
    the files defined by the {\em IDF} parameter in the {\em OPEN} statement
\item[nbr] specifies the number of concurrent opened files with this dation
     specified by {\em CREATED(...)} in the userd taion declaraction.
\end{description}

Supported attributes: FORWARD, DIRECT, IN, OUT, INOUT

Multiple dations may by declared on the same path.


Sample Usage:
\begin{PEARLCode}
...
SYSTEM;
   disc: Disc('/tmp/' ,10); ! linux specific disc definition

PROBLEM;
   SPC disc DATION INOUT SYSTEM ALL;
   !
   ! define several user dation upon the same system dation
   DCL logBook DATION OUT ALPHIC FORWARD DIM(*,80)
                      STREAM NOCYCL CREATED(disc);
   DCL dataBase DATION INOUT FIXED DIRECT DIM(10,10) 
                      NOSTREAM NOCYCL CREATED(disc);
...
ttt: TASK;
   DCL record FIXED(10);

   OPEN logBook BY IDF('log.txt'), ANY;
   PUT NOW,'Tast ttt started' TO logBook BY T(13,1),X,A,SKIP;
...
   OPEN dataBase BY IDF('table.dat'),OLD;
   ! read 10 FIXED values
   READ record FROM dataBase BY POS(2,1);
   record(1) := record(1) + 1;
   ! update first value in 2nd line
   WRITE record(1) TO dataBase BY POS(2,1);
...
   CLOSE dataBase;
   CLOSE logBook;
...
END;
...
\end{PEARLCode}

\section{Standard Streams: StdIn, StdOut, StdError}
The system devices \verb|StdIn|,
\verb|StdOut| an \verb|StdError| allows the access of UNIX standard streams
stdin, stdout and stderr.

\begin{tabular}{ll}
Synopsis: & \verb|StdIn| \\
          & \verb|StdOut| \\
          & \verb|StdError| \\
\end{tabular}

Sample usage:
\begin{PEARLCode}
...
SYSTEM;
  stdout: StdOut;

PROBLEM;
   SPC stdout DATION OUT SYSTEM ALPHIC;
   DCL con    DATION OUT ALPHIC DIM(*,80) CREATED(stdout);

ttt: TASK;
   PUT 'hallo' TO con BY A, SKIP;
...
END;
...
\end{PEARLCode}

\paragraph{Note:}\ 
\begin{itemize}
\item Supported attributes: FORWARD, IN, OUT --- IN and OUT depend on the
particular stream.
\item Only one system device shall be defined per standard stream to avoid
intermixing of input and output.
\end{itemize}

\section{Pipe}
The system dation {\em Pipe} is useful for testing and interaction with
other applications. The dation provides access to named pipes.
The linux pipe blocks write attempts, if no reading file descriptor
is opend on the pipe. 
A dummy open with read may be achieved with the option OPEN1.

Synopsis: \verb|Pipe(path, nbr, params)|

\begin{description}
\item [path] is a character string,
     which specifies the system path of the fifo file
    of the named pipe
\item[nbr] specifies the number (as int) of concurrent opened files with this dation
     specified by {\em CREATED(...)} in the userdation declaration
\item [params] is a string value for detailed control of the behavior
     at creation and termination. The parameters are searched in the string
     case sensitive. The sequence of the parameters is artificial.
   \begin{description}
     \item[OLD] the entry in the file system must exist before starting
     \item[NEW] the entry in the file system must not exist at starting
     \item[ANY] the entry may exist or will be created else; an existing
                file syste entry must by of type FIFO (default).
     \item[OPEN1] specifies that one file descriptor should be opened 
                automatically.
     \item[CAN] the file system entry shall be deleted at the end of the 
                application
     \item[PRM] the file system entry shall remain at the end of the
                application (default)
   \end{description} 
   The permissions are set to \verb|ugo=rwx| for new file system entries.
\end{description}

Sample usage:
\begin{PEARLCode}
...
SYSTEM;
   pipe: Pipe('/tmp/pipefile', 2, 'OPEN1 ANY CAN');
...
PROBLEM;
   SPC pipe DATION INOUT SYSTEM ALL;
   DCL writeFormatted DATION OUT ALPHIC FORWARD DIM(*,80) CREATED(pipe);
   DCL readInternal   DATION IN  CHAR   FORWARD DIM(*,80) CREATED(pipe);

producer: TASK;
   DCL floatVal FLOAT INIT(3.1415);

   OPEN writeFormatted;
   PUT  floatVal TO writeFormatted BY F(6,3), SKIP;
...
END;

consumer: TASK;
   DCL line CHAR(80);
   OPEN readInternal;
   READ line FROM readInternal;
   IF line ==' 3.142' THEN
      PUT 'ok' TO console BY A, SKIP;
   ELSE
      PUT 'wrong format' TO console BY A, SKIP;
   FIN;
...
END;
...
\end{PEARLCode}

\paragraph{Notes:} \
\begin{itemize}
\item Supported attributes: FORWARD, IN, OUT
\item The declaration of multiple pipe devices upon the same system file shall
\item The current user needs read/write permissions on the file system entry
   denoted by \texttt{path}.
be avoided.
\end{itemize}

\section{RPiDigitalOut, RPiDigitalIn}
On Raspberry Pi systems the general purpose input and output bits (GPIO)
may be used as digital inputs and putputs.

The devices RPiDigitalIn and  RPiDigitalOut provides 
an access to digital input and output bits
on the GPIO-bits of the raspberry pi.

\begin{tabular}{ll}
Synopsis: & \verb|RPiDigitalOut(start, width)| \\
          & \verb|RPiDigitalIn(start, width, pud)| \\
\end{tabular}

\begin{description}
\item[start] specifies the number of leftmost bit of the group. Valid numbers
     depend on the model of the raspberry pi. The starting number if
     the leftmost gpio bit number (large value)
\item [width] specifies the number of bits which are grouped together.
     Valid numbers depend on the model of the raspberry pi.
\item [pud] Pull-up / pull-down configuration.
    Valid values are 'u' for pull up, 'd' for pull down and 
    'n' for no pull-up/pull-down resistor. This option is only
    available for digital inputs. The value of the pullk up/down resistor is
    approx. $50 k \Omega$
\end{description}

The access to the dation is done by writing a BIT(width)-value to the
opened dation. 
Writing other data than BIT-types  may cause a signal, if the size of the
data does not match the width parameter. 


Supported bits:

\begin{tabular}{|l|l|}
\hline
Model & Bits \\
\hline
1B & 2,3,4,7,8,9,10,11,17,18,22,23,24,25,27,28,29,30,31 \\
\hline
2B,3B & 2-27 (except 14, 15) \\
\hline
\end{tabular}  

Sample usage:
\begin{PEARLCode}
...
SYSTEM;
  led1 : RPiDigitalOut(8,1);

  ! enable the pull-up resistor and connect the switch between 
  ! GPIO bit 9 and GND
  switch: RPiDigitalIn(9,1,'u');

PROBLEM;
   SPC led1 DATION OUT SYSTEM BASIC BIT(1);
   DCL uLed DATION OUT BASIC BIT(1) CREATED(led1);

   SPC switch1 DATION IN SYSTEM BASIC BIT(1);
   DCL uSwitch DATION IN BASIC BIT(1) CREATED(switch1);
...
ttt: TASK;
   DCL b BIT(1);

   OPEN uLed;
   OPEN uSwitch;
   TAKE b FROM uSwitch;
   IF b THEN
      PUT 'switch is open' TO console BY A, SKIP;
   ELSE
      PUT 'switch is set short' TO console BY A, SKIP;
   FIN;
   SEND b TO uLed;  ! echo switch status
...
END;
...
\end{PEARLCode}

\paragraph{Note:}  
\begin{itemize}
\item It is checked that a i/o-bit is not used by different declarations within
one application.
\item the Broadcom gpio block supports the access to the bits only 
via set and clear registers. Thus the writing of a BIT(x) value with x $>$ 1
need two sequential operations to clear and set the output bits, which lead
to a short period of an intermediate value. E.g. '10' shall be changed to '01'. Bit 1 will be cleared leading to the output value '00' and bit 2 
is set afterwards leading to '01'.

\item Bits 14 and 15 are used by the kernel for the 
  system serial console during system start. If this does not matter,
  they can be used at runtime.
\item Other bits become not available, if 
  the i2c and/or spi is enabled in the
  raspberry configuration.
\item the current user needs to access the system device 
  \verb|/dev/gpiomem|. This may be achieved by running as root, or setting
  the permissions for \verb|/dev/gpiomem| accordingly.
\end{itemize}
 
\section{I2CBus}
Linux supports normal access to an i2c bus via a device node,
 if i2c support is configured and the corresponding devices are installed.
The corresponding device file are named like \texttt{/dev/i2c-0}.

This system entry may not be used in the problem part. This element
provides the access to an I2C bus for the supported I2C devices like 
PCF8574 or LM75.

\begin{tabular}{ll}
Synopsis: & \verb|I2CBus(deviceName)|\\ 
\end{tabular}

\begin{description}
\item[deviceName] specifies the name to the device file (e.g. '/dev/i2c-0')
\end{description}

Sample usage:
\begin{PEARLCode}
...
SYSTEM;
   led1: PCF8475Out('21'B4, 5,1) --- I2CBus('/dev/i2c-0');

   i2cbus2: I2CBus('/dev/i2c-1');
   in1: PCF8475In('21'B4,7,1) --- i2cbus2;
   in2: PCF8475In('21'B4,6,1) --- i2cbus2;
   in3: PCF8475In('21'B4,5,1) --- i2cbus2;
   t1: LM75('48'B4) --- i2cbus2;
   t2: LM75('49'B4) --- i2cbus2;
...
\end{PEARLCode}

\paragraph{Note:}
\begin{itemize}
\item The I2C bus speed must be set at system level depending on the 
i2c device module.
\item if more than one i2c device is used on the same bus, a symbolic
   links must be used in the system part
%Some I2C-devices need the {\em repeated start} feature 
%of the I2C specification.
%This feature is not supported by all i2c bus drivers. The presence is
%checked.
\end{itemize}

\section{PEAK CAN Bus Interface}
The CAN bus interface from the company PEAK provides the access via a character 
device node (e.q. \texttt{/dev/pcan32} and via SocketCAN.
The socket CAN infrastructure allows no change of the communication setting
like bitrate for normal users --- the character device from PEAK provides
such features. 

After installation of the PCAN device driver on the system as character 
device, the corresponding device nodes and libraries are installed 
in the system (e.g. \verb|/dev/pcan32|). Refere to the installation guide 
from PEAK about the installation procedure.


\begin{tabular}{ll}
Synopsis: & \verb|PCan(deviceName, speed)|\\ 
\end{tabular}

\begin{description}
\item[deviceName] specifies the name to the device file (e.g. '/dev/pcan32')
\item[bus speed] specifies the bus speed. 
Only the values 125000, 250000, 500000 and 1000000 bps are supported by 
the OpenPEARL driver.
\end{description}

The transmission data must have the following structure:

\begin{verbatim}
   STRUCT [
     identifier FIXED(11);  ! the message id (0..1023) 
     rtrRequest BIT(1);     ! '1'B if a RTR frame transmission is requested
     dataLength FIXED(4);   ! 0..8 as number of bytes to read or write
     data(8)    BIT(8);     ! the storage of the read/written data bytes
   ];
\end{verbatim}

Sample usage:
\begin{PEARLCode}
...
SYSTEM;
   can: PCAN('/dev/pcan32', 10000000);

PROBLEM;
   TYPE ... 
   SPC can DATION INOUT SYSTEM ALL;
   DCL uCan DATION INOUT canMessage FORWARD DIM(*) CREATED(can);

ttt: TASK;
   DCL cm canMessage;
   OPEN uCan;
   TAKE cm FROM uCan;
   PUT 'received message id', cm.id TO console BY A, F(4), SKIP;
...
   SEND cm TO uCan;
...
END;
...
\end{PEARLCode}

\section{Tcp/IP Server Socket}
The device provides a TCP/IP server socket.

\begin{tabular}{ll}
Synopsis: & \verb|TcpIpServerSocket(port)|\\ 
\end{tabular}

\begin{description}
\item[port] specifies the port numbe to listen (e.g. 1234)
\end{description}

The corresponding user dation starts waiting for one (1) connection with {\em OPEN}.
Subsequent PUT/GET statements transfer the data between client ansd server.
The {\em CLOSE} statement terminates the connection.

\paragraph{Note:} The system dation is still experimental. 
There is no complete error treatment, especially with connection loss due 
to network issues or remote termination.


